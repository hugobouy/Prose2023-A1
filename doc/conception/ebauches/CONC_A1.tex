%-----------------------------------------------
%   LaTeX template
%   Crée par: Thorkel-dev
%-----------------------------------------------

\documentclass[a4paper,11pt,titlepage,dvipsnames]{article}

\usepackage{lipsum} % Pas trop compris à quoi ça sert 

\usepackage[
            imageFooterLeft      =  {latex.png},                     % Optionnel
            imageHeaderLeft      =  {eseo.png},                      % Optionnel
            %subject              =  {{Subject of the document}},    % Optionnel
            %keywords             =  {{Example, Template, LaTeX}},   % Optionnel
            language             =  {french},                      
            % encoding             =  {},                            % Defaut utf8
            ]{Parameter}

\usepackage[
            logoRight       =   {ST-logo.png},                          % Optionnel
            logoLeft        =   {eseo.png},                             % Optionnel
            %logoCenter      =   {latex.png},                           % Optionnel
            subject         =   {{Incrément 2}},                        % Optionnel
            projectName     =   {PSC~(Prototype~Sonnette~Connectée)},   % Optionnel
            enterpriseName  =   {STMicroelectronics},                   % Optionnel
            responsableName =   {Hugo~BOUY},                            % Optionnel
            etatDocument    =   {Version~finale},                       % Optionnel
            % revision        =   {},                                   % Optionnel
            version         =   {2.3},                                  % Optionnel
            ]{FrontPage}

\makeglossaries
\loadglsentries{glossary.tex} % Charge le glossaire
\loadglsentries{acronyms.tex} % Charge les acronymes
\addbibresource{biblio.bib}   % Charge la bibliographie

%-----------------------------------------
%      Variables
%-----------------------------------------
%Exemple : le projet \projet~est réalisé par l'équipe \groupe~pour l'entreprise \client.
%/!\Un '~' est obligatoire pour créer un espace insécable après l'utilisation d'une macro.
\author{Prose-2024-A1}        % Auteur
\title{Dossier de conception}     % Titre
\date{\normalsize\today}    % Date actuelle, automatique
\newcommand{\client}{STMicroelectronics}
\newcommand{\groupe}{Prose-2024-A1}
\newcommand{\amodifier}{{\it À compléter et/ou modifier}}

\newcommand{\projet}{PSC (Prototype Sonnette Connectée)}

\newcommand{\refSpec}{[V2\_SPEC\_A1]}
\newcommand{\refPAQL}{[PAQL\_A1]}

\newcommand{\appliC}{SoftSonnette et SoftPorte}
\newcommand{\appliLin}{SoftSonnette}
\newcommand{\appliPo}{SoftPorte}
\newcommand{\appliA}{AOP}

\newcommand{\actT}{Testeur}
\newcommand{\actD}{Démonstrateur}
\newcommand{\IA}{RecognitionAI}

\graphicspath{{../schemas/}{../figures/}{../schemas/anime_uml_generated}{../schemas/anime_uml_detailed_sp_generated}{../schemas/anime_uml_detailed_ss_generated}{../schemas/anime_uml_detailed_aop_generated}}

%Hyphenation
\hyphenation{firstName}
\hyphenation{workingHours}
\hyphenation{SoftSonnette}
\hyphenation{DoorManager}
\hyphenation{PostmanAOP}
\hyphenation{ConnectionManager}

%-----------------------------------------
%      Other configurations
%-----------------------------------------

% Exemple d'arrière plan
\backgroundsetup{
    scale=1,
    opacity=0.1,
    angle=0,
    contents={%
            \includegraphics[width=\textwidth]{eseo.png}
        }%
}

\setlength{\parskip}{0.5em} % Espacement entre les paragraphes
\setlength{\belowcaptionskip}{-0.5em} % Espacement entre les images
%\setlength{\parindent}{0pt} % Pas d'indentation
\setcounter{tocdepth}{4}

%-----------------------------------------
%      Document
%-----------------------------------------

\begin{document}

\maketitle

\BgThispage % Utiliser l'arrière plan
\vspace*{\fill}
\noindent
AVERTISSEMENT : \\
Le présent document est un document à but pédagogique.
Il a été réalisé sous la direction de Jérôme DELATOUR, en collaboration avec des enseignants et les étudiants de l'option SE, groupe A1 (Hugo BOUY, Bastien CASSAR, Paul CHIRON, Paul JURET, Laurent LETENNEUR, Mathis MOULIN, Romain TROVALLET) du groupe ESEO.
Ce document est la propriété de Jérôme DELATOUR du groupe ESEO. En dehors des activités pédagogiques de l'ESEO, ce document ne peut être diffusé ou recopié sans l'autorisation écrite de ses propriétaires.
\vspace*{\fill}
\clearpage

\section*{Table des versions}
\begin{xltabular}{\linewidth}{|c|X|>{\centering\arraybackslash}X|c|c|}

    \hline \textbf{Date} & \textbf{Actions} & \textbf{Auteur} & \textbf{Version} & \textbf{Révision} \\\hline
    \endfirsthead

    \multicolumn{5}{l}%
    {\textbf{}}\\
    \hline \textbf{Date} & \textbf{Actions} & \textbf{Auteur} & \textbf{Version} & \textbf{Révision} \\\hline
    \endhead

    \multicolumn{5}{r}%
    {\textbf{}}\tabularnewline
    \endfoot
    \endlastfoot
    27/05/2023 & Divers corrections de typos et de cohérence & Romain TROVALLET & 2.5 & 0 \\ \hline
    27/05/2023 & Correction divers (typo, cohérence) & Hugo BOUY & 2.4 & 0 \\ \hline
    27/05/2023 & Ajout de la Pop-up et de la logique vidéo indisponible et suppression de TAM & Hugo BOUY & 2.3 & 1 \\ \hline
    27/05/2023 & Suppression du moteur & Hugo BOUY & 2.3 & 0 \\ \hline
    25/05/2023 & Relecture et correction diverse après AN & Hugo BOUY et Romain TROVALLET & 2.2 & 0 \\ \hline
    25/05/2023 & Correction des CU et des fonctions après AN & Hugo BOUY & 2.1 & 1 \\ \hline
    24/05/2023 & Ajout descriptions manquantes dans contexte physique & Hugo BOUY & 2.1 & 0 \\ \hline
    23/05/2023 & Relecture et validation pour AN Spec du 24/05/2023 & Toute l'équipe & 2.0 & 0 \\ \hline
    22/05/2023 & Suppression du terme "Base de données" & Hugo BOUY & 1.5 & 1 \\ \hline
    22/05/2023 & Mise à jour du CU Ouvrir Porte & Hugo BOUY & 1.5 & 0 \\ \hline
    22/05/2023 & Mise à jour des CU avec la communication entre AOP et SS & Hugo BOUY & 1.4 & 1 \\ \hline
    22/05/2023 & Mise à jour du CU Initialiser Board & Hugo BOUY & 1.4 & 0 \\ \hline
    22/05/2023 & Mise à jour CU Consulter Calendrier + CU Reconnaître Visage & Romain TROVALLET & 1.3 & 0 \\ \hline
    09/05/2023 & Correction avant Incrément 2 & Romain TROVALLET & 1.2 & 0 \\ \hline
    21/03/2023 & Correction des CUs après AC & Laurent LETENNEUR & 1.1 & 3 \\ \hline
    20/03/2023 & Correction architecture matérielle après AC & Hugo BOUY & 1.1 & 2 \\ \hline
    18/03/2023 & Màj des références documentaires & Paul CHIRON & 1.1 & 1 \\ \hline
    17/03/2023 & Corrections mineurs en live de l'AC Spec & Paul JURET et Romain TROVALLET & 1.1 & 0 \\ \hline
    16/03/2023 & Relecture et validation pour AC Spec du 17/03/2023 & Toute l'équipe & 1.0 & 0 \\ \hline
    15/03/2023 & Màj archi matérielle et description générale après consulting & Hugo BOUY & 0.9 & 1 \\ \hline
    15/03/2023 & Ajout des parties 2.4, 2.5 et 2.6 & Paul JURET & 0.9 & 0 \\ \hline
    15/03/2023 & Modification des CUs & Laurent LETENNEUR & 0.8 & 7 \\ \hline
    15/03/2023 & Relecture CUs & Paul CHIRON & 0.8 & 6 \\ \hline
    15/03/2023 & Modification contexte physique & Paul JURET & 0.8 & 5 \\ \hline
    15/03/2023 & Ajout de mots aux dictionnaire du domaine et des hyperliens & Bastien CASSAR & 0.8 & 4 \\ \hline
    13/03/2023 & Correction des CUs, mise en page et dictionnaire de domaine & Hugo BOUY & 0.8 & 3 \\ \hline
    12/03/2023 & Modification des CUs & Laurent LETENNEUR & 0.8 & 2 \\ \hline
    12/03/2023 & Modification bibliographie et dictionnaire de domaine & Paul CHIRON & 0.8 & 1 \\ \hline
    12/03/2023 & Modifications et finitions contexte physique & Mathis MOULIN & 0.8 & 0 \\ \hline
    12/03/2023 & Relecture des CUs & Paul CHIRON & 0.7 & 1 \\ \hline
    11/03/2023 & Ajout des CUs & Laurent LETENNEUR & 0.7 & 0 \\ \hline
    10/03/2023 & Relecture description générale et caractéristiques des acteurs & Hugo BOUY & 0.6 & 1 \\ \hline
    09/03/2023 & Modifications et finitions contexte logique & Bastien CASSAR & 0.6 & 0 \\ \hline
    08/03/2023 & Modification Intro et Acteurs & Paul CHIRON & 0.5 & 1 \\ \hline
    08/03/2023 & Ajout partie description générale & Paul JURET & 0.5 & 0 \\ \hline
    08/03/2023 & Modification IHM et Modification MàE & Bastien CASSAR & 0.4 & 2 \\ \hline
    07/03/2023 & Modification acronymes et dictionnaire de domaine & Paul CHIRON & 0.4 & 1 \\ \hline
    07/03/2023 & Ajout partie architecture matérielle & Hugo BOUY & 0.4 & 0 \\ \hline
    06/03/2023 & Ajout des descriptions générale et textuelles & Mathis MOULIN & 0.3 & 2 \\ \hline
    06/03/2023 & Ajout d'une IHM & Bastien CASSAR & 0.3 & 1 \\ \hline
    05/03/2023 & Ajout partie Acteurs & Paul CHIRON & 0.3 & 0 \\ \hline
    05/03/2023 & Ajout partie Intro & Paul CHIRON & 0.2 & 0 \\ \hline
    04/03/2023 & Ajout partie IHM et Modification MàE & Bastien CASSAR & 0.1 & 0 \\ \hline
    02/03/2023 & Création du document & Paul CHIRON & 0.0 & 0 \\ \hline
    
\end{xltabular}
 \label{TableOfVersion} % Ajoute la table des versions
\clearpage

\addtocontents{toc}{\protect\setlength{\parskip}{0pt}}
\tableofcontents % Ajoute la table des matières

%-----------------------------------------
%      Partie I
%-----------------------------------------

\section{Introduction}

\subsection{Objet}

Ce dossier de conception a pour objectif de rassembler toute la conception du logiciel PSC. 
Il permettra à l’équipe A1 de développer le logiciel ainsi qu’élaborer des tests.

Les éléments de conception présentés dans ce document ont été déterminés suite à l’étude du dossier de spécification \refSpec.

Ce dossier de conception suit les recommandations de la norme IEEE 29148 [IEEE-29148:2018]. 
Il utilise des schémas et illustrations respectant la norme UML en version 2.5.1 [UML\_2.5.1\_2017].
Il respecte les exigences du Plan d’Assurance Qualité Logicielle (PAQL) défini par l’équipe A1 \refPAQL.


\subsection{Portée}

Ce document décrit les éléments de conception du Système à l'Étude (SàE). Il est destiné :

\begin{itemize}
    \item À l'équipe de développement C et celle de développement Android afin de préciser l'implémentation des objets constituant le SàE.
    \item Aux testeurs, afin qu'ils puissent élaborer les tests adéquats vérifiant la philosophie de conception adaptée par l'équipe.
    \item Aux auditeurs de la société FORMATO lors de leurs différents suivis du projet.
    \item Au client pour que le cadre du projet et la direction prise par l'équipe soient claires et dans la continuité des spécifications.
\end{itemize}



\newpage
\newpage
\setglossarysection{subsection}
\printglossary[type=\acronymtype,style=longborder ,title={Définitions, acronymes et abréviations}]
\glsaddallunused

\subsection{Références}

\begin{table}[H]
    \centering
    \begin{tabularx}{\textwidth}{|X|X|}
      \hline 
        \textbf{Document}  & \textbf{Référence}   \\
        \hline
        [ISO/IEC/IEEE 29148:2018]               & ISO/IEC/IEEE, “International Standard, Systems and software engineering — Life cycle processes — Requirements engineering”, version du 30/11/2018
        
        https://standards.ieee.org/standard/29148-2018.html \\ 
        \hline
        [UML\_2.5.1\_2017]                      & OMG, “Unified Modeling Language”, version 2.5.1 de décembre 2017                                                                                                                           \\ 
        \hline
        [PAQL\_A1]                             & \og Plan d'Assurance Qualité Logicielle \fg version 4.0 du 15/05/2023. 
        
        Dépôt SE 2024 A1 - ST.doc, chemin d'accès: /qualite/livrable/PAQL\_A1.pdf        \\ 
        \hline
        [ST\_Sonnette\_connectée\_industrielle\_2023] & \client, \og Cahier des charges initial pour le projet \fg, version original du 09/02/2023.
        
        Dépôt SE 2024 A1 - ST.doc, chemin d'accès: /gestion\_de\_projet/client/ ST\_Sonnette\_connectée\_industrielle.pdf               \\ 
        \hline                                  
        [CR\_R2\_09\_02\_2023]                  & \og Compte rendu de la réunion du 09/02/2023 avec les clients \fg.
        
        Dépôt SE 2024 A1 - ST.doc, chemin d'accès: /gestion\_de\_projet/réunions/ CR\_R2\_09\_02\_2023.pdf               \\ 
        \hline
    \end{tabularx}
    \caption{Références des documents utilisés dans ce dossier}
    \label{tableau-references}
  \end{table}
  

\subsection{Vue d'ensemble}
Ce document de spécification est structuré en trois parties :
\begin{itemize}
      \item La partie I définit l’objet et la portée du document ainsi qu’une liste des abréviations utilisées dans ce document et les références des documents cités.
      \item La partie II, intitulée \og description générale \fg{}, a pour objectif de présenter l'environnement et le contexte du PSC, ainsi que les fonctionnalités principales attendues pour "SoftSonnette", "SoftPorte" et "AOP".
      \item La partie III présente en détail les IHM attendues du PSC, les fonctionnalités détaillées ainsi que le dictionnaire du domaine.
\end{itemize}

Les mots en italiques sont des liens symboliques.
Certains nom d'écran des IHM sont également des liens symboliques.
Le symbole \og * \fg est un lien vers la description d'une fonction.

 % Exemple d'inclusion de fichier

%-----------------------------------------
%      Partie II
%-----------------------------------------

\section{Conception générale}

\subsection{Architecture candidate}

Dans le diagramme suivant est présenté l'architecture candidate du SàE. 
Les différents objets sont présentes ainsi que leurs méthodes.

\begin{figure} [H]
    \centering
    \includegraphics[width=\textwidth]{architecture_candidate_etendue}
    \caption{Architecture candidate}
    \label{Archicandidate}
\end{figure}

Lors de la partie conception générale, est faite l'hypothèse d'un système matériel à ressources infinies.
\newpage
On distingue dans la représentation de la figure \ref{Archicandidate} les 3 grandes entités du SàE : AOP, SoftSonnette et SoftPorte.

\begin{itemize}
    \item AOP est l'application  développée pour le téléphone et à destination du Démonstrateur. Elle contient les objets suivants :
    \begin{itemize}
        \item[$-$] Connection Manager : Permet de gérer la connexion avec SoftSonnette.
        \item[$-$] GUI : Interface graphique permettant à l'utilisateur d'interagir avec AOP.
    \end{itemize}
\end{itemize}

\begin{itemize}
    \item SoftSonnette est l'application exécutée sur le microprocesseur. Elle contient les objets suivants :
    \begin {itemize}
        \item[$-$] Employee Manager : Permet de gérer les données des employés.
        \item[$-$] Clock : Permet à SoftSonnette et AOP de synchroniser leur heure.
        \item[$-$] Cameraman : Permet de gérer le flux vidéo.
        \item[$-$] Guard : Vérifie le mot de passe pour la connexion entre SoftSonnette et AOP.
        \item[$-$] Bouncer : Permet de gérer les accès aux Testeurs se présentant devant la Sonnette.
        \item[$-$] UISS : Interface graphique permettant à l'utilisateur d'interagir avec SoftSonnette.
        \item[$-$] \IA : Permet d'interfacer la librairie de reconnaissance faciale propriétaire du client. 
        La librairie n'étant pas fournie à l'équipe projet, il est de la responsabilité du client d'implémenter dans cet objet les différents appels à sa librairie pour permettre le bon fonctionnement du PSC.
        L'équipe de développement implémente cet objet en mode "boîte noire".
    \end{itemize}
\end{itemize}

\begin {itemize}
    \item SoftPorte est l'application exécutée sur le microcontrôleur. Elle contient les objets suivants :
    \begin{itemize}
        \item[$-$] Door Manager : Permet le contrôle de la porte simulée.
        \item[$-$] UISP : Interface physique permettant à l'utilisateur de visualiser l'état de SoftPorte.
    \end{itemize}
\end{itemize}

\newpage

\subsection{Grands principes de fonctionnement}

Cette partie représente le fonctionnement général de SoftSonnette, SoftPorte, AOP et leurs interactions respectives.
Les diagrammes de séquences présentés sont : 
\begin {itemize}
    \item Présenter les capacités de la STM32MP15 au travers d'une application de Sonnette Connectée
    \item Initialiser Board
    \item Se connecter
    \item Demander à entrer
    \item Regarder vidéo
    \item Consulter calendrier
    \item Contrôler Porte à distance
    \item Consulter liste des employés
    \item Quitter SàE
\end {itemize}

Les diagrammes de séquence présentent l'histoire nominal des Cas d'Utilisation du dossier de spécification \refSpec. 
Ils ne traitent donc pas les erreurs de fonctionnement.

\newpage

\subsubsection{Présenter les capacités de la STM32MP15 au travers d'une application de Sonnette Connectée} %2.2.1
Ce diagramme représente le scénario nominal de l'ensemble des CUs du dossier de spécification \refSpec.
Il s'agit du CU Stratégique du SàE.\\

\begin{figure} [H]
    \centering
    \includegraphics[scale=.5,max width=\textwidth,max height=.9\textheight]{architecture_candidate_anime_uml-sequence-CU_Strategique}
    \caption{Diagramme de séquence du scénario nominal}
    \label{CU-Stratégique}
\end{figure}

\newpage

\subsubsection{Initialiser Board} %2.2.2
Ce diagramme représente le scénario nominal du CU "Initialiser Board" dans le dossier de spécification \refSpec.\\

\begin{figure} [H]
    \centering
    \includegraphics[scale=.5,max width=\textwidth,max height=.9\textheight]{architecture_candidate_anime_uml-sequence-CU_InitialiserBoard}
    \caption{Diagramme de séquence de l'initialisation de Board}
    \label{CU-InitialiserBoard}
\end{figure}

\newpage

\subsubsection{Se connecter} %2.2.3
Ce diagramme représente le scénario nominal du CU "Se connecter" dans le dossier de spécification \refSpec.\\

\begin{figure} [H]
    \centering
    \includegraphics[scale=.5,max width=\textwidth,max height=.9\textheight]{architecture_candidate_anime_uml-sequence-CU_SeConnecter}
    \caption{Diagramme de séquence de la connexion entre AOP et SoftSonnette}
    \label{CU-SeConnecter}
\end{figure}

\subsubsection{Demander à entrer} %2.2.4
Ce diagramme représente le scénario nominal du CU "Demander à entrer" dans le dossier de spécification \refSpec.\\

\begin{figure} [H]
    \centering
    \includegraphics[scale=.5,max width=\textwidth,max height=.9\textheight]{architecture_candidate_anime_uml-sequence-CU_Demander_Entrer}
    \caption{Diagramme de séquence du CU "Demander à entrer" }
    \label{CU-Entrer}
\end{figure}

\subsubsection{Ouvrir Porte} %2.2.5
Ce diagramme représente le scénario nominal du CU "Ouvrir Porte" dans le dossier de spécification \refSpec.\\

\begin{figure} [H]
    \centering
    \includegraphics[scale=.5,max width=\textwidth,max height=.9\textheight]{architecture_candidate_anime_uml-sequence-CU_OuvrirPorte}
    \caption{Diagramme de séquence du CU "Ouvrir Porte"}
    \label{CU-OuvrirPorte}
\end{figure}

\subsubsection{Regarder vidéo} %2.2.6
Ce diagramme représente le scénario nominal du CU "Regarder vidéo" dans le dossier de spécification \refSpec.\\

\begin{figure} [H]
    \centering
    \includegraphics[scale=.5,max width=\textwidth,max height=.9\textheight]{architecture_candidate_anime_uml-sequence-CU_Regarder_Video}
    \caption{Diagramme de séquence du CU "Regarder vidéo"}
    \label{CU-Regarder_Video}
\end{figure}

\subsubsection{Consulter calendrier} %2.2.7
Ce diagramme représente le scénario nominal du CU "Consulter calendrier" dans le dossier de spécification \refSpec.\\

\begin{figure} [H]
    \centering
    \includegraphics[scale=.5,max width=\textwidth,max height=.9\textheight]{architecture_candidate_anime_uml-sequence-CU_Consulter_Calendrier}
    \caption{Diagramme de séquence du CU "Consulter calendrier"}
    \label{CU-Consulter_Calendrier}
\end{figure}

\subsubsection{Contrôler Porte à distance} %2.2.8
Ce diagramme représente le scénario nominal du CU "Contrôle Porte à distance" dans le dossier de spécification \refSpec.\\

\begin{figure} [H]
    \centering
    \includegraphics[scale=.5,max width=\textwidth,max height=.9\textheight]{architecture_candidate_anime_uml-sequence-CU_Contrôler_Porte_A_Distance}
    \caption{Diagramme de séquence du CU "Contrôle Porte à distance"}
    \label{CU-Contrôler_Porte_A_Distance}
\end{figure}

\subsubsection{Consulter liste employés} %2.2.9
Ce diagramme représente le scénario nominal du CU "Consulter liste employés" dans le dossier de spécification \refSpec.\\

\begin{figure} [H]
    \centering
    \includegraphics[scale=.5,max width=\textwidth,max height=.9\textheight]{architecture_candidate_anime_uml-sequence-CU_Consulter_Liste}
    \caption{Diagramme de séquence du CU "Consulter liste employés"}
    \label{CU-Consulter_Liste}
\end{figure}

\subsubsection{Quitter SàE} %2.2.10
Ce diagramme représente le scénario nominal du CU "Quitter SàE" dans le dossier de spécification \refSpec.\\

\begin{figure} [H]
    \centering
    \includegraphics[scale=.5,max width=\textwidth,max height=.9\textheight]{architecture_candidate_anime_uml-sequence-CU_QuitterSAE}
    \caption{Diagramme de séquence du CU "Quitter SàE"}
    \label{CU-Quitter}
\end{figure}

\newpage

\subsection{Description des composants}

\subsubsection{Description des types manipulés entre composants}%2.3.1

\begin{itemize}
    \item ConnectionStatus : Énumération représentant l'état de la connexion entre \appliA~et \appliLin.
    ConnectionStatus peut prendre pour valeur : PASS\_KO, CONNECT\_KO et ALL\_OK.
    \item Day : Énumération représentant un jour de la semaine.
    Day peut prendre pour valeur : MONDAY, TUESDAY, WEDNESDAY, THURSDAY, FRIDAY, SATURDAY et SUNDAY.
    \item \hyperlink{emp}{\textit{Employee}} : Structure contenant les informations de l'employé.
    Employee contient : firstName , name, role (Role), id (EmployeeID), workingHours~[~]~[~] et picture (chemin local vers l'image).
    \begin{itemize}
        \item \hyperlink{prenom}{\textit{firstName}} est une chaîne de caractère comprenant entre 2 et 12 caractères alphanumériques encodés en UTF-8.
        \item \hyperlink{nom}{\textit{name}} suit les mêmes règles que firstName à la différence qu'il peut être comprit entre 1 et 12 caractères alphanumériques.
        \item workingHours~[~]~[~] est une matrice de 7 lignes (jours de la semaine) et 2 colonnes (horaires début et fin de journée).
    \end{itemize}    
    \item \hyperlink{idEmp}{\textit{EmployeeID}} : Un entier qui représente le numéro d'un employé allant jusqu'à MAX\_EMPLOYE qui vaut 10. La constante "UNKNOWN" vaut -1, et signifie que l'employé n'est pas reconnu.
    \item \hyperlink{hor}{\textit{Hour}} : Structure contenant un horaire de début ou de fin de journée, enregistrée au format 24 h avec une précision de 30 minutes.
    Hour contient l'heure et la minute de l'horaire.
    \item \hyperlink{IP}{\textit{Ip}} : Chaîne de caractères de type IPV4 au format X.X.X.X où X est un entier compris entre 0 et 255. 
    Correspond à l'adresse de la Board.
    \item \hyperlink{mdp}{\textit{Password}} : Chaîne de caractères. Les caractéristiques sont définies dans le dictionnaire des domaines.
    \item \hyperlink{photo}{\textit{Picture}} : Image du visage d'un employé qui sera utilisée pour la reconnaissance faciale.
    \item PopUpID : Énumération représentant les différentes pop-up affichables sur \appliA.
    PopUpID peut prendre pour valeur : WAITING\_ID, ERROR\_PASS\_ID, ERROR\_CONNECT\_ID, DELETE\_ID et VIDEO\_NOT\_AVAILABLE.  
    \item \hyperlink{rol}{\textit{Role}} : Énumération représentant les différents types d'employés pour délimiter leurs horaires.
    Role peut prendre pour valeur : E\_MORNING, E\_DAY, E\_EVENING, E\_SECURITY et E\_SPECIAL.
    \item ScreenID : Énumération représentant les différents écrans affichables sur \appliA.
    ScreenID peut prendre pour valeur : STARTER\_ID, HOME\_ID, ADD\_EMPLOYEE\_ID, ADD\_SPECIAL\_EMPLOYEE\_ID, EMPLOYEE\_LIST\_ID, DOOR\_CONTROL\_ID, CALENDAR\_ID et VIDEO\_ID.
    \item Time : respecte la norme ISO 8601 : YYYY-MM-DD-HH: MM: SS.
    Time se met à jour lors de la connexion entre \appliA~et \appliLin.
    \item SSState : Énumération des affichages de \appliLin.
    SSState peut prendre pour valeur: STATE\_IDLE, STATE\_WEBCAM\_CONNECTED, STATE\_WEBCAM\_NOT\_CONNECTED, STATE\_ERROR\_COM.
    À noter que STATE\_IDLE correspond à l'affichage vide de l'écran de SoftSonnette (sans affichage vidéo et sans label d'erreur).
\end{itemize}

\newpage

\subsubsection{Diagramme de classe}%2.3.2
Est représenté ci-dessous le diagramme de classe de PSC, composé de 11 classes avec pour chacune leurs attributs et leurs méthodes décrites dans la suite du document.

\begin{figure} [H]
    \centering
    \includegraphics[scale=.5,max width=\textwidth,max height=.9\textheight]{diagramme_de_classe}
    \caption{Diagramme de classes du SàE}
    \label{Diagramme_de_classes}
\end{figure}

\newpage

\subsubsection{Description des classes}%2.3.3
    \paragraph{[Object] ConnectionManager}%2.3.3.1
    \begin{figure} [H]
        \centering
        \includegraphics[scale=.5,max width=\textwidth,max height=.9\textheight]{architecture_candidate_anime_uml-class-ConnectionManager.eps}
        \caption{Diagramme de classe représentant ConnectionManager}
        \label{Classe_ConnectionManager}
    \end{figure}
        \subparagraph{Philosophie de conception}%2.3.3.1.1
        La classe ConnectionManager permet de transiter les informations de connexion.
        Elle permet aussi d'assurer que les applications sont connectées et d'agir en conséquence.
        Cette classe interagit avec Guard, Clock, et GUI.
        \subparagraph{Description structurelle}%2.3.3.1.2
            \subsubparagraph{Attributs}%2.3.3.1.2.1
            \begin{itemize}
                \item connectionState : bool : Exprime l'état de la connexion actuelle.
            \end{itemize}
            \subsubparagraph{Services offerts}%2.3.3.1.2.2
            \begin{itemize}
                \item {checkConnection() : void : Tâche de fond qui vérifie si la connexion est toujours opérationnelle.
                Dans le cas d'une perte de connexion cette fonction informe les objets intéressés.}
                \item {getLocalTime() : Time : Récupère l'heure et la date du téléphone}
                \item {askConnection(ip : IP, pass : Password) : void : Envoie les informations de connexion vers Guard.}
                \item {validatePass(passValidated : bool) : void : Reçoit l'information de connexion qui valide ou non la connexion.}
            \end{itemize}
\newpage
        \subparagraph{Description comportementale}%2.3.3.1.3
        La MAE suivante permet d'expliquer le fonctionnement de l'objet actif ConnectionManager :
        \begin{figure} [H]
            \centering
            \includegraphics[scale=.5,max width=\textwidth,max height=.9\textheight]{architecture_candidate_anime_uml-connectionManager-SM.eps}
            \caption{Machine à état représentant ConnectionManager}
            \label{MaE_ConnectionManager}     
        \end{figure}

\newpage

    \paragraph{[IHM] GUI}%2.3.3.2
    \begin{figure} [H]
        \centering
        \includegraphics[scale=.5,max width=\textwidth,max height=.9\textheight]{architecture_candidate_anime_uml-class-GUI.eps}
        \caption{Diagramme de classe représentant GUI}
        \label{Classe_GUI}
    \end{figure}
        \subparagraph{Philosophie de conception}%2.3.3.2.1
        La classe GUI permet de gérer les interfaces utilisateurs entre le Démonstrateur et AOP en affichant les différentes vues de AOP.
        Cette classe interagit avec ConnectionManager, DoorManager, EmployeeManager et Cameraman.
        \subparagraph{Description structurelle}%2.3.3.2.2
            \subsubparagraph{Attributs}%2.3.3.2.2.1
            \begin{itemize}
                \item myPass : Password : Contient le mot de passe entré lors de la connexion.
                \item myIP : IP : Contient l'adresse IP entrée lors de la connexion.
                \item myName : String : Contient le nom entré lors de l'ajout d'un employé.
                \item myFirstName : String : Contient le prénom entré lors de l'ajout d'un employé.
                \item myPicture : Picture : Contient la photo d'un employé à ajouter.
                \item myRole : Role : Contient le rôle d'un employé lors de son ajout.
                \item myWorkingHours : Hour[] : Contient les horaires de l'employé pour les 7 jours de la semaine.
                \item employeeList : Employee[] : Contient la liste des employés envoyés depuis la Board.
            \end{itemize}

            \subsubparagraph{Services offerts}%2.3.3.2.2.2
            \begin{itemize}
                \item displayScreen(screenID : ScreenID) : void : Affiche l'écran entré en paramètre de la fonction.
                \item displayPopUp(popUpID : PopUpID) : void : Affiche la PopUp entrée en paramètre de la fonction.
                \item displaySpecialEmployeeField(displaySpecial : bool) : void : Affiche les options afin d'ajouter les horaires d'un employé spécial.
                \item refreshVideoScreen() : Tâche de fond en charge de lire le flux vidéo entrant et de l'afficher sur l'écran permettant de regarder la vidéo.
                \item refreshCalendar(employeeID : EmployeeID) : void : Met à jour l'affichage du calendrier en fonction de l'ID de l'employé à afficher.
                \item refreshDoorState(doorState : bool) : Met à jour l'affichage de l'état de la Porte sur l'écran de contrôle à distance de la Porte en fonction du paramètre de type booléen.
                \item launchAOP() : void : Lance AOP.
                \item updateDoorState(doorState : bool) : void : Évènement pour demander à AOP de mettre à jour l'état de la Porte sur l'écran de contrôle à distance de la Porte.
                \item askScreen(screenID : ScreenID) : void : Démonstrateur demande à afficher l'écran entré en paramètre de la fonction.
                \item setIP(ip : IP) : void : Modifie l’attribut myIP avec le paramètre de la fonction.
                \item setPass(pass : Password) : void : Modifie l’attribut myPass avec le paramètre de la fonction.
                \item connect() : void : Demande la connexion.
                \item setConnectionStatus(status : ConnectionStatus) : void : Vérifie si la connexion est établie via le paramètre de la fonction status de type ConnectionStatus.
                \item askCalendar(employeeID : EmployeeID) : void : Demande d'afficher le calendrier correspondant au paramètre de la fonction de type EmployeeID.
                \item setEmployeeList(employeeListRemote : Employee[]) : void : Retour asynchrone permettant de mettre à jour la copie de la liste des employés de AOP.
                \item askOpenDoor() : void : Demande l'ouverture de la Porte.
                \item askAddEmployee() : void: Demande à ajouter un employé.
                \item setEmployeeName(name : String) : void : Modifie l’attribut myName avec le paramètre de la fonction.
                \item setEmployeeFirstName(firstName : String) : void :  Modifie l’attribut myFirstName avec le paramètre de la fonction.
                \item setEmployeePicture(picture : Picture) : void : Modifie l’attribut myPicture avec le paramètre de la fonction.
                \item setEmployeeRole(role : Role) : void : Modifie l’attribut myRole avec le paramètre de la fonction.
                \item setSpecialEmployeeAccess(day : Day, startHour : Hour, stopHour : Hour) : void : Entre les horaires de l'employé spécial.
                Avec comme paramètres day, startHour et stopHour qui prend en compte le jour, le début et de la fin de la journée de l'employé.
                Ces paramètres permettent la mise à jour de myWorkingHours.
                \item askDeleteEmployee(employeeID : EmployeeID) : void : Demande la suppression d'un employé. L'employé est déterminé par le paramètre de la fonction.
                \item confirm() : void : Confirme les modifications faites.
                \item cancel() : void : Annule les modifications effectuées.
                \item return() : void : Retourne à l'écran précédent.
                \item quitAOP() : void : Quitte AOP.
            \end{itemize}    
            
        \subparagraph{Description comportementale}%2.3.3.2.3
        Les MAE suivantes permettent d'expliquer le fonctionnement de l'objet actif GUI :
        \begin{figure} [H]
            \centering
            \includegraphics[scale=.5,max width=\textwidth,max height=.9\textheight]{architecture_candidate_anime_uml-gui-SM.eps}
            \caption{Machine à état représentant GUI}
            \label{MaE_connexion_GUI}
        \end{figure}

        \begin{figure} [H]
            \centering
            \includegraphics[scale=.5,max width=\textwidth,max height=.9\textheight]{architecture_candidate_anime_uml-gui-gui.Active.S_CONNECTED-SM.eps}
            \caption{Machine à état représentant GUI}
            \label{MaE_home_GUI}
        \end{figure}

        \begin{figure} [H]
            \centering
            \includegraphics[scale=.5,max width=\textwidth,max height=.9\textheight]{architecture_candidate_anime_uml-gui-gui.Active.S_CONNECTED.S_EMPLOYEE_LIST-SM.eps}
            \caption{Machine à état représentant employé liste}
            \label{MaE_home_GUI_employee_list}
        \end{figure}

\newpage

    \paragraph{[Object] EmployeeManager}%2.3.3.3
    \begin{figure} [H]
        \centering
        \includegraphics[scale=.5,max width=\textwidth,max height=.9\textheight]{architecture_candidate_anime_uml-class-EmployeeManager.eps}
        \caption{Diagramme de classe représentant EmployeeManager}
        \label{Classe_EmployeeManager}
    \end{figure}
        \subparagraph{Philosophie de conception}%2.3.3.3.1
        La classe EmployeeManager répertorie la liste des employés enregistrés dans l'application.
        Cet objet gère l'ajout et la suppression d'employés et interagit avec GUI, Bouncer. 
        \subparagraph{Description structurelle}%2.3.3.3.2
            \subsubparagraph{Attributs}%2.3.3.3.2.1
            \begin{itemize}
                \item {employeeList : Employee[] : Contient la liste des employés. 
                Il s'agit d'une copie des données persistantes chargée en mémoire.}
            \end{itemize}
            \subsubparagraph{Services offerts}%2.3.3.3.2.2
            \begin{itemize}
                \item {addEmployee(name : String, firstName : String, picture : Picture, role : String, workingHours : Hour[]) : void : Ajoute un employé à la liste à partir de son nom, prénom, sa photo, son rôle et les horaires de travail qui lui sont liés.}
                \item {deleteEmployee(employeeID : EmployeeID) : void : Supprime l'employé lié à l'identifiant employeeID passé en paramètre.}
                \item {getEmployeeList() : Employee[] : Renvoie la liste des employés à SoftSonnette.}
                \item {getEmployee(employeeID : EmployeeID) : Employee : Retourne l'objet employé lié à l'identifiant employeeID passé en paramètre.}
                \item {askEmployeeList() : Employee[] : Renvoie la liste des employés à AOP.}
                \item {load() : void : Charge les données persistantes des informations employés en mémoire vive.}
                \item {save() : void : Sauvegarde les données persistantes en mémoire non volatile.}
            \end{itemize}

\newpage

    \paragraph{[Object] Clock}%2.3.3.4
    \begin{figure} [H]
        \centering
        \includegraphics[scale=.5,max width=\textwidth,max height=.9\textheight]{architecture_candidate_anime_uml-class-Clock.eps}
        \caption{Diagramme de classe représentant Clock}
        \label{Classe_Clock}
    \end{figure}
        \subparagraph{Philosophie de conception}%2.3.3.4.1
        La classe Clock est en charge du temps.
        Elle synchronise les heures entre AOP et SoftSonnette.
        Elle est aussi utilisée pour savoir si un employé peut entrer en fonction de ses horaires attribués. 
        \subparagraph{Description structurelle}%2.3.3.4.2
            \subsubparagraph{Attributs}%2.3.3.4.2.1
            \begin{itemize}
                \item {currentTime : Time : Contient le temps courant.}
            \end{itemize}
            \subsubparagraph{Services offerts}%2.3.3.4.2.2
            \begin{itemize}
                \item {getCurrentTime(): Time : Permet de donner l'heure courante pour la comparaison des heures dans Bouncer.}
                \item {setCurrentTime(time : Time): void : Permet de modifier l'heure courante de la Board.}
            \end{itemize}
\newpage

    \paragraph{[Object] Cameraman}%2.3.3.5
    \begin{figure} [H]
        \centering
        \includegraphics[scale=.5,max width=\textwidth,max height=.9\textheight]{architecture_candidate_anime_uml-class-Cameraman.eps}
        \caption{Diagramme de classe de Cameraman}
        \label{Classe-Cameraman}
    \end{figure}
        \subparagraph{Philosophie de conception}%2.3.3.5.1
        La classe Cameraman interface E\_Caméra.
        Il permet de prendre une image du Testeur (pour qu’il se fasse reconnaître) et de capturer le flux vidéo retransmis sur l’écran de SoftSonnette et sur l’AOP.
        \subparagraph{Description structurelle}%2.3.3.5.2
            \subsubparagraph{Attributs}%2.3.3.5.2.1
            \begin{itemize}
                \item {cameraAlive : bool : Indique si la caméra est connectée ou non.}
                \item {alreadyAlive : bool : Indique si un flux vidéo est déja en cours d'utilisation}
            \end{itemize}
            \subsubparagraph{Services offerts}%2.3.3.5.2.2
            \begin{itemize}
                \item {streamToAOPandScreen() : void : Tâche de fond permettant de streamer la vidéo sur AOP et sur l'écran de SoftSonnette.}
                \item {streamToScreenOnly() : void : Tâche de fond permettant de streamer la vidéo sur l'écran de SoftSonnette.}
                \item {checkCameraConnected() : void : Vérifie la présence de la caméra.}
                \item {takePicture() : Picture : Fonction passive qui prend une photo de ce que voit Cameraman. 
                Cette photo est utilisée pour reconnaître le visage d’un Testeur.}
                \item {subscribeToVideoStream(enable : bool) : void : Fonction appelée par AOP pour demander à Cameraman de lui Streamer la vidéo.}
                \item {startStreaming() : void : Permet de démarrer le streaming.}
                \item {stopStreaming() : void : Stop définitivement le streaming - Sortie de la MàE.}
                \item {suspendStreaming() : void : Permet de mettre en pause le streaming.}
                \item {resumeStreaming() : void : Permet reprendre le streaming.}
            \end{itemize}
\newpage
        \subparagraph{Description comportementale}%2.3.3.3.3
            La MAE suivante permet d'expliquer le fonctionnement de l'objet actif Cameraman:
            \begin{figure} [H]
                \centering
                \includegraphics[scale=.5,max width=\textwidth,max height=.9\textheight]{architecture_candidate_anime_uml-cameraman-SM.eps}
                \caption{Machine à États de Cameraman}
                \label{MAE-Cameraman}
            \end{figure}

\newpage

    \paragraph{[Object] Guard}%2.3.3.6
        \begin{figure} [H]
            \centering
            \includegraphics[scale=.5,max width=\textwidth,max height=.9\textheight]{architecture_candidate_anime_uml-class-Guard.eps}
            \caption{Diagramme de classe de Guard}
            \label{Classe-Guard}
        \end{figure}
        \subparagraph{Philosophie de conception}%2.3.3.6.1
        La classe Guard est le gardien du mot de passe permettant à \appliA~d'interagir avec \appliLin.
        Il vérifie que le mot de passe reçu est celui codé en dur dans l'application puis autorise la connexion si ce dernier est correct.
        \subparagraph{Description structurelle}%2.3.3.6.2
            \subsubparagraph{Attributs}%2.3.3.6.2.1
            N.A.
            \subsubparagraph{Services offerts}%2.3.3.6.2.2
            \begin{itemize}
                \item{askCheckPass(pass : Password) : void : Permet de demander à Guard de vérifier le mot de passe passé en paramètre.}
            \end{itemize}

\newpage

    \paragraph{[Object] Bouncer}%2.3.3.7
    \begin{figure} [H]
        \centering
        \includegraphics[scale=.5,max width=\textwidth,max height=.9\textheight]{architecture_candidate_anime_uml-class-Bouncer.eps}
        \caption{Diagramme de classe représentant Bouncer}
        \label{Classe_Bouncer}
    \end{figure}
        \subparagraph{Philosophie de conception}%2.3.3.7.1
        La classe Bouncer permet au SàE d'autoriser ou non un Testeur à entrer.
        En interaction avec Cameraman elle déclenche la prise d'une photo du Testeur se présentant à la sonnette.
        Elle interagit également avec EmployeeManager pour récupérer la liste des employés.
        Elle déclenche la reconnaissance faciale avec la classe RecognitionAI à qui elle envoie toutes les données nécessaires.
        Elle vérifie qu'un employé reconnu ait l'autorisation d'entrer sur l'horaire courant qu'elle récupère auprès de la classe Clock.
        Elle demande enfin l'ouverture de la Porte à DoorManager ou signale à UISP que le Testeur n'est pas reconnu ou non autorisé.
        \subparagraph{Description structurelle}%2.3.3.7.2
            \subsubparagraph{Attributs}%2.3.3.7.2.1
            \begin{itemize}
                \item{picture : Picture : Photo du Testeur à reconnaître.}
                \item{employeeList : Employee[] : Liste courante des employés.}
                \item{recognizedEmployee : EmployeeID : ID de l'employé reconnu par RecognitionAI.}
                \item{currentTime : Time : Temps utilisé pour vérifier que l'employé est autorisé à entrer.}
            \end{itemize}
            \subsubparagraph{Services offerts}%2.3.3.7.2.2
            \begin{itemize}
                \item {checkEmployeeAllow(time : Time, employeeID : EmployeeID) : bool : À partir du temps courant et de l'ID de l'employé, elle vérifie si l'employé est autorisé à entrer à l'horaire courant.}
                \item {setRecognizeFace(employeeID : EmployeeID) : void : Appelée par RecognitionAI pour retourne l'employé reconnu par l'IA, ou UNKNOWN s'il n'y a pas de correspondance.}
                \item {askFaceRecognition() : void : Trigger permettant de déclencher le processus reconnaissance facial et d'autorisation d'entrée.}
            \end{itemize}
\newpage
        \subparagraph{Description comportementale}%2.3.3.7.3
        La MAE suivante permet d'expliquer le fonctionnement de l'objet actif Bouncer:
        \begin{figure} [H]
            \centering
            \includegraphics[scale=.5,max width=\textwidth,max height=.9\textheight]{architecture_candidate_anime_uml-bouncer-SM.eps}
            \caption{Machine à États de Bouncer}
            \label{MAE-Bouncer}
        \end{figure}
    
\newpage

    \paragraph{[IHM] UISS}%2.3.3.8
    \begin{figure} [H]
        \centering
        \includegraphics[scale=.5,max width=\textwidth,max height=.9\textheight]{architecture_candidate_anime_uml-class-UISS.eps}
        \caption{Diagramme de classe d'UISS}
        \label{Classe-UISS}
    \end{figure}
        \subparagraph{Philosophie de conception}%2.3.3.8.1
        La classe UISS s'occupe de la gestion de l'interface utilisateur entre le \actT~, le \actD~et \appliLin.
        Elle permet l'affichage de la vidéo, l'état de la Porte mais aussi au \actT~de sonner et au \actD~de quitter l'application.
        \subparagraph{Description structurelle}%2.3.3.8.2
            \subsubparagraph{Attributs}%2.3.3.8.2.1
            \begin{itemize}
                 \item {appState : SSState : Variable de mise a jour de l'écran qui prend trois valeurs possibles.}
            \end{itemize}
            \subsubparagraph{Services offerts}%2.3.3.8.2.2
            \begin{itemize}
                \item {display(state : SSState) : void : Affiche l'écran de \appliLin~ avec la variante transmise par le paramètre state.}
                \item {refreshDoorState(doorState : bool) : Met à jour l'affichage de l'état de la Porte sur l'écran de SoftSonnette en fonction du paramètre de type booléen passé.}
                \item {launchSS() : void : Démarre l'application \appliLin.}
                \item {quitSS() : void : Ferme l'application \appliLin.}
                \item {askStartCom() : void : Évènement reçu de la part de SoftPorte qui demande l'établissement de la communication.}
                \item {ring() : void : \actT~sonne.}
                \item {updateCamState(bool) : void : Évènement en provenance de Cameraman indiquant à UISS l'état de la caméra en fonction du paramètre de type booléen passé.
                UISS met ensuite à jour l'écran avec un appel à la fonction display().}
                \item {updateDoorState(doorState : bool) : void : Évènement en provenance de DoorManager indiquant à UISS l'état de la Porte en fonction du paramètre de type booléen passé.
                UISS met ensuite à jour le label indiquant l'état de la Porte sur l'écran avec un appel à la fonction refreshDoorState().}
            \end{itemize}
\newpage
        \subparagraph{Description comportementale}%2.3.3.8.3
        La MAE suivante permet d'expliquer le fonctionnement de l'objet actif UISS.
        \begin{figure} [H]
            \centering
            \includegraphics[scale=.5,max width=\textwidth,max height=.9\textheight]{architecture_candidate_anime_uml-uiss-SM.eps}
            \caption{Machine à États d'UISS}
            \label{MAE-UISS}
        \end{figure}

\newpage

    \paragraph{[Object] DoorManager}%2.3.3.9
    \begin{figure} [H]
        \centering
        \includegraphics[scale=.5,max width=\textwidth,max height=.9\textheight]{architecture_candidate_anime_uml-class-DoorManager.eps}
        \caption{Diagramme de classe représentant DoorManager}
        \label{DoorManager}
    \end{figure}
        \subparagraph{Philosophie de conception}%2.3.3.9.1
        La classe DoorManager est en charge du contrôle de la Porte simulée.
        Elle permet de commander son déverrouillage, et de fournir l'état de cette dernière.
    
        \subparagraph{Description structurelle}%2.3.3.9.2
            \subsubparagraph{Attributs}%2.3.3.9.2.1
            \begin{itemize}
                \item {doorState : bool : Indique l'état de la Porte (ouvert/fermé).}
            \end{itemize}
            \subsubparagraph{Services offerts}%2.3.3.9.2.2
            \begin{itemize}
                \item {unlockDoor() : void : Déverrouille la Porte pendant \hyperlink{top}{\textit{TOP}}.}
                \item {askOpenDoor() : void : Envoie la demande d'ouverture de la Porte vers UISP.}
                \item {askDoorState() : void : Demande l'état de la porte à UISP (retour asynchrone).}
            \end{itemize}
        \subparagraph{Description comportementale}%2.3.3.9.3
        \begin{figure} [H]
            \centering
            \includegraphics[scale=.5,max width=\textwidth,max height=.9\textheight]{architecture_candidate_anime_uml-doorManager-SM.eps}
            \caption{Machine à état représentant DoorManager}
            \label{MaE_DoorManager}
        \end{figure}

\newpage

    \paragraph{[IHM] UISP}%2.3.3.10
    \begin{figure} [H]
        \centering
        \includegraphics[scale=.5,max width=\textwidth,max height=.9\textheight]{architecture_candidate_anime_uml-class-UISP.eps}
        \caption{Diagramme de classe d'UISP}
        \label{UISP}
    \end{figure}
        \subparagraph{Philosophie de conception}%2.3.3.10.1
        La classe UISP permet de mettre à jour les différents éléments physiques de la Board.
        Elle met en place la communication entre SoftPorte et SoftSonnette, signale son bon démarrage en agissant sur la LED LD5 verte et signale lorsqu'un Testeur n'est pas autorisé à entrer avec la LED LD6 rouge.
        \subparagraph{Description structurelle}%2.3.3.10.2
            \subsubparagraph{Attributs}%2.3.3.10.2.1
            N.A.
            \subsubparagraph{Services offerts}%2.3.3.10.2.2
            \begin{itemize}
                \item {signalAppState() : void : Signale le lancement de l'application en allumant la LED verte.}
                \item {launchSP() : void : Lance SoftPorte.}
                \item {quitSP() : void : Quitte SoftPorte.}
                \item {signalNotAllowed(): void : Signale l'interdiction d'entrer en allumant la LED rouge.}
                \item {ackStartCom() : void : Confirme l'initialisation de la communication entre SoftPorte et SoftSonnette. 
                Cette fonction appel ensuite signalAppState() si la communication est établie avec succès.}
            \end{itemize}

\newpage

    \paragraph{[Object] RecognitionAI}%2.3.3.11
        \begin{figure} [H]
            \centering
            \includegraphics[scale=.5,max width=\textwidth,max height=.9\textheight]{architecture_candidate_anime_uml-class-RecognitionAI.eps}
            \caption{Diagramme de classe de RecognitionAI}
            \label{Classe-RecognitionAI}
        \end{figure}
            \subparagraph{Philosophie de conception}%2.3.3.11.1
            La classe RecognitionAI interface la librairie client de reconnaissance faciale.
            En lui fournissant la photo d’un Testeur, celle-ci détermine si le Testeur en question est autorisé à entrer.
            Cet objet est appelé par Bouncer lorsqu’une demande d’ouverture est faite.
            Une fois la reconnaissance faciale terminée, RecognitionAI envoie à Bouncer le résultat.
            \subparagraph{Description structurelle}%2.3.3.11.2
                \subsubparagraph{Attributs}%2.3.3.11.2.1
                N.A.
                \subsubparagraph{Services offerts}%2.3.3.11.2.2
                \begin{itemize}
                    \item{launch(picture : Picture, employeeList : Employee[]) : void : Lance le processus de reconnaissance faciale.}
                \end{itemize}



%-----------------------------------------
%      Partie III
%-----------------------------------------

\section{Conception détaillée}

La conception détaillée définit l'architecture logique du système, c'est à dire:
\begin{itemize}
    \item Répartir le système sur l'architecture matérielle.
    \item Gérer les entrées et sorties, ainsi que les IHM.
    \item Élaborer les protocoles de communication.
    \item Définir les parallélismes et l'initialisation.
    \item Définir le démarrage, l'arrêt et la destruction du SàE.
\end{itemize}

\subsection{Architecture détaillée}

Les diagrammes ci-dessous représentent le SàE dans son intégralité. 
En plus des classes vues précédemment, ce diagramme présente la gestion de la communication ainsi que les objets frontières nécessaires au bon fonctionnement du logiciel.

Ces diagrammes, dans un souci de lisibilité, ne détaillent pas les méthodes ainsi que les attributs des classes.
Ces derniers sont détaillés dans les diagrammes de classe qui suivent.

\begin{figure} [H]
    \centering
    \includegraphics[scale=.5,max width=\textwidth,max height=.9\textheight]{architecture_candidate_AOP_plantuml.eps}
    \caption{Architecture candidate détaillée de AOP représentée par un diagramme de communication}
    \label{archiDetailleAOP}
\end{figure}

Il est a noté que le classe GUI de la conception générale a été divisée en plusieurs Fragment représentant les différents écrans.
Ces fragments sont regroupés dans le package gui.

AOP adopte une stratégie de communication en utilisant une classe Communication, Protocol, Postman et Dispatcher.
A l'inverse, SoftSonnette et SoftPorte adoptent une stratégie avec des Proxys, Postamn, Protcol et Dispatcher.
Des détails sur ces deux stratégies sont proposés dans la partie 3.3. "Protocole de communication".

\begin{figure} [H]
    \centering
    \includegraphics[scale=.5,max width=\textwidth,max height=.9\textheight]{architecture_candidate_softSonnette.eps}
    \caption{Architecture candidate détaillée de SoftSonnette représentée par un diagramme de communication}
    \label{archiDetailleSS}
\end{figure}

% Important : Par soucis de lisibilité, ne sont pas représentés sur ce diagramme les classes protocoles qui interfacent les objets Proxy et Dispatcher comme suit :

% \begin{figure} [H]
%     \centering
%     \includegraphics[scale=.5,max width=\textwidth,max height=.9\textheight]{architecture_candidate_softSonnette_Protocol}
%     \caption{Diagramme de classes présentant les interactions avec les objets ProtocolSP et ProtocolSS}
%     \label{archiSimplifieeProtocol}
% \end{figure}

\begin{figure} [H]
    \centering
    \includegraphics[scale=.5,max width=\textwidth,max height=.9\textheight]{architecture_candidate_SP}
    \caption{Architecture candidate détaillée de SoftPorte représentée par un diagramme de communication}
    \label{archiDetailleSoftPorte}
\end{figure}

Les objets Starter ne sont pas représentés dans les diagrammes de communication et sont détaillés dans la suite du document.


\newpage

\subsection{Description des classes}

\subsubsection{Description des classes de \appliLin}%3.2.1

    \paragraph{[Object] Starter}%3.2.1.2

        \begin{figure} [H]
            \centering
            \includegraphics[scale=.5,max width=\textwidth,max height=.9\textheight]{architecture_candidate_softSonnette-class-Starter.eps}
            \caption{Diagramme de classe de Starter}
            \label{Classe-Starter}
        \end{figure}
    
        \subparagraph{Philosophie de Conception}%3.2.1.2.1

        La classe Starter a pour rôle d'instancier tous les objets actifs utilisés par SoftSonnette et de démarrer leur machine à état respective.
        Les objets instanciés sont les suivants : 

        \begin{itemize}
            \item {Cameraman}
            \item {UISS}
            \item {PostmanAOP}
            \item {DispatcherAOP}
            \item {Bouncer}
        \end{itemize}

        La classe Starter permet aussi de stopper les machines à état et détruire tous les objets instanciés lors de l'arrêt de SoftSonnette.
        
        \subparagraph{Description structurelle}%3.2.1.2.2

            \subsubparagraph{Attributs}%3.2.1.2.2.1
            N.A.
            \subsubparagraph{Services offerts}%3.2.1.2.2.2
            \begin{itemize}
                \item {main() : int : La méthode initiale appelée au lancement de l'application SoftSonnette. Elle instancie et démarre les objets indiqués ci-dessus.}
                \item {cleanup() : void : Détruit les objets après l'arrêt de l'application et libère la mémoire.}
            \end{itemize}
\newpage
        \subparagraph{Diagramme de séquence du démarrage et de l'arrêt de \appliLin}%3.2.1.2.3
        Le diagramme suivant représente la séquence d'initialisation de \appliLin.

        \begin{figure} [H]
            \centering
            \includegraphics[scale=.5,max width=\textwidth,max height=.9\textheight]{diagramme_séquence_démarrage_SS.eps}
            \caption{Diagramme de séquence de l'initialisation de SoftSonnette}
            \label{Sequ-Demarrage-SS}
        \end{figure}
\newpage
        Le diagramme suivant représente la séquence d'arrêt de \appliLin.

        \begin{figure} [H]
            \centering
            \includegraphics[scale=.5,max width=\textwidth,max height=.9\textheight]{diagramme_séquence_arret_SS.eps}
            \caption{Diagramme de séquence de l'arrêt de SoftSonnette}
            \label{Sequ-Arret-SS}
        \end{figure}

\newpage

    \paragraph{[Medium] ProxyGUI}

        \begin{figure} [H]
            \centering
            \includegraphics[scale=.5,max width=\textwidth,max height=.9\textheight]{architecture_candidate_softSonnette-class-ProxyGUI.eps}
            \caption{Diagramme de classe de ProxyGUI}
            \label{Classe-ProxyGUI}
        \end{figure}
        
        \subparagraph{Philosophie de Conception}%3.2.1.2.1

        La classe ProxyGUI dans SoftSonnette simule le comportement de la classe GUI et gère l'encodage de la trame pour la communication vers AOP.
        Cette classe sert également d'intermédiaire pour la communication entre SoftPorte et AOP en ré-encodant les messages en provenance de SoftPorte et à destination de AOP. 
        Elle gère la communication d'une seule méthode entre DoorManager et GUI, à travers DispatcherSP et PostmanAOP.
        
        \subparagraph{Description structurelle}%3.2.1.2.2

            \subsubparagraph{Attributs}%3.2.1.2.2.1
            N.A.
            \subsubparagraph{Services offerts}%3.2.1.2.2.2
            \begin{itemize}
                \item {updateDoorState(doorState : bool) : void : Met à jour l'affichage de l'état de la Porte sur AOP selon l'état state.}
            \end{itemize}

\newpage

    \paragraph{[Medium] ProxyConnectionManager}

        \begin{figure} [H]
            \centering
            \includegraphics[scale=.5,max width=\textwidth,max height=.9\textheight]{architecture_candidate_softSonnette-class-ProxyConnectionManager.eps}
            \caption{Diagramme de classe de ProxyConnectionManager}
            \label{Classe-ProxyConnectionManager}
        \end{figure}

        \subparagraph{Philosophie de Conception}%3.2.1.2.1

        La classe ProxyConnectionManager simule le comportement de la classe ConnectionManager, et permet la communication entre Guard et ConnectionManager en passant par l'objet PostmanAOP.
        
        \subparagraph{Description structurelle}%3.2.1.2.2

            \subsubparagraph{Attributs}%3.2.1.2.2.1
                N.A. 
            \subsubparagraph{Services offerts}%3.2.1.2.2.2
            \begin{itemize}
                \item {validatePass(passValidated : bool) : void : Méthode permettant d'envoyer le résultat de la vérification de la connexion à AOP en passant par PostmanAOP, en l'encodant.}
            \end{itemize}
            \newpage

    \paragraph{[Medium][Boundary] Streamer}

        \begin{figure} [H]
            \centering
            \includegraphics[scale=.5,max width=\textwidth,max height=.9\textheight]{architecture_candidate_softSonnette-class-Streamer.eps}
            \caption{Diagramme de classe de Streamer}
            \label{Classe-Streamer}
        \end{figure}

        \subparagraph{Philosophie de Conception}%3.2.1.2.1

        La classe Streamer est un objet frontière actif.
        Il permet l'envoie du flux vidéo depuis SoftSonnette vers AOP.
        Streamer est basé sur la librairie Linux open-source Gstreamer.
        
        \subparagraph{Description structurelle}%3.2.1.2.2

            \subsubparagraph{Attributs}%3.2.1.2.2.1
            \begin{itemize}
                \item {inforStream : CustomData : Variable stockant les données du flux vidéo flux}
            \end{itemize}

            CustomData est une structure permettant de créer le flux video.
            CustomData contient les informations sur le bus, le pipeline, l'état, l'état de la boucle de rafraîchissement et la valeur du flux video.

            \subsubparagraph{Services offerts}%3.2.1.2.2.2
            \begin{itemize}
                \item {initGstreamer() : void : Méthode permettant d'initialiser les instances pour utiliser Gstreamer.}
                \item {streamEcran() : void : Méthode permettant d'afficher le flux video sur l'écran de SoftSonnette.}
                \item {streamTel() : void : Méthode permettant d'afficher le flux video sur l'écran de SoftSonnette et d'AOP.}
                \item {cleanPipeline() : void : Méthode permettant de nettoyer le pipeline du flux video.}
                \item {stopPipeline() : void : Méthode permettant d'arrêter le flux vidéo.}
            \end{itemize}

\newpage 

    \paragraph{[Medium] ProxyUISP}

        \begin{figure} [H]
            \centering
            \includegraphics[scale=.5,max width=\textwidth,max height=.9\textheight]{architecture_candidate_softSonnette-class-ProxyUISP.eps}
            \caption{Diagramme de classe de ProxyUISP}
            \label{Classe-UISP}
        \end{figure}
    
            \subparagraph{Philosophie de Conception}%3.2.1.8.1
                
            La classe ProxyUISP simule le comportement de la classe UISP, et permet la communication entre Bouncer et UISP en passant par l'objet PostmanSP.                      
            \subparagraph{Description structurelle}%3.2.1.8.2
                
            \subsubparagraph{Attributs}%3.2.1.8.2.1
            N.A.
            \subsubparagraph{Services offerts}%3.2.1.8.2.2
            \begin{itemize}
              \item {launchSP() : void : Méthode utilisée pour démarrer SoftPorte.}    
            \end{itemize}  

\newpage

    \paragraph{[Medium] ProxyDoorManager}

        \begin{figure} [H]
            \centering
            \includegraphics[scale=.5,max width=\textwidth,max height=.9\textheight]{architecture_candidate_softSonnette-class-ProxyDoorManager.eps}
            \caption{Diagramme de classe de ProxyDoorManager}
            \label{Classe-ProxyDoorManager}
        \end{figure}
    
            \subparagraph{Philosophie de Conception}%3.2.1.9.1
                
            La classe ProxyDoorManager simule le comportement de la classe DoorManager. Elle permet ainsi la communication entre Bouncer et DoorManager en passant par l'objet PostmanSP.                      
            \subparagraph{Description structurelle}%3.2.1.9.2
                
            \subsubparagraph{Attributs}%3.2.1.9.2.1
            N.A.
            \subsubparagraph{Services offerts}%3.2.1.9.2.2
            \begin{itemize}
              \item {askOpenDoor() : void : Méthode utilisée pour demander l'ouverture la Porte.}  
              \item {getDoorState() : bool : Méthode utilisée pour savoir si la Porte est actuellement ouverte ou fermée.}    
            \end{itemize}

\newpage

    \paragraph{[Medium] PostmanSP}

        \begin{figure} [H]
            \centering
            \includegraphics[scale=.5,max width=\textwidth,max height=.9\textheight]{architecture_candidate_softSonnette-class-PostmanSP.eps}
            \caption{Diagramme de classe de PostmanSP}
            \label{Classe-PostmanSP}
        \end{figure}
    
            \subparagraph{Philosophie de Conception}%3.2.1.10.1
                
            La classe PostmanSP est utilisée pour permettre à SoftSonnette d'envoyer des messages à SoftPorte.                

            \subparagraph{Description structurelle}%3.2.1.10.2
                
            \subsubparagraph{Attributs}%3.2.1.10.2.1

            \subsubparagraph{Services offerts}%3.2.1.10.2.2

            \begin{itemize}
                \item {readMessage(size : int) : void : Réception d'un message provenant de SoftPorte.}
                \item {askSendMessage(message : char*) : void : Envoi d'un message à SoftPorte.} 
            \end{itemize}  

\newpage

    \paragraph{[Object] ProtocolSS}

        \begin{figure} [H]
            \centering
            \includegraphics[scale=.5,max width=\textwidth,max height=.9\textheight]{architecture_candidate_softSonnette-class-ProtocolSS.eps}
            \caption{Diagramme de classe de ProtocolSS}
            \label{Classe-ProtocolSS}
        \end{figure}

            \subparagraph{Philosophie de conception}

            La classe ProtocolSS gère les fonctions d'encodage et de décodage des trames lors de la communication entre AOP et SoftSonnette, côté SoftSonnette.

            \subparagraph{Description structurelle}

            \subsubparagraph{Attributs}
            N.A. 
            \subsubparagraph{Services offerts}

            \begin{itemize}
                \item {encode(cmd\_id : int, nbargs : int, data : char*) : char* : Fonction haut niveau d'encodage de la trame. 
                Construit la trame en utilisant les fonctions intermédiaires d'encodage.}
                \item {decode(frame : char[]) : Decoded\_Frame : Fonction de décodage de la trame. 
                Reconstruit la trame via une instance de la structure Decoded\_Frame.} 
            \end{itemize} 

\newpage

    \paragraph{[Object] ProtocolSP}

        \begin{figure} [H]
            \centering
            \includegraphics[scale=.5,max width=\textwidth,max height=.9\textheight]{architecture_candidate_softSonnette-class-ProtocolSP.eps}
            \caption{Diagramme de classe de ProtocolSP}
            \label{Classe-ProtocolSP}
        \end{figure}
    
                \subparagraph{Philosophie de conception}
    
                L'objet ProtocolSP gère les fonctions d'encodage et de décodage des trames lors de la communication entre SoftPorte et SoftSonnette, côté SoftSonnette.
    
                \subparagraph{Description structurelle}
    
                \subsubparagraph{Attributs}
                N.A. 
                \subsubparagraph{Services offerts}
    
                \begin{itemize}
                    \item {encode(cmd\_id : int, target : int) : char* : Fonction haut niveau d'encodage de la trame. Construit la trame en utilisant les fonctions intermédiaires d'encodage.}
                    \item {decode(frame : char[]) : Decoded\_Frame\_SP : Fonction de décodage de la trame. Reconstruit la trame via une instance de la structure Decoded\_Frame\_SP.} 
                \end{itemize} 

\newpage

    \paragraph{[Medium] PostmanAOP}

        \begin{figure} [H]
            \centering
            \includegraphics[scale=.5,max width=\textwidth,max height=.9\textheight]{architecture_candidate_softSonnette-class-PostmanAOP.eps}
            \caption{Diagramme de classe de PostmanAOP}
            \label{Classe-PostmanAOP}
        \end{figure}
    
        \subparagraph{Philosophie de Conception}%3.2.1.10.1
                
            La classe PostmanAOP a pour objectif de gérer l'envoi et la réception de messages à travers les sockets de communication entre SoftSonnette et AOP.
            C'est aussi la classe qui met en place le serveur de cette communication.
            Elle est en interaction avec les proxys de SoftSonnette et DispatcherAOP pour l'encodage et le décodage des trames.

            \subparagraph{Description structurelle}%3.2.1.10.2
                
            \subsubparagraph{Attributs}%3.2.1.10.2.1

            \begin{itemize}
                \item {socketSoftSonnette : int : Socket serveur qui écoute les connexions d'AOP sur le port 1234.} 
                \item {socketAOP : int : Socket client utilisé par AOP pour communiquer avec SoftSonnette.}    
            \end{itemize}  
            
            \subsubparagraph{Services offerts}%3.2.1.10.2.2

            \begin{itemize}
                \item {readMessage(size : int) : void : Réception d'un message provenant d'AOP.}
                \item {askSendMessage(message : char[]) : void : Envoi d'un message à AOP via la boîte aux lettres.} 
                \item {setUpConnection() : void : Initialise le socket serveur et attend la connexion du client.}
            \end{itemize}  

            \subparagraph{Gestion du multitâche}%3.2.1.10.2

            La machine à état de PostmanAOP ainsi que l'attente de la connexion du client s’exécutent dans des threads séparés lancés à partir de la méthode start() par le thread principal.
            Les évènements tels que l'envoi d'un message ou l'arrêt du service sont transmis à partir d'une boîte aux lettres. 
            \newpage

    \paragraph{[Medium] DispatcherAOP}

        \begin{figure} [H]
            \centering
            \includegraphics[scale=.5,max width=\textwidth,max height=.9\textheight]{architecture_candidate_softSonnette-class-DispatcherAOP.eps}
            \caption{Diagramme de classe de DispatcherAOP}
            \label{Classe-DispatcherAOP}
        \end{figure}
    
        \subparagraph{Philosophie de Conception}%3.2.1.10.1
                        
            L'objet DispatcherAOP a pour objectif de lire les messages reçus depuis le facteur afin de les décoder et de les renvoyer à l'objet SoftSonnette correspondant.

        \subparagraph{Description structurelle}%3.2.1.10.2
                
            \subsubparagraph{Attributs}%3.2.1.10.2.1
            N.A.
            
            \subsubparagraph{Services offerts}%3.2.1.10.2.2

            \begin{itemize}
                \item {dispatch() : void : Décode le message lu sur le facteur de SoftSonnette, puis envoie le message à l'objet concerné.}
                \item {setConnected(state : bool) : void : Indique au dispatcher le statut de connexion.}
            \end{itemize}  
            \newpage

    \paragraph{[Medium] DispatcherSP}

        \begin{figure} [H]
            \centering
            \includegraphics[scale=.5,max width=\textwidth,max height=.9\textheight]{architecture_candidate_softSonnette-class-DispatcherSP.eps}
            \caption{Diagramme de classe de DispatcherSP}
            \label{Classe-DispatcherSP}
        \end{figure}
    
            \subparagraph{Philosophie de Conception}%3.2.1.11.1
            DispatcherSP est l'objet permettant de recevoir et décoder les messages en provenance de l'objet PostmanSP.
            Ces messages sont ensuite envoyés à UISS ou à ProxyGUI.                
            \subparagraph{Description structurelle}%3.2.1.11.2
                
            \subsubparagraph{Attributs}%3.2.1.11.2.1
            N.A.
            \subsubparagraph{Services offerts}%3.2.1.11.2.2
            \begin{itemize}
                \item {dispatch() : void : Décode le message lu sur le facteur de SoftPorte, puis envoie le message à l'objet UISS ou à l'objet ProxyGUI.}
            \end{itemize} 
            \newpage

    \paragraph{[Entity] DataEmployee}

        \begin{figure} [H]
            \centering
            \includegraphics[scale=.5,max width=\textwidth,max height=.9\textheight]{architecture_candidate_softSonnette-class-DataEmployee.eps}
            \caption{Diagramme de classe de DataEmployee}
            \label{Classe-DataEmployee}
        \end{figure}
    
            \subparagraph{Philosophie de Conception}%3.2.1.13.1
                
            L'entité DataEmployee représente les données persistantes des employés.
            Cette entité est interfacée uniquement par EmployeeManager.              
            \subparagraph{Description structurelle}%3.2.1.13.2
                
            \subsubparagraph{Attributs}%3.2.1.13.2.1
            N.A.
            \subsubparagraph{Services offerts}%3.2.1.13.2.2
            N.A.
            \newpage

    \subsubsection{Description des classes de \appliA}%3.2.2

    \paragraph{[Object] StarterAOP}

        \begin{figure} [H]
            \centering
            \includegraphics[scale=.5,max width=\textwidth,max height=.9\textheight]{architecture_candidate_AOP_anim_uml-class-StarterAOP.eps}
            \caption{Diagramme de classe de StarterAOP}
            \label{Classe-StarterAOP}
        \end{figure}

            \subparagraph{Philosophie de Conception}
            La classe StartAOP permet d'instancier les autres objets de AOP.  
            La classe StartAOP est reliée à MainActivity.            
            \subparagraph{Description structurelle}
                
            \subsubparagraph{Attributs}
            N.A.
            \subsubparagraph{Services offerts}
            \begin{itemize}
                \item {launchAOP() : void : Démarrage des instanciations.}
                \item {quitAOP() : void : Détruit les instanciations.}
            \end{itemize}

        \subparagraph{Diagramme de séquence du démarrage et de l'arrêt de \appliA}

        Le diagramme suivant représente la séquence d'initialisation d'\appliA.
        \begin{figure} [H]
            \centering
            \includegraphics[scale=.5,max width=\textwidth,max height=.9\textheight]{diagramme_séquence_démarrage_AOP.eps}
            \caption{Diagramme de séquence de l'initialisation d'\appliA}
            \label{Seq-Init-AOP}
        \end{figure}
    
        Le diagramme suivant représente la séquence d'arrêt d'\appliA.
        \begin{figure} [H]
            \centering
            \includegraphics[scale=.5,max width=\textwidth,max height=.9\textheight]{diagramme_séquence_arret_AOP.eps}
            \caption{Diagramme de séquence de l'arrêt d'\appliA}
            \label{Seq-Arret-AOP}
        \end{figure}

\newpage

    \paragraph{[Package] GUI}

            \begin{figure} [H]
                \centering
                \includegraphics[scale=.5,max width=\textwidth,max height=.9\textheight]{package_gui.eps}
                \caption{Package de GUI}
                \label{Package gui}
            \end{figure}
            \subparagraph{Description structurelle de GUI}%3.2.1.13.2
            Le package GUI correspond à l'objet GUI de la conception générale.
            Il se compose de plusieurs fragments, chaque fragment étant un écran de l'application AOP.
            GUI possède également une activité qui gère toute l'application et est en charge des fragments : FragmentCalendar, FragmentVideo, FragmentListEmployee, FragmentAddEmployee et FragmentDoor.
            \newpage

    \paragraph{[Medium] Communication}

        \begin{figure} [H]
            \centering
            \includegraphics[scale=.5,max width=\textwidth,max height=.9\textheight]{architecture_candidate_AOP_anim_uml-class-Communication.eps}
            \caption{Diagramme de classe de Communication}
            \label{Classe-Communication}
        \end{figure}
    
            \subparagraph{Philosophie de Conception}%3.2.1.13.1
                
            La classe Communication permet de transiter les informations de connexion.
            Elle permet aussi d’assurer que les applications sont connectées et d’agir en conséquence. 
            Elle reçoit les demandes de l'utilisateur puis communique avec le Dispatcher et le PostmanSoftSonnette. 
            setConnectionStatus() est défini par les méthodes onConnectionEstablished(), onConnectionLost() et onConnectionFailed().
            \subparagraph{Description structurelle}%3.2.1.13.2
                
            \subsubparagraph{Attributs}%3.2.1.13.2.1
            N.A.
            \subsubparagraph{Services offerts}%3.2.1.13.2.2
            \begin{itemize}
                \item {onConnectionEstablished() : void : Évènement appelé lorsque la connexion entre AOP et SoftSonnette est faite.} 
                \item {onConnectionLost() : void : Évènement appelé lors de la perte de la connexion entre AOP et SoftSonnette.} 
                \item {onConnectionFailed() : void : Évènement appelé lors de l'échec de la connexion entre AOP et SoftSonnette.}
                \item {errorConnection() : void : Permet de signaler à l’utilisateur une erreur lors de la connexion.}
                \item {askCheckPass (pass : String) : void : Envoie le mot de passe à SoftSonnette.}
                \item {setCurrentTime(time : Time) : void : Envoie l'heure d'AOP à SoftSonnette.}
                \item {sendClock(clock : String) : void: Envoie la clock de AOP à SoftSonnette.}
                \item {askCalendar() : void : Demande à recevoir la liste des employés pour le calendrier.}
                \item {askEmployeeList() : void : Demande la liste des employés.}
                \item {askOpenDoor() : void : Demande l’ouverture de la Porte.}
                \item {askDoorState() : void : Demande l'état de la Porte.}
                \item {addEmployee(employeeID : byte, name : String, firstName : String, role : byte, workingHours : byte[][]) : void : Demande l'ajout d'un employé.}
                \item {deleteEmployee(employeeID : EmployeeID) : void : Demande la suppression d'un employé des données persistantes.}
                \item {subscribeToVideoStream (enable : bool) : void : Active ou désactive l'affichage de la vidéo.}
                \item {run() : void : Lance un thread lisant les messages envoyés par SoftSonnette.}
                \item {beginConnection() : void : Lance la connexion avec SoftSonnette.} 
                \item {endConnection() : void : Termine la connexion avec SoftSonnette.} 
            \end{itemize} 
            \newpage

        %     \paragraph{[Medium] ConnectionManager}

        % \begin{figure} [H]
        %     \centering
        %     \includegraphics[scale=.5,max width=\textwidth,max height=.9\textheight]{architecture_candidate_AOP_anim_uml-class-ConnectionManager.eps}
        %     \caption{Diagramme de classe de ConnectionManager}
        %     \label{Classe-ConnectionManager}
        % \end{figure}
    
        %     \subparagraph{Philosophie de Conception}%3.2.1.13.1
                
        %     L'objet ConnectionManager permet de gérer la connexion entre la partie IHM et la partie Communication d'AOP.           
        %     \subparagraph{Description structurelle}%3.2.1.13.2
                
        %     \subsubparagraph{Attributs}%3.2.1.13.2.1
        %     N.A.
        %     \subsubparagraph{Services offerts}%3.2.1.13.2.2
        %     \begin{itemize}
        %         \item {checkIp(myIp : String) : bool : Retourne si le format de l'adresse IP est correct.}
        %         \item {getLocalTime() : void : Permet d'obtenir l'heure d'AOP.}
        %         \item {validatePass(valid : byte) : void : Récupère le retour du serveur concernant la validité du mot de passe entré par l'utilisateur.}
        %         \item {checkConnection() : void : Détecte s'il y a une erreur de connexion avec le serveur.}
        %         \item {askConnection(myIp : String, myPass : String) : void : Vérifie l'adresse IP puis démarre la connexion avec le serveur.}
        %     \end{itemize} 
        %     \newpage

    \paragraph{[Medium] PostmanSoftSonnette}

        \begin{figure} [H]
            \centering
            \includegraphics[scale=.5,max width=\textwidth,max height=.9\textheight]{architecture_candidate_AOP_anim_uml-class-PostmanSoftSonnette.eps}
            \caption{Diagramme de classe de PostmanSoftSonnette}
            \label{Classe-PostmanSoftSonnette}
        \end{figure}
    
            \subparagraph{Philosophie de Conception}%3.2.1.13.1
                
            La classe PostmanSoftSonnette a pour objectif de gérer l'envoi et la réception de messages à travers les sockets de communication entre SoftSonnette et AOP.
            C'est aussi l'objet qui met en place le serveur de cette communication.
            Il est en interaction avec les proxys de GUI et UISS pour l'encodage et le décodage des trames.            
            \subparagraph{Description structurelle}%3.2.1.13.2
                
            \subsubparagraph{Attributs}%3.2.1.13.2.1
            N.A.
            \subsubparagraph{Services offerts}%3.2.1.13.2.2
            \begin{itemize}
                \item {mySocket() : void : Enregistre le socket utilisé pour la communication avec SoftSonnette.}
                \item {connectToServer(ip : IP, port : Port) : void : Lance la connexion grâce au socket et à l'adresse IP entrés en paramètre.}
                \item {readMsg(short) : byte[ ] : Lis dans le socket la taille du message envoyé, puis lit le reste du message grâce au paramètre de la méthode.}
                \item {sendMsg(byte : byte[ ]) : void : Envoie un message à SoftSonnette.}
                \item {disconnect() : void : Ferme le socket.}
            \end{itemize} 
            \newpage

    \paragraph{[Medium] Dispatcher}

        \begin{figure} [H]
            \centering
            \includegraphics[scale=.5,max width=\textwidth,max height=.9\textheight]{architecture_candidate_AOP_anim_uml-class-Dispatcher.eps}
            \caption{Diagramme de classe de Dispatcher}
            \label{Classe-Dispatcher}
        \end{figure}
    
            \subparagraph{Philosophie de Conception}%3.2.1.13.1
                
            La classe Dispatcher a pour objectif de gérer l'envoi et la réception de messages à travers les sockets de communication.
            C'est aussi l'objet qui met en place le serveur de cette communication.
            \subparagraph{Description structurelle}%3.2.1.13.2
                
            \subsubparagraph{Attributs}%3.2.1.13.2.1
            N.A.
            \subsubparagraph{Services offerts}%3.2.1.13.2.2
            \begin{itemize}
                \item {dispatch(size : byte[], msg : int[]) : void : Dispatche les évènements reçus par le serveur.
                Prend en premier paramètre la taille du message et en deuxième paramètre le message.}
            \end{itemize} 
            \newpage   

    \paragraph{[Object] Protocol}

        \begin{figure} [H]
            \centering
            \includegraphics[scale=.5,max width=\textwidth,max height=.9\textheight]{architecture_candidate_AOP_anim_uml-class-Protocol.eps}
            \caption{Diagramme de classe de Protocol}
            \label{Classe-Protocol}
        \end{figure}
    
            \subparagraph{Philosophie de Conception}%3.2.1.13.1
            La classe Protocol est utilisée pour encoder les messages à envoyer et décoder les différents messages reçus.
            
            \subparagraph{Description structurelle}%3.2.1.13.2
                
            \subsubparagraph{Attributs}%3.2.1.13.2.1
            \begin{itemize}
                \item {cmd\_setCurrentTime : int : ID de la commande pour l'envoi de l'heure.}
                \item {cmd\_askCheckPass : int : ID de la commande pour la demande de vérification du mot de passe.}
                \item {cmd\_subscribeToVideoStream : int : ID de la commande pour la demande de stream.}
                \item {cmd\_askDoorState : int : ID de la commande pour la demande de l'état de la Porte.}
                \item {cmd\_askOpenDoor : int : ID de la commande pour l'ouverture de la Porte.}
                \item {cmd\_askEmployeeList : int : ID de la commande pour recevoir la liste des employés.}
                \item {cmd\_addEmployee : int : ID de la commande pour ajouter un employé.}
                \item {cmd\_deleteEmployee : int : ID de la commande pour supprimer un employé.}
            \end{itemize} 
            \subsubparagraph{Services offerts}%3.2.1.13.2.2
            \begin{itemize}
                \item {encodeMessage(cmdId : int, nbargs : int, data : String) : byte: Encode le message pour l'envoyer à SoftSonnette.}
                \item {decodeMessage(size : byte, messageDecode : byte[]): byte : Décode la trame reçue de SoftSonnette.}
            \end{itemize} 

\newpage   
        
    \paragraph{[Medium] PostmanVideo}%3.2.3.3

    \begin{figure} [H]
        \centering
        \includegraphics[scale=.5,max width=\textwidth,max height=.9\textheight]{architecture_candidate_AOP_anim_uml-class-PostmanVideo.eps}
        \caption{Diagramme de classe de PostmanVideo}
        \label{Classe-PostmanVideo}
    \end{figure}
        
        \subparagraph{Philosophie de Conception}%3.2.3.2.1
        La classe PostmanVideo est utilisée pour réceptionner du flux UDP provenant de \appliLin.

        \subparagraph{Description structurelle}%3.2.3.2.2
        
        \subsubparagraph{Attributs}%3.2.3.2.2.1
            N.A.
            \subsubparagraph{Services offerts}%3.2.3.2.2.2
            \begin{itemize}
                \item {readData() : byte[] : Lit le paquet de  données reçu et retourne sa taille.}
            \end{itemize} 

\newpage 

    \paragraph{[Cache] CacheCameraman}%3.2.3.4

    \begin{figure} [H]
        \centering
        \includegraphics[scale=.5,max width=\textwidth,max height=.9\textheight]{architecture_candidate_AOP_anim_uml-class-CacheCameraman.eps}
        \caption{Diagramme de classe de CacheCameraman}
        \label{Classe-CacheCameraman}
    \end{figure}

        \subparagraph{Philosophie de Conception}%3.2.3.4.1
        La classe CacheCameraman est utilisée pour le traitement des données du flux vidéo, et donc de la mise à jour de la vidéo.
        Le temps de synchronisation est difficile à évalué, car il dépend de la vitesse de streaming du Streamer, ainsi que de la vitesse de calcul du téléphone.
        La vidéo étant envoyé à 15 images par secondes, le temps de synchronisation minimum sera donc de 15 fois par secondes.

        CacheCameraman utilise un mediaCodec, c'est un objet permettant de décoder le flux vidéo en lui fournissant des trames.

        \subparagraph{Description structurelle}%3.2.3.4.2
        
            \subsubparagraph{Attributs}%3.2.3.4.2.1
            \begin{itemize}
            \item {dataFrame : byte[] : Stocke temporairement les données reçues.}
            \end{itemize}

            \subsubparagraph{Services offerts}%3.2.3.4.2.2
            \begin{itemize}
                \item {initializeMediaCodec() : void : Initialise le mediaCodec (avec les paramètres de la vidéo entrés en dur).}
                \item {callback(frameData : byte[]) : void : Envoie au mediaCodec les trames détectées pour qu'il les recompose.}
                \item {storeData(receivedData : byte[]) : void : Stocke les données temporairement avant de les donner au mediaCodec.}
                \item {stream(frameData : byte[]) : void : Traite la frame qui vient d'être détectée avant de l'envoyer au mediaCodec.}
                \item {runVideo() : void : Analyse les données reçues et les décompose en trames.}
                \item {askSubscribeVideoStream(enable : bool) : void : Demande le début du streaming de la vidéo.}
            \end{itemize}

\newpage 

    \paragraph{[Object] Door}%3.2.3.4

    \begin{figure} [H]
    \centering
    \includegraphics[scale=.5,max width=\textwidth,max height=.9\textheight]{architecture_candidate_AOP_anim_uml-class-Door.eps}
    \caption{Diagramme de classe de Door}
    \label{Classe-Door}
    \end{figure}

    \subparagraph{Philosophie de Conception}%3.2.3.4.1
    La classe Door permet d'avoir l'état de la Porte et de pouvoir ouvrir la Porte.
    \subparagraph{Description structurelle}%3.2.3.4.2

        \subsubparagraph{Attributs}%3.2.3.4.2.1
        N.A.
        \subsubparagraph{Services offerts}%3.2.3.4.2.2
        \begin{itemize}
            \item {getDoorState(state : bool) : void : Permet d'avoir l'état de la Porte.}
        \end{itemize}
\newpage      

    \paragraph{[Object] WeeklyCalendar}%3.2.3.3

    \begin{figure} [H]
        \centering
        \includegraphics[scale=.5,max width=\textwidth,max height=.9\textheight]{architecture_candidate_AOP_anim_uml-class-WeeklyCalendar.eps}
        \caption{Diagramme de classe de WeeklyCalendar}
        \label{Classe-WeeklyCalendar}
    \end{figure}

        \subparagraph{Philosophie de Conception}%3.2.3.2.1
        La classe WeeklyCalendar est utilisée pour créer et dessiner le calendrier.
        \subparagraph{Description structurelle}%3.2.3.2.2
        
        \subsubparagraph{Attributs}%3.2.3.2.2.1
            N.A.
            \subsubparagraph{Services offerts}%3.2.3.2.2.2
            \begin{itemize}
                \item {drawCalendar(canvas : Canvas) : void : Méthode de dessin du calendrier.}
                \item {editCalendar(name : String) : void : Coloriage des créneaux horaires à dessiner selon l'employé.}
                \item {onMeasure(widthMeasureSpec : int, heightMeasureSpec : int) : void: Méthode de mesure permettant le dessin du calendrier.}
            \end{itemize}

\newpage 

    \paragraph{[Cache] CacheEmployeeManager}%3.2.3.3

    \begin{figure} [H]
        \centering
        \includegraphics[scale=.5,max width=\textwidth,max height=.9\textheight]{architecture_candidate_AOP_anim_uml-class-CacheEmployeeManager.eps}
        \caption{Diagramme de classe de CacheEmployeeManager}
        \label{Classe-CacheEmployeeManager}
    \end{figure}

            \subparagraph{Philosophie de Conception}%3.2.3.2.1
            La classe CacheEmployeeManager est utilisée pour stocker temporairement les données des employés.
            Il est appelé lorsque l'utilisateur souhaite consulter le calendrier ou la liste des employés.
            Chaque ajout d'employé est ajouté dans le cache puis envoyé au serveur.
            Ce cache est synchronisé à chaque appel des écrans EMPLOYEE\_LIST\_ID et CALENDAR\_ID.
            \subparagraph{Description structurelle}%3.2.3.2.2
            
            \subsubparagraph{Attributs}%3.2.3.2.2.1
            \begin{itemize}
                \item {listEmployees : Employee[] : Liste courante des employés.}       
            \end{itemize} 
            \subsubparagraph{Services offerts}%3.2.3.2.2.2
            \begin{itemize}
                \item {askCalendarEmployee(name : String): int[] : Méthode permettant de préparer l'affichage du calendrier selon l'employé. Elle permet d'obtenir les heures de début et de fin de chaque jour.}       
                \item {askEmployeeList(message : byte[]): void : Méthode permettant d'obtenir la liste des employés.}             
                \item {setDataEmployee(idEmployee : int, name : String, firstName : String, role : int, workingHours : byte[]) : void : Méthode permettant de modifier la liste des employés dans le cache.}
                \item {getDataEmployee(): Employee[] : Fonction qui retourne la liste cache des employés.}
                \item {sendEmployee(name : String, firstName : String, picture : Picture, role : Role, workingHours : int[]): void : Méthode d'envoi des données de l'employé vers les données persistantes du projet.}
                \item {sendDeleteEmployee(employeeID : EmployeeID) : void : Méthode d'envoi de suppression d'un employé.}
            \end{itemize}

\newpage 

\subsubsection{Description des classes de \appliPo}%3.2.2
    
    \paragraph{[Object] Starter}%3.2.2.2

        \begin{figure} [H]
            \centering
            \includegraphics[scale=.5,max width=\textwidth,max height=.9\textheight]{architecture_candidate_SP_anim_uml-class-Starter.eps}
            \caption{Diagramme de classe de Starter}
            \label{Class-Starter}
        \end{figure}
    
        \subparagraph{Philosophie de Conception}%3.2.2.2.1
        La classe Starter correspond au thread principal de l'application SoftPorte.
        C'est elle qui possède la méthode main(), méthode appelée au lancement de l'application.
        \subparagraph{Description structurelle}%3.2.2.2.2
        \subsubparagraph{Attributs}%3.2.3.2.2.1
        N.A.
        \subsubparagraph{Services offerts}%3.2.3.2.2.2
        \begin{itemize}
            \item {main : Méthode d'entrée du programme}
            \item {run : Tâche de fond qui attend l'évènement indiquant la fin de l'execution}
        \end{itemize}
        
\newpage

        \subparagraph{Diagramme de séquence du démarrage et de l'arrêt de \appliPo}%3.2.2.2.3
        Le diagramme suivant représente la séquence d'initialisation d'\appliPo :
        \begin{figure} [H]
            \centering
            \includegraphics[scale=.5,max width=\textwidth,max height=.9\textheight]{diagramme_séquence_démarrage_SP.eps}
            \caption{Diagramme de séquence de l'initialisation de \appliPo}
            \label{Seq-Init-SP}
        \end{figure}
    
        Le diagramme suivant représente la séquence d'arrêt d'\appliA.
        \begin{figure} [H]
            \centering
            \includegraphics[scale=.5,max width=\textwidth,max height=.9\textheight]{diagramme_séquence_arret_SP.eps}
            \caption{Diagramme de séquence de l'arrêt de \appliPo}
            \label{Seq-Arret-SP}
        \end{figure}

\newpage
    
        \paragraph{[Medium] ProxyGUI}%3.2.3.

        \begin{figure} [H]
            \centering
            \includegraphics[scale=.5,max width=\textwidth,max height=.9\textheight]{architecture_candidate_SP_anim_uml-class-ProxyGUI.eps}
            \caption{Diagramme de classe de ProxyGUI}
            \label{Class-ProxyGUI}
            \end{figure}
        \subparagraph{Philosophie de Conception}%3.2.3.2.1
        La classe ProxyGUI est utilisé pour simuler le comportement de la classe GUI.
        Elle encode les trames à destination de GUI avant de les envoyer au postman.
        
        \subparagraph{Description structurelle}%3.2.3.2.2
        \subsubparagraph{Attributs}%3.2.3.2.2.1
        N.A.
        \subsubparagraph{Services offerts}%3.2.3.2.2.2
        \begin{itemize}
            \item {updateDoorState(doorState : bool): Informe GUI du nouvel état de la porte.}
        \end{itemize}

\newpage

        \paragraph{[Medium] ProxyUISS}%3.2.3.

        \begin{figure} [H]
            \centering
            \includegraphics[scale=.5,max width=\textwidth,max height=.9\textheight]{architecture_candidate_SP_anim_uml-class-ProxyUISS.eps}
            \caption{Diagramme de classe de ProxyUISS}
            \label{Class-ProxyUISS}
        \end{figure}
        \subparagraph{Philosophie de Conception}%3.2.3.2.1
        La classe ProxyUISS est utilisée pour simuler le comportement de la classe UISS.
        Elle encode les trames à destination de UISS avant de les envoyer au postman.
        
        \subparagraph{Description structurelle}%3.2.3.2.2
        \subsubparagraph{Attributs}%3.2.3.2.2.1
        N.A.
        \subsubparagraph{Services offerts}%3.2.3.2.2.2
        \begin{itemize}
            \item {updateDoorState(doorState): Informe UISS du nouvel état de la porte.}
        \end{itemize}

\newpage
    
        \paragraph{[Medium] PostmanSS}%3.2.3.
        \begin{figure} [H]
            \centering
            \includegraphics[scale=.5,max width=\textwidth,max height=.9\textheight]{architecture_candidate_SP_anim_uml-class-PostmanSS.eps}
            \caption{Diagramme de classe de PostmanSS}
            \label{Class-PostmanSS_SP}
        \end{figure}
        \subparagraph{Philosophie de Conception}%3.2.3.2.1
        La classe PostmanSS est en charge de la communication entre SoftPorte et SoftSonnette.
        C'est elle qui gère l'envoie et la réception de messages sur le Virtual UART.
        
        \subparagraph{Description structurelle}%3.2.3.2.2
        \subsubparagraph{Attributs}%3.2.3.2.2.1
        N.A.
        \subsubparagraph{Services offerts}%3.2.3.2.2.2
        \begin{itemize}
            \item {read() : Tâche de fond qui lis les messages reçu et les envoie au dispatcher.}
            \item {run() : Tâche de fond qui envoie les messages en attentes à SoftSonnette.}
            \item {send() : Fonction utilisée pour mettre en attente l'envoi d'un message vers SoftSonnette.}            
        \end{itemize}
        
\newpage

    \paragraph{[Medium] DispatcherSS}%3.2.3.
    \begin{figure} [H]
        \centering
        \includegraphics[scale=.5,max width=\textwidth,max height=.9\textheight]{architecture_candidate_SP_anim_uml-class-DispatcherSS.eps}
        \caption{Diagramme de classe de DispatcherSS}
        \label{Classe-DispatcherSS}
    \end{figure}
    \subparagraph{Philosophie de Conception}%3.2.3.2.1
    La classe DispatcherSS a pour objectif de recevoir et de décoder les différents messages reçus afin de les distribuer à DoorManager ou UISP.
    
    \subparagraph{Description structurelle}%3.2.3.2.2
    \subsubparagraph{Attributs}%3.2.3.2.2.1
    N.A.
    \subsubparagraph{Services offerts}%3.2.3.2.2.2
    \begin{itemize}
        \item {dispatch(encodedMessage byte[*]): Fonction utilisée pour décoder les messages reçus et les dispatcher aux objets destinataire.
        Cette fonction prend en argument le message brut encodé sur 2 octets}
    \end{itemize}
    \newpage


\subsection{Protocole de communication}

Les communications entre AOP et SoftSonnette se font via socket TCP/IP et celles entre SoftSonnette et SoftPorte via VirtualUART. 

\subsubsection{Protocole de communication entre \appliLin~et \appliA}%3.3.1

    \paragraph{Formalisation du protocole}%3.3.1.1

    Le protocole de communication entre SoftSonnette et AOP est défini comme une suite d'octets selon la forme suivante:

    \begin{center}
        \begin{tabular}{|c|c|c|c|}
            \hline
            TAILLE & CMD & NB\_ARGS & DONNEES \\
            \hline
        \end{tabular}
    \end{center}

    Ce protocole est dit symétrique.
    Cela signifie que l'encodage de la trame est là même d'un bout à l'autre de la communication.
    Une trame se compose ainsi :
    
    \begin{itemize}
        \item {TAILLE : Nombre entier codé sur deux octets donnant la taille totale de la trame en octet.
        Cette taille comprend ces 2 octets de TAILLE.}
        \item {CMD : Valeur pouvant aller de 0x00 à 0x0B et indiquant la commande que l'on veut exécuter.}
        \item {NB\_ARGS : Indique le nombre d'arguments situés dans la partie données, codé sur un octet.}
        \item {DONNEES : La partie données a une taille variable et est facultative, elle est construite de cette façon:}
       
    \end{itemize} 

        \begin{center}

            \begin{tabular}{|c|c|}
                \hline
                TAILLE\_ARG & ARG \\
                \hline
            \end{tabular}
        \end{center}

        Précisons que la taille variable de la partie "données" dépend du nombre d'arguments que l'on veut transmettre.
        Voici la description de ses différentes parties:

        \begin{itemize}
            \item {TAILLE\_ARG : Entier codé sur deux octets donnant la taille totale de l'argument en octet.
            Contrairement à TAILLE, TAILLE\_ARG ne compte pas sa propre taille.}
            \item {ARG : Chaîne de caractères contenant l'argument.}
        \end{itemize} 
        
        Cette dernière structure est donc à répéter selon la valeur de NB\_ARGS, et donc de CMD.
        Par exemple, une trame ne contenant pas d'argument avec un CMD de 5 prend la forme suivante : 

        \begin {table}[H]
        \center
        \begin{tabular}{|c|c|c|}
            \hline
            \textbf{TAILLE} & \textbf{CMD} & \textbf{NB\_ARGS} \\
            \hline
            0x00 0x04 & 0x05 & 0x00 \\
            \hline
        \end{tabular}
        \end{table}

\newpage

        Tableau récapitulatif des différentes commandes: 

            \begin {table}[H]
            \center
            \begin{tabularx}{\textwidth}{|p{0.5cm}|X|X|}
                \hline
                \textbf{ID} & \textbf{CMD} & \textbf{Commentaire} \\
                \hline
                0 & ERR & Trame réservée à la mention d'une erreur de réception \\
                \hline
                1 & SUBSCRIBE\_VIDEO & GUI demande à recevoir la vidéo de SoftSonnette \\
                \hline 
                2 & ASK\_CHECK\_PASS & ConnectionManager demande à Guard si le mot de passe entré sur l'application est correct \\
                \hline
                3 & VALIDATE\_PASS & Réponse de Guard si le mot de passe est valide \\
                \hline
                4 & SET\_TIME & ConnectionManager modifie l'heure courante sur Clock \\
                \hline
                5 & ASK\_OPEN\_DOOR & Demande d'ouverture de la Porte depuis AOP vers DoorManager\\
                \hline
                6 & ASK\_DOOR\_STATE & Demande si la Porte est ouverte ou fermée \\
                \hline
                7 & UPDATE\_DOOR\_STATE & DoorManager met à jour l'état de la porte sur l'écran d'AOP \\
                \hline
                8 & ADD\_EMPLOYEE & AOP demande à ajouter un employé dans les données persistantes situées sur la carte \\
                \hline
                9 & DELETE\_EMPLOYEE & AOP demande à supprimer un employé des données persistantes situées sur la carte \\
                \hline
                10 & ASK\_EMPLOYEE\_LIST & AOP demande la liste des employés à SoftSonnette \\
                \hline
                11 & SET\_EMPLOYEE\_LIST & Réponse d'EmployeeManager sur la liste d'employés \\
                \hline
            \end{tabularx}
            \end{table}

            \begin {table}[H]
            Les différents arguments selon la CMD sont les suivants : 
            \center
            \begin{tabularx}{\textwidth}{|X|p{1.9cm}|p{1.3cm}|p{1.3cm}|p{1.1cm}|p{1.1cm}|}
                \hline
                \textbf{CMD} & \textbf{ARG1} & \textbf{ARG2} & \textbf{ARG3} & \textbf{ARG4} & \textbf{ARG5}\\
                \hline
                ERR & X & X & X & X & X \\
                \hline
                SUBSCRIBE\_VIDEO & Bool & X & X & X & X \\
                \hline 
                ASK\_CHECK\_PASS & Password & X & X & X & X \\
                \hline
                VALIDATE\_PASS & Bool & X & X & X & X \\
                \hline
                SET\_TIME & Time & X & X & X & X \\
                \hline
                ASK\_OPEN\_DOOR & X & X & X & X & X \\
                \hline
                ASK\_DOOR\_STATE & X & X & X & X & X \\
                \hline
                UPDATE\_DOOR\_STATE & Bool & X & X & X & X \\
                \hline
                ADD\_EMPLOYEE & Name & Surname & Picture & Role & Hours \\
                \hline
                DELETE\_EMPLOYEE & ID & X & X & X & X \\
                \hline
                ASK\_EMPLOYEE\_LIST & X & X & X & X & X \\
                \hline
                SET\_EMPLOYEE\_LIST & EmployeeID & Name & Surname & Role & Hours \\
                \hline
            \end{tabularx}
            \end{table}

\newpage

    Les tailles des différents types sont précisées ci-dessous : 

    \begin{itemize}
        \item {Bool : Booléen codé sur un octet, prend les valeurs 0x00 ou 0x01.}
        \item {Password : Chaîne de caractères codé sur 4 octets pour les quatre chiffres qu'il contient.}
        \item {Time : Chaîne de caractères codé sur 7 octets : deux pour l'année, puis un pour le mois, le jour, l'heure, la minute et la seconde.}
        \item {EmployeeID : Entier codé sur 1 octet.}
        \item {Name : Chaîne de caractères de taille variable, codé entre 1 et 12 octets selon TAILLE\_ARG.}
        \item {Surname : Chaîne de caractères de taille variable, codé entre 2 et 12 octets selon TAILLE\_ARG.}
        \item {Picture : Chaîne d'octets de taille variable selon TAILLE\_ARG.}
        \item {Rôle : Entier codé sur 1 octet.}
        \item {Hours : Chaîne de caractères, codé sur 14 octets soit 7x2 créneaux horaires.}
       
    \end{itemize} 

    \paragraph{Exemples}%3.3.1.2

        \subparagraph{Demande de connexion entre \appliA~et \appliLin}
    
        Le diagramme de séquence ci-dessous décrit le processus de connexion entre AOP et SoftSonnette.
        A noter que la phase de synchronisation du temps qui survient après validatePass() n'est pas présentée par soucis de lisibilité.

        \begin{figure} [H]
            \centering
            \includegraphics[scale=.5,max width=\textwidth,max height=.9\textheight]{diagramme_séquence_détaillé_se_connecter_AOP}
            \caption{Processus de connexion entre AOP et SoftSonnette du point de vue de AOP}
            \label{Sequ-seConnecter-AOP}
        \end{figure}

        \begin{figure} [H]
            \centering
            \includegraphics[scale=.5,max width=\textwidth,max height=.9\textheight]{diagramme_séquence_détaillé_se_connecter_softSonnette}
            \caption{Processus de connexion entre AOP et SoftSonnette du point de vue de SoftSonnette}
            \label{Sequ-seConnecter-softSonnette}
        \end{figure}

        Dans ce processus de connexion, une requête ASK\_CHECK\_PASS correspondant à l'appel de la méthode askCheckPass() est envoyée.
        La requête est encodée de cette manière : 

        \begin {table}[H]
        \center
        \begin{tabularx}{\textwidth}{|p{2cm}|p{2cm}|p{3cm}|p{3cm}|X|}
            \hline
            \textbf{TAILLE} & \textbf{CMD} & \textbf{NB\_ARGS} & \textbf{TAILLE\_ARG} & \textbf{ARG} \\
            \hline
            0x00 0x0A & 0x02 & 0x01 & 0x00 0x04 & 0x31 0x32 0x33 0x34\\
            \hline
        \end{tabularx}
        \end{table}

        Dans cet exemple : 

        \begin{itemize}
            \item {TAILLE vaut 0x00 0x0A ce qui signifie que la trame a une longueur de 10 octets.}
            \item {CMD vaut 0x02 ce qui équivaut à la requête ASK\_CHECK\_PASS.}
            \item {NB\_ARGS vaut 0x01 car il n'y a qu'un seul argument à suivre.}
            \item {TAILLE\_ARG vaut 0x00 0x04 indiquant que ARG a une taille de 4 octets.}
            \item {ARG vaut 0x31 0x32 0x33 0x34 correspondant au code en ASCII de la chaîne "1234".}
        \end{itemize} 

\newpage

\subsubsection{Protocole de communication entre \appliLin~et \appliPo}%3.3.2

    \paragraph{Formalisation du protocole}%3.3.2.1

    Le protocole de communication entre SoftSonnette et SoftPorte est défini comme une suite d'octets selon la forme suivante:

    \begin{center}
        \begin{tabular}{|c|c|}
            \hline
            CMD & TARGET \\
            \hline
        \end{tabular}
    \end{center}

    \begin{itemize}
        \item {CMD : Valeur pouvant aller de 0x00 à 0x09 et indiquant la commande que l'on veut exécuter.}
        \item {TARGET : Indique la cible du message entre AOP et SoftSonnette, codé sur un octet. 
        Elle peut prendre les valeurs SOFTS\_ID = 0x00, AOP\_ID = 0x01 et SOFTP\_ID = 0x02.}
       
    \end{itemize} 

    Tableau récapitulatif des différentes commandes: 

    \begin {table}[H]
    \center
    \begin{tabularx}{\textwidth}{|p{0.5cm}|X|X|}
        \hline
        \textbf{ID} & \textbf{CMD} & \textbf{Commentaire} \\
        \hline
        0 & ERR & Trame réservée à la mention d'une erreur de réception \\
        \hline
        1 & ASK\_OPEN\_DOOR & Demande d'ouverture de la Porte depuis SoftSonnette ou AOP vers DoorManager\\
        \hline
        2 & UPDATE\_DOOR\_STATE\_TRUE & DoorManager met à jour l'état de la Porte (ouverte) sur l'écran d'AOP ou sur l'écran d'UISS\\
        \hline
        3 & UPDATE\_DOOR\_STATE\_FALSE & DoorManager met à jour l'état de la Porte (fermée) sur l'écran d'AOP ou sur l'écran d'UISS\\
        \hline
        4 & ASK\_DOOR\_STATE & Demande si la Porte est ouverte ou fermée \\
        \hline
        5 & SIGNAL\_NOT\_ALLOWED & Bouncer signale à UISP l'interdiction d'ouvrir la Porte \\
        \hline
        6 & LAUNCH\_SP & Lance l'application SoftPorte \\
        \hline
        7 & QUIT\_SP & Quitte l'application SoftPorte \\
        \hline
        8 & ASK\_START\_COM & Envoie un signal à UISS \\
        \hline
        9 & ACK\_START\_COM & Réponse de UISS à UISP \\
        \hline
    \end{tabularx}
    \end{table}

    \begin {table}[H]
    Les différents arguments selon la CMD sont les suivants : 
    \center
    \begin{tabularx}{\textwidth}{|X|X|}
        \hline
        \textbf{CMD} & \textbf{TARGET} \\
        \hline
        ERR & SOFTS\_ID / AOP\_ID / SOFTP\_ID \\
        \hline
        ASK\_OPEN\_DOOR & SOFTP\_ID \\
        \hline 
        UPDATE\_DOOR\_STATE\_TRUE & SOFTS\_ID / AOP\_ID \\
        \hline
        UPDATE\_DOOR\_STATE\_FALSE & SOFTS\_ID / AOP\_ID \\
        \hline
        ASK\_DOOR\_STATE & SOFTP\_ID \\
        \hline
        SIGNAL\_NOT\_ALLOWED & SOFTP\_ID \\
        \hline
        LAUNCH\_SP & SOFTP\_ID \\
        \hline
        QUIT\_SP & SOFTP\_ID \\
        \hline
        ASK\_START\_COM & SOFTS\_ID \\
        \hline
        ACK\_START\_COM & SOFTP\_ID \\
        \hline
    \end{tabularx}
    \end{table}

    \paragraph{Exemples}%3.3.2.2

        Le diagramme de séquence ci-dessous décrit le processus d'ouverture de la Porte.

        \begin{figure} [H]
            \centering
            \includegraphics[scale=.5,max width=\textwidth,max height=.9\textheight]{diagramme_séquence_détaillé_ouvrir_porte}
            \caption{Processus d'ouverture de la Porte}
            \label{Sequ-ouvrirPorte}
        \end{figure}

        Dans cet exemple, une requête UPDATE\_DOOR\_STATE\_TRUE correspondant à l'appel de la méthode updateDoorState(true) sur l'objet GUI est envoyée.
        Cette requête est encodée de telle manière : 

        \begin {table}[H]
        \center
        \begin{tabular}{|c|c|}
            \hline
            \textbf{CMD} & \textbf{TARGET} \\
            \hline
            0x02 & 0x01 \\
            \hline
        \end{tabular}
        \end{table}

        Dans cet exemple : 

        \begin{itemize}
            \item {CMD vaut 0x02 ce qui équivaut à la requête UPDATE\_DOOR\_STATE\_TRUE.}
            \item {TARGET vaut 0x01 ce qui indique que la cible est un objet d'AOP, ici GUI.}
        \end{itemize} 


\newpage
\subsection{Gestion du multitâche}
\subsubsection{Identification des accès concurrents dans \appliLin}%3.4.1

Dans un soucis de lisibilité, le nom des classes est abrégé dans le tableau \ref{tableau-acces-concurrents-1} répertoriant les accès concourants.
Les acronymes utilisés sont répertoriés dans le tableau ci-dessous:

\begin{table}[H]
    \centering
    \begin{tabularx}{\textwidth}{|X|p{2cm}|X|p{2cm}|}
      \hline 
         \textbf{Tâche}  &  \textbf{Abrégé} & \textbf{Autre} & \textbf{Abrégé} \\ 
        \hline
        RecognitionAI  & RAI & Clock &  Cl \\ 
        \hline
        Bouncer  & Bo & EmployeeManager &  EM \\ 
        \hline
        Cameraman  & Ca & ProxyGUI &  PG \\ 
        \hline
        UISS  &  UISS & ProxyConnectionManager &  PCM \\ 
        \hline
        Starter  & Sta & ProtocolSP &  ProSP \\ 
        \hline
        PostmanAOP  & PAOP & Guard &  G \\ 
        \hline
        PostmanSP  & PSP & ProxyDoorManager &  PDM \\ 
        \hline
        DispatcherAOP  & DAOP & DataEmployee &  DE \\ 
        \hline
        DispatcherSP  & DSP & ProtocolSS &  PSS \\ 
        \hline
        Streamer & Str & ProxyUISP &  PUISP \\ 
        \hline

    \end{tabularx}
    \caption{Tableau des classes abrégées  dans \appliLin}
    \label{tableau-abreger-classe-1}
  \end{table}

\begin{table}[H]
    \centering
    \begin{tabularx}{\textwidth}{|X|X|X|X|X|X|X|X|X|X|X|}
      \hline 
            & Cl & EM & PG & PCM & ProSP & G & PDM & DE & PSS & PUISP \\ 
        \hline
            RAI  &   &   &   &   &   &   &   &   &   &      \\ 
        \hline
            Bo   & X & X & X  &   &   &   & X  & X &   & X     \\ 
        \hline
            Ca   &   &   &   &   &  &   &   &   &   &    \\ 
        \hline
            UISS &   &   &   &   &   &   &   &   &   &  X  \\ 
        \hline
            Sta  &   &   &   &   &   &   &   &   &   &    \\ 
        \hline                                  
            PAOP &   &   &   &   &   &   &   &   &   &      \\ 
        \hline
            PSP  &   &   &   &   &   &   &   &   &   &    \\ 
        \hline
            DAOP & X & X &  X & X &   & X & X & X  & X &    \\ 
        \hline
            DSP  &   &   & X &   & X  &   &   &   &   &    \\ 
        \hline
            Str  &   &   &  &   &   &   &   &   &   &    \\ 
        \hline
    \end{tabularx}
    \caption{Tableau des accès concurrents dans \appliLin}
    \label{tableau-acces-concurrents-1}
  \end{table}

  Les classes DispatcherAOP et Bouncer appellent concurremment Clock et EmployeeManager.
  Pour éviter tout soucis ces deux classes possèdent l'attribut "protected" indiquant aux développeur de mettre en place des stratégies de verrouillage des ressources critiques permettant ainsi d'éviter les accès concourants.
  

\subsubsection{Identification des accès concurrents dans \appliPo}%3.4.2

\begin{table}[H]
  \centering
  \begin{tabularx}{\textwidth}{|X|X|X|X|X|}
    \hline 
                        & ProxyGUI  & ProxyUISS & UISP  & DispatcherSS  \\ 
      \hline
          DoorManager   &     X     &     X     &       &               \\ 
      \hline
          PostmanSS     &           &           &   X   &       X       \\ 
      \hline
  \end{tabularx}
  \caption{Tableau des accès concurrents dans \appliPo}
  \label{tableau-acces-concurrents-2}
\end{table}

Il n'y a pas de problème de concurrence.

\newpage
\subsubsection{Identification des accès concurrents dans \appliA}%3.4.3

Dans un soucis de lisibilité, le nom des classes est abrégé dans le tableau \ref{tableau-acces-concurrents-3} répertoriant les accès concourants.
Les acronymes utilisés sont répertoriés dans le tableau ci-dessous:

\begin{table}[H]
    \centering
    \begin{tabularx}{\textwidth}{|X|p{2cm}|X|p{2cm}|}
      \hline 
         \textbf{Tâche}  &  \textbf{abréger} & \textbf{Autre} & \textbf{abréger} \\ 
        \hline
        Communication  & Com & Protocol &  Pr \\ 
        \hline
        PostmanVideo  & PV & ConnectionManager &  CM \\ 
        \hline
        PostmanSoftSonnette  & PS & Dispatcher &  Di \\ 
        \hline
        GUI  & GUI & CacheEmployeeManager  &  CEM \\ 
        \hline
        CacheCameraman  &  CCa & WeeklyCalendar  &  WC \\ 
        \hline
          &  & ProxyDoorManager   &  PDM \\ 
        \hline
          &  & ProxyClock   &  PC \\ 
        \hline
          &  & ProxyGuard   &  PG \\ 
        \hline 


    \end{tabularx}
    \caption{Tableau des classe abrégées dans \appliA}
    \label{tableau-abreger-classe-2}
  \end{table}

\begin{table}[H]
    \centering
    \begin{tabularx}{\textwidth}{|X|X|X|X|X|X|X|X|X|X|}
      \hline 
                 & Pr & CM & Di  & CEM  & WC & PDM & PC & PG \\ 
        \hline
            GUI  &    &  X &  X  &  X   &  X &  X  & X  & X  \\ 
        \hline
            Com  &  X &  X &  X  &  X   &  X &  X  & X  & X  \\ 
        \hline
            PV   &    &    &     &      &    &     &    &    \\ 
        \hline
            PS   &    &    &     &      &    &     &    &    \\ 
        \hline
            CCa  &    &    &     &      &    &     &    &    \\      
        \hline
    \end{tabularx}
    \caption{Tableau des accès concurrents dans \appliA}
    \label{tableau-acces-concurrents-3}
  \end{table}


  La tâche MainActivity, appartenant à GUI, est le cœur d'AOP. 
  Or, MainActivity n’est pas susceptible de manipuler des ressources critiques puisqu'elle délègue ses tâches aux objets concernés, il n’y a donc pas de problèmes de concurrence ici.
  
  La tâche Communication se trouve en haut de la hiérarchie, régissant les demandes, les appels à ses méthodes de lecture et d’écriture, et gère le socket de communication, comme son nom l’indique elle communique avec tous les objets.
    
  Les tâches PostmanVideo, PostmanSoftSonnette et CacheCameraman n’accèdent à aucun objet “non actif”, aucun problème de concurrence n’est à détecter ici.
  


%-----------------------------------------
%      Partie IV
%-----------------------------------------

\newpage
\setglossarysection{section}
\printglossary[title=Dictionnaire du domaine] \label{dictionnaireDomaine}
\glsaddallunused

\newpage
\listoffigures \label{TableOfFigure} % Affiche la table des figures

%\printbibliography[heading=bibnumbered, label=bibliography] % Affiche les références/bibliographie
%\nocite{*} % Afficher toutes les références (même celles non utilisées)

\clearpage % Ajoute une page vide
\null
\thispagestyle{empty}%
\addtocounter{page}{-1}% % Ne compte pas dans les numéros de page
\BgThispage
\end{document}