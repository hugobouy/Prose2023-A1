\newpage

\subsection{Grands principes de fonctionnement}

Cette partie représente le fonctionnement général de SoftSonnette, SoftPorte, AOP et leurs interactions respectives.
Les diagrammes de séquences présentés sont : 
\begin {itemize}
    \item Présenter les capacités de la STM32MP15 au travers d'une application de Sonnette Connectée
    \item Initialiser Board
    \item Se connecter
    \item Demander à entrer
    \item Regarder vidéo
    \item Consulter calendrier
    \item Contrôler Porte à distance
    \item Consulter liste des employés
    \item Quitter SàE
\end {itemize}

Les diagrammes de séquence présentent l'histoire nominal des Cas d'Utilisation du dossier de spécification \refSpec. 
Ils ne traitent donc pas les erreurs de fonctionnement.

\newpage

\subsubsection{Présenter les capacités de la STM32MP15 au travers d'une application de Sonnette Connectée} %2.2.1
Ce diagramme représente le scénario nominal de l'ensemble des CUs du dossier de spécification \refSpec.
Il s'agit du CU Stratégique du SàE.\\

\begin{figure} [H]
    \centering
    \includegraphics[scale=.5,max width=\textwidth,max height=.9\textheight]{architecture_candidate_anime_uml-sequence-CU_Strategique}
    \caption{Diagramme de séquence du scénario nominal}
    \label{CU-Stratégique}
\end{figure}

\newpage

\subsubsection{Initialiser Board} %2.2.2
Ce diagramme représente le scénario nominal du CU "Initialiser Board" dans le dossier de spécification \refSpec.\\

\begin{figure} [H]
    \centering
    \includegraphics[scale=.5,max width=\textwidth,max height=.9\textheight]{architecture_candidate_anime_uml-sequence-CU_InitialiserBoard}
    \caption{Diagramme de séquence de l'initialisation de Board}
    \label{CU-InitialiserBoard}
\end{figure}

\newpage

\subsubsection{Se connecter} %2.2.3
Ce diagramme représente le scénario nominal du CU "Se connecter" dans le dossier de spécification \refSpec.\\

\begin{figure} [H]
    \centering
    \includegraphics[scale=.5,max width=\textwidth,max height=.9\textheight]{architecture_candidate_anime_uml-sequence-CU_SeConnecter}
    \caption{Diagramme de séquence de la connexion entre AOP et SoftSonnette}
    \label{CU-SeConnecter}
\end{figure}

\subsubsection{Demander à entrer} %2.2.4
Ce diagramme représente le scénario nominal du CU "Demander à entrer" dans le dossier de spécification \refSpec.\\

\begin{figure} [H]
    \centering
    \includegraphics[scale=.5,max width=\textwidth,max height=.9\textheight]{architecture_candidate_anime_uml-sequence-CU_Demander_Entrer}
    \caption{Diagramme de séquence du CU "Demander à entrer" }
    \label{CU-Entrer}
\end{figure}

\subsubsection{Ouvrir Porte} %2.2.5
Ce diagramme représente le scénario nominal du CU "Ouvrir Porte" dans le dossier de spécification \refSpec.\\

\begin{figure} [H]
    \centering
    \includegraphics[scale=.5,max width=\textwidth,max height=.9\textheight]{architecture_candidate_anime_uml-sequence-CU_OuvrirPorte}
    \caption{Diagramme de séquence du CU "Ouvrir Porte"}
    \label{CU-OuvrirPorte}
\end{figure}

\subsubsection{Regarder vidéo} %2.2.6
Ce diagramme représente le scénario nominal du CU "Regarder vidéo" dans le dossier de spécification \refSpec.\\

\begin{figure} [H]
    \centering
    \includegraphics[scale=.5,max width=\textwidth,max height=.9\textheight]{architecture_candidate_anime_uml-sequence-CU_Regarder_Video}
    \caption{Diagramme de séquence du CU "Regarder vidéo"}
    \label{CU-Regarder_Video}
\end{figure}

\subsubsection{Consulter calendrier} %2.2.7
Ce diagramme représente le scénario nominal du CU "Consulter calendrier" dans le dossier de spécification \refSpec.\\

\begin{figure} [H]
    \centering
    \includegraphics[scale=.5,max width=\textwidth,max height=.9\textheight]{architecture_candidate_anime_uml-sequence-CU_Consulter_Calendrier}
    \caption{Diagramme de séquence du CU "Consulter calendrier"}
    \label{CU-Consulter_Calendrier}
\end{figure}

\subsubsection{Contrôler Porte à distance} %2.2.8
Ce diagramme représente le scénario nominal du CU "Contrôle Porte à distance" dans le dossier de spécification \refSpec.\\

\begin{figure} [H]
    \centering
    \includegraphics[scale=.5,max width=\textwidth,max height=.9\textheight]{architecture_candidate_anime_uml-sequence-CU_Contrôler_Porte_A_Distance}
    \caption{Diagramme de séquence du CU "Contrôle Porte à distance"}
    \label{CU-Contrôler_Porte_A_Distance}
\end{figure}

\subsubsection{Consulter liste employés} %2.2.9
Ce diagramme représente le scénario nominal du CU "Consulter liste employés" dans le dossier de spécification \refSpec.\\

\begin{figure} [H]
    \centering
    \includegraphics[scale=.5,max width=\textwidth,max height=.9\textheight]{architecture_candidate_anime_uml-sequence-CU_Consulter_Liste}
    \caption{Diagramme de séquence du CU "Consulter liste employés"}
    \label{CU-Consulter_Liste}
\end{figure}

\subsubsection{Quitter SàE} %2.2.10
Ce diagramme représente le scénario nominal du CU "Quitter SàE" dans le dossier de spécification \refSpec.\\

\begin{figure} [H]
    \centering
    \includegraphics[scale=.5,max width=\textwidth,max height=.9\textheight]{architecture_candidate_anime_uml-sequence-CU_QuitterSAE}
    \caption{Diagramme de séquence du CU "Quitter SàE"}
    \label{CU-Quitter}
\end{figure}