\newpage

\subsection{Description des composants}

\subsubsection{Description des types manipulés entre composants}%2.3.1

\begin{itemize}
    \item ConnectionStatus : Énumération représentant l'état de la connexion entre \appliA~et \appliLin.
    ConnectionStatus peut prendre pour valeur : PASS\_KO, CONNECT\_KO et ALL\_OK.
    \item Day : Énumération représentant un jour de la semaine.
    Day peut prendre pour valeur : MONDAY, TUESDAY, WEDNESDAY, THURSDAY, FRIDAY, SATURDAY et SUNDAY.
    \item \hyperlink{emp}{\textit{Employee}} : Structure contenant les informations de l'employé.
    Employee contient : firstName , name, role (Role), id (EmployeeID), workingHours~[~]~[~] et picture (chemin local vers l'image).
    \begin{itemize}
        \item \hyperlink{prenom}{\textit{firstName}} est une chaîne de caractère comprenant entre 2 et 12 caractères alphanumériques encodés en UTF-8.
        \item \hyperlink{nom}{\textit{name}} suit les mêmes règles que firstName à la différence qu'il peut être comprit entre 1 et 12 caractères alphanumériques.
        \item workingHours~[~]~[~] est une matrice de 7 lignes (jours de la semaine) et 2 colonnes (horaires début et fin de journée).
    \end{itemize}    
    \item \hyperlink{idEmp}{\textit{EmployeeID}} : Un entier qui représente le numéro d'un employé allant jusqu'à MAX\_EMPLOYE qui vaut 10. La constante "UNKNOWN" vaut -1, et signifie que l'employé n'est pas reconnu.
    \item \hyperlink{hor}{\textit{Hour}} : Structure contenant un horaire de début ou de fin de journée, enregistrée au format 24 h avec une précision de 30 minutes.
    Hour contient l'heure et la minute de l'horaire.
    \item \hyperlink{IP}{\textit{Ip}} : Chaîne de caractères de type IPV4 au format X.X.X.X où X est un entier compris entre 0 et 255. 
    Correspond à l'adresse de la Board.
    \item \hyperlink{mdp}{\textit{Password}} : Chaîne de caractères. Les caractéristiques sont définies dans le dictionnaire des domaines.
    \item \hyperlink{photo}{\textit{Picture}} : Image du visage d'un employé qui sera utilisée pour la reconnaissance faciale.
    \item PopUpID : Énumération représentant les différentes pop-up affichables sur \appliA.
    PopUpID peut prendre pour valeur : WAITING\_ID, ERROR\_PASS\_ID, ERROR\_CONNECT\_ID, DELETE\_ID et VIDEO\_NOT\_AVAILABLE.  
    \item \hyperlink{rol}{\textit{Role}} : Énumération représentant les différents types d'employés pour délimiter leurs horaires.
    Role peut prendre pour valeur : E\_MORNING, E\_DAY, E\_EVENING, E\_SECURITY et E\_SPECIAL.
    \item ScreenID : Énumération représentant les différents écrans affichables sur \appliA.
    ScreenID peut prendre pour valeur : STARTER\_ID, HOME\_ID, ADD\_EMPLOYEE\_ID, ADD\_SPECIAL\_EMPLOYEE\_ID, EMPLOYEE\_LIST\_ID, DOOR\_CONTROL\_ID, CALENDAR\_ID et VIDEO\_ID.
    \item Time : respecte la norme ISO 8601 : YYYY-MM-DD-HH: MM: SS.
    Time se met à jour lors de la connexion entre \appliA~et \appliLin.
    \item SSState : Énumération des affichages de \appliLin.
    SSState peut prendre pour valeur: STATE\_IDLE, STATE\_WEBCAM\_CONNECTED, STATE\_WEBCAM\_NOT\_CONNECTED, STATE\_ERROR\_COM.
    À noter que STATE\_IDLE correspond à l'affichage vide de l'écran de SoftSonnette (sans affichage vidéo et sans label d'erreur).
\end{itemize}

\newpage

\subsubsection{Diagramme de classe}%2.3.2
Est représenté ci-dessous le diagramme de classe de PSC, composé de 11 classes avec pour chacune leurs attributs et leurs méthodes décrites dans la suite du document.

\begin{figure} [H]
    \centering
    \includegraphics[scale=.5,max width=\textwidth,max height=.9\textheight]{diagramme_de_classe}
    \caption{Diagramme de classes du SàE}
    \label{Diagramme_de_classes}
\end{figure}

\newpage

\subsubsection{Description des classes}%2.3.3
    \paragraph{[Object] ConnectionManager}%2.3.3.1
    \begin{figure} [H]
        \centering
        \includegraphics[scale=.5,max width=\textwidth,max height=.9\textheight]{architecture_candidate_anime_uml-class-ConnectionManager.eps}
        \caption{Diagramme de classe représentant ConnectionManager}
        \label{Classe_ConnectionManager}
    \end{figure}
        \subparagraph{Philosophie de conception}%2.3.3.1.1
        La classe ConnectionManager permet de transiter les informations de connexion.
        Elle permet aussi d'assurer que les applications sont connectées et d'agir en conséquence.
        Cette classe interagit avec Guard, Clock, et GUI.
        \subparagraph{Description structurelle}%2.3.3.1.2
            \subsubparagraph{Attributs}%2.3.3.1.2.1
            \begin{itemize}
                \item connectionState : bool : Exprime l'état de la connexion actuelle.
            \end{itemize}
            \subsubparagraph{Services offerts}%2.3.3.1.2.2
            \begin{itemize}
                \item {checkConnection() : void : Tâche de fond qui vérifie si la connexion est toujours opérationnelle.
                Dans le cas d'une perte de connexion cette fonction informe les objets intéressés.}
                \item {getLocalTime() : Time : Récupère l'heure et la date du téléphone}
                \item {askConnection(ip : IP, pass : Password) : void : Envoie les informations de connexion vers Guard.}
                \item {validatePass(passValidated : bool) : void : Reçoit l'information de connexion qui valide ou non la connexion.}
            \end{itemize}
\newpage
        \subparagraph{Description comportementale}%2.3.3.1.3
        La MAE suivante permet d'expliquer le fonctionnement de l'objet actif ConnectionManager :
        \begin{figure} [H]
            \centering
            \includegraphics[scale=.5,max width=\textwidth,max height=.9\textheight]{architecture_candidate_anime_uml-connectionManager-SM.eps}
            \caption{Machine à état représentant ConnectionManager}
            \label{MaE_ConnectionManager}     
        \end{figure}

\newpage

    \paragraph{[IHM] GUI}%2.3.3.2
    \begin{figure} [H]
        \centering
        \includegraphics[scale=.5,max width=\textwidth,max height=.9\textheight]{architecture_candidate_anime_uml-class-GUI.eps}
        \caption{Diagramme de classe représentant GUI}
        \label{Classe_GUI}
    \end{figure}
        \subparagraph{Philosophie de conception}%2.3.3.2.1
        La classe GUI permet de gérer les interfaces utilisateurs entre le Démonstrateur et AOP en affichant les différentes vues de AOP.
        Cette classe interagit avec ConnectionManager, DoorManager, EmployeeManager et Cameraman.
        \subparagraph{Description structurelle}%2.3.3.2.2
            \subsubparagraph{Attributs}%2.3.3.2.2.1
            \begin{itemize}
                \item myPass : Password : Contient le mot de passe entré lors de la connexion.
                \item myIP : IP : Contient l'adresse IP entrée lors de la connexion.
                \item myName : String : Contient le nom entré lors de l'ajout d'un employé.
                \item myFirstName : String : Contient le prénom entré lors de l'ajout d'un employé.
                \item myPicture : Picture : Contient la photo d'un employé à ajouter.
                \item myRole : Role : Contient le rôle d'un employé lors de son ajout.
                \item myWorkingHours : Hour[] : Contient les horaires de l'employé pour les 7 jours de la semaine.
                \item employeeList : Employee[] : Contient la liste des employés envoyés depuis la Board.
            \end{itemize}

            \subsubparagraph{Services offerts}%2.3.3.2.2.2
            \begin{itemize}
                \item displayScreen(screenID : ScreenID) : void : Affiche l'écran entré en paramètre de la fonction.
                \item displayPopUp(popUpID : PopUpID) : void : Affiche la PopUp entrée en paramètre de la fonction.
                \item displaySpecialEmployeeField(displaySpecial : bool) : void : Affiche les options afin d'ajouter les horaires d'un employé spécial.
                \item refreshVideoScreen() : Tâche de fond en charge de lire le flux vidéo entrant et de l'afficher sur l'écran permettant de regarder la vidéo.
                \item refreshCalendar(employeeID : EmployeeID) : void : Met à jour l'affichage du calendrier en fonction de l'ID de l'employé à afficher.
                \item refreshDoorState(doorState : bool) : Met à jour l'affichage de l'état de la Porte sur l'écran de contrôle à distance de la Porte en fonction du paramètre de type booléen.
                \item launchAOP() : void : Lance AOP.
                \item updateDoorState(doorState : bool) : void : Évènement pour demander à AOP de mettre à jour l'état de la Porte sur l'écran de contrôle à distance de la Porte.
                \item askScreen(screenID : ScreenID) : void : Démonstrateur demande à afficher l'écran entré en paramètre de la fonction.
                \item setIP(ip : IP) : void : Modifie l’attribut myIP avec le paramètre de la fonction.
                \item setPass(pass : Password) : void : Modifie l’attribut myPass avec le paramètre de la fonction.
                \item connect() : void : Demande la connexion.
                \item setConnectionStatus(status : ConnectionStatus) : void : Vérifie si la connexion est établie via le paramètre de la fonction status de type ConnectionStatus.
                \item askCalendar(employeeID : EmployeeID) : void : Demande d'afficher le calendrier correspondant au paramètre de la fonction de type EmployeeID.
                \item setEmployeeList(employeeListRemote : Employee[]) : void : Retour asynchrone permettant de mettre à jour la copie de la liste des employés de AOP.
                \item askOpenDoor() : void : Demande l'ouverture de la Porte.
                \item askAddEmployee() : void: Demande à ajouter un employé.
                \item setEmployeeName(name : String) : void : Modifie l’attribut myName avec le paramètre de la fonction.
                \item setEmployeeFirstName(firstName : String) : void :  Modifie l’attribut myFirstName avec le paramètre de la fonction.
                \item setEmployeePicture(picture : Picture) : void : Modifie l’attribut myPicture avec le paramètre de la fonction.
                \item setEmployeeRole(role : Role) : void : Modifie l’attribut myRole avec le paramètre de la fonction.
                \item setSpecialEmployeeAccess(day : Day, startHour : Hour, stopHour : Hour) : void : Entre les horaires de l'employé spécial.
                Avec comme paramètres day, startHour et stopHour qui prend en compte le jour, le début et de la fin de la journée de l'employé.
                Ces paramètres permettent la mise à jour de myWorkingHours.
                \item askDeleteEmployee(employeeID : EmployeeID) : void : Demande la suppression d'un employé. L'employé est déterminé par le paramètre de la fonction.
                \item confirm() : void : Confirme les modifications faites.
                \item cancel() : void : Annule les modifications effectuées.
                \item return() : void : Retourne à l'écran précédent.
                \item quitAOP() : void : Quitte AOP.
            \end{itemize}    
            
        \subparagraph{Description comportementale}%2.3.3.2.3
        Les MAE suivantes permettent d'expliquer le fonctionnement de l'objet actif GUI :
        \begin{figure} [H]
            \centering
            \includegraphics[scale=.5,max width=\textwidth,max height=.9\textheight]{architecture_candidate_anime_uml-gui-SM.eps}
            \caption{Machine à état représentant GUI}
            \label{MaE_connexion_GUI}
        \end{figure}

        \begin{figure} [H]
            \centering
            \includegraphics[scale=.5,max width=\textwidth,max height=.9\textheight]{architecture_candidate_anime_uml-gui-gui.Active.S_CONNECTED-SM.eps}
            \caption{Machine à état représentant GUI}
            \label{MaE_home_GUI}
        \end{figure}

        \begin{figure} [H]
            \centering
            \includegraphics[scale=.5,max width=\textwidth,max height=.9\textheight]{architecture_candidate_anime_uml-gui-gui.Active.S_CONNECTED.S_EMPLOYEE_LIST-SM.eps}
            \caption{Machine à état représentant employé liste}
            \label{MaE_home_GUI_employee_list}
        \end{figure}

\newpage

    \paragraph{[Object] EmployeeManager}%2.3.3.3
    \begin{figure} [H]
        \centering
        \includegraphics[scale=.5,max width=\textwidth,max height=.9\textheight]{architecture_candidate_anime_uml-class-EmployeeManager.eps}
        \caption{Diagramme de classe représentant EmployeeManager}
        \label{Classe_EmployeeManager}
    \end{figure}
        \subparagraph{Philosophie de conception}%2.3.3.3.1
        La classe EmployeeManager répertorie la liste des employés enregistrés dans l'application.
        Cet objet gère l'ajout et la suppression d'employés et interagit avec GUI, Bouncer. 
        \subparagraph{Description structurelle}%2.3.3.3.2
            \subsubparagraph{Attributs}%2.3.3.3.2.1
            \begin{itemize}
                \item {employeeList : Employee[] : Contient la liste des employés. 
                Il s'agit d'une copie des données persistantes chargée en mémoire.}
            \end{itemize}
            \subsubparagraph{Services offerts}%2.3.3.3.2.2
            \begin{itemize}
                \item {addEmployee(name : String, firstName : String, picture : Picture, role : String, workingHours : Hour[]) : void : Ajoute un employé à la liste à partir de son nom, prénom, sa photo, son rôle et les horaires de travail qui lui sont liés.}
                \item {deleteEmployee(employeeID : EmployeeID) : void : Supprime l'employé lié à l'identifiant employeeID passé en paramètre.}
                \item {getEmployeeList() : Employee[] : Renvoie la liste des employés à SoftSonnette.}
                \item {getEmployee(employeeID : EmployeeID) : Employee : Retourne l'objet employé lié à l'identifiant employeeID passé en paramètre.}
                \item {askEmployeeList() : Employee[] : Renvoie la liste des employés à AOP.}
                \item {load() : void : Charge les données persistantes des informations employés en mémoire vive.}
                \item {save() : void : Sauvegarde les données persistantes en mémoire non volatile.}
            \end{itemize}

\newpage

    \paragraph{[Object] Clock}%2.3.3.4
    \begin{figure} [H]
        \centering
        \includegraphics[scale=.5,max width=\textwidth,max height=.9\textheight]{architecture_candidate_anime_uml-class-Clock.eps}
        \caption{Diagramme de classe représentant Clock}
        \label{Classe_Clock}
    \end{figure}
        \subparagraph{Philosophie de conception}%2.3.3.4.1
        La classe Clock est en charge du temps.
        Elle synchronise les heures entre AOP et SoftSonnette.
        Elle est aussi utilisée pour savoir si un employé peut entrer en fonction de ses horaires attribués. 
        \subparagraph{Description structurelle}%2.3.3.4.2
            \subsubparagraph{Attributs}%2.3.3.4.2.1
            \begin{itemize}
                \item {currentTime : Time : Contient le temps courant.}
            \end{itemize}
            \subsubparagraph{Services offerts}%2.3.3.4.2.2
            \begin{itemize}
                \item {getCurrentTime(): Time : Permet de donner l'heure courante pour la comparaison des heures dans Bouncer.}
                \item {setCurrentTime(time : Time): void : Permet de modifier l'heure courante de la Board.}
            \end{itemize}
\newpage

    \paragraph{[Object] Cameraman}%2.3.3.5
    \begin{figure} [H]
        \centering
        \includegraphics[scale=.5,max width=\textwidth,max height=.9\textheight]{architecture_candidate_anime_uml-class-Cameraman.eps}
        \caption{Diagramme de classe de Cameraman}
        \label{Classe-Cameraman}
    \end{figure}
        \subparagraph{Philosophie de conception}%2.3.3.5.1
        La classe Cameraman interface E\_Caméra.
        Il permet de prendre une image du Testeur (pour qu’il se fasse reconnaître) et de capturer le flux vidéo retransmis sur l’écran de SoftSonnette et sur l’AOP.
        \subparagraph{Description structurelle}%2.3.3.5.2
            \subsubparagraph{Attributs}%2.3.3.5.2.1
            \begin{itemize}
                \item {cameraAlive : bool : Indique si la caméra est connectée ou non.}
                \item {alreadyAlive : bool : Indique si un flux vidéo est déja en cours d'utilisation}
            \end{itemize}
            \subsubparagraph{Services offerts}%2.3.3.5.2.2
            \begin{itemize}
                \item {streamToAOPandScreen() : void : Tâche de fond permettant de streamer la vidéo sur AOP et sur l'écran de SoftSonnette.}
                \item {streamToScreenOnly() : void : Tâche de fond permettant de streamer la vidéo sur l'écran de SoftSonnette.}
                \item {checkCameraConnected() : void : Vérifie la présence de la caméra.}
                \item {takePicture() : Picture : Fonction passive qui prend une photo de ce que voit Cameraman. 
                Cette photo est utilisée pour reconnaître le visage d’un Testeur.}
                \item {subscribeToVideoStream(enable : bool) : void : Fonction appelée par AOP pour demander à Cameraman de lui Streamer la vidéo.}
                \item {startStreaming() : void : Permet de démarrer le streaming.}
                \item {stopStreaming() : void : Stop définitivement le streaming - Sortie de la MàE.}
                \item {suspendStreaming() : void : Permet de mettre en pause le streaming.}
                \item {resumeStreaming() : void : Permet reprendre le streaming.}
            \end{itemize}
\newpage
        \subparagraph{Description comportementale}%2.3.3.3.3
            La MAE suivante permet d'expliquer le fonctionnement de l'objet actif Cameraman:
            \begin{figure} [H]
                \centering
                \includegraphics[scale=.5,max width=\textwidth,max height=.9\textheight]{architecture_candidate_anime_uml-cameraman-SM.eps}
                \caption{Machine à États de Cameraman}
                \label{MAE-Cameraman}
            \end{figure}

\newpage

    \paragraph{[Object] Guard}%2.3.3.6
        \begin{figure} [H]
            \centering
            \includegraphics[scale=.5,max width=\textwidth,max height=.9\textheight]{architecture_candidate_anime_uml-class-Guard.eps}
            \caption{Diagramme de classe de Guard}
            \label{Classe-Guard}
        \end{figure}
        \subparagraph{Philosophie de conception}%2.3.3.6.1
        La classe Guard est le gardien du mot de passe permettant à \appliA~d'interagir avec \appliLin.
        Il vérifie que le mot de passe reçu est celui codé en dur dans l'application puis autorise la connexion si ce dernier est correct.
        \subparagraph{Description structurelle}%2.3.3.6.2
            \subsubparagraph{Attributs}%2.3.3.6.2.1
            N.A.
            \subsubparagraph{Services offerts}%2.3.3.6.2.2
            \begin{itemize}
                \item{askCheckPass(pass : Password) : void : Permet de demander à Guard de vérifier le mot de passe passé en paramètre.}
            \end{itemize}

\newpage

    \paragraph{[Object] Bouncer}%2.3.3.7
    \begin{figure} [H]
        \centering
        \includegraphics[scale=.5,max width=\textwidth,max height=.9\textheight]{architecture_candidate_anime_uml-class-Bouncer.eps}
        \caption{Diagramme de classe représentant Bouncer}
        \label{Classe_Bouncer}
    \end{figure}
        \subparagraph{Philosophie de conception}%2.3.3.7.1
        La classe Bouncer permet au SàE d'autoriser ou non un Testeur à entrer.
        En interaction avec Cameraman elle déclenche la prise d'une photo du Testeur se présentant à la sonnette.
        Elle interagit également avec EmployeeManager pour récupérer la liste des employés.
        Elle déclenche la reconnaissance faciale avec la classe RecognitionAI à qui elle envoie toutes les données nécessaires.
        Elle vérifie qu'un employé reconnu ait l'autorisation d'entrer sur l'horaire courant qu'elle récupère auprès de la classe Clock.
        Elle demande enfin l'ouverture de la Porte à DoorManager ou signale à UISP que le Testeur n'est pas reconnu ou non autorisé.
        \subparagraph{Description structurelle}%2.3.3.7.2
            \subsubparagraph{Attributs}%2.3.3.7.2.1
            \begin{itemize}
                \item{picture : Picture : Photo du Testeur à reconnaître.}
                \item{employeeList : Employee[] : Liste courante des employés.}
                \item{recognizedEmployee : EmployeeID : ID de l'employé reconnu par RecognitionAI.}
                \item{currentTime : Time : Temps utilisé pour vérifier que l'employé est autorisé à entrer.}
            \end{itemize}
            \subsubparagraph{Services offerts}%2.3.3.7.2.2
            \begin{itemize}
                \item {checkEmployeeAllow(time : Time, employeeID : EmployeeID) : bool : À partir du temps courant et de l'ID de l'employé, elle vérifie si l'employé est autorisé à entrer à l'horaire courant.}
                \item {setRecognizeFace(employeeID : EmployeeID) : void : Appelée par RecognitionAI pour retourne l'employé reconnu par l'IA, ou UNKNOWN s'il n'y a pas de correspondance.}
                \item {askFaceRecognition() : void : Trigger permettant de déclencher le processus reconnaissance facial et d'autorisation d'entrée.}
            \end{itemize}
\newpage
        \subparagraph{Description comportementale}%2.3.3.7.3
        La MAE suivante permet d'expliquer le fonctionnement de l'objet actif Bouncer:
        \begin{figure} [H]
            \centering
            \includegraphics[scale=.5,max width=\textwidth,max height=.9\textheight]{architecture_candidate_anime_uml-bouncer-SM.eps}
            \caption{Machine à États de Bouncer}
            \label{MAE-Bouncer}
        \end{figure}
    
\newpage

    \paragraph{[IHM] UISS}%2.3.3.8
    \begin{figure} [H]
        \centering
        \includegraphics[scale=.5,max width=\textwidth,max height=.9\textheight]{architecture_candidate_anime_uml-class-UISS.eps}
        \caption{Diagramme de classe d'UISS}
        \label{Classe-UISS}
    \end{figure}
        \subparagraph{Philosophie de conception}%2.3.3.8.1
        La classe UISS s'occupe de la gestion de l'interface utilisateur entre le \actT~, le \actD~et \appliLin.
        Elle permet l'affichage de la vidéo, l'état de la Porte mais aussi au \actT~de sonner et au \actD~de quitter l'application.
        \subparagraph{Description structurelle}%2.3.3.8.2
            \subsubparagraph{Attributs}%2.3.3.8.2.1
            \begin{itemize}
                 \item {appState : SSState : Variable de mise a jour de l'écran qui prend trois valeurs possibles.}
            \end{itemize}
            \subsubparagraph{Services offerts}%2.3.3.8.2.2
            \begin{itemize}
                \item {display(state : SSState) : void : Affiche l'écran de \appliLin~ avec la variante transmise par le paramètre state.}
                \item {refreshDoorState(doorState : bool) : Met à jour l'affichage de l'état de la Porte sur l'écran de SoftSonnette en fonction du paramètre de type booléen passé.}
                \item {launchSS() : void : Démarre l'application \appliLin.}
                \item {quitSS() : void : Ferme l'application \appliLin.}
                \item {askStartCom() : void : Évènement reçu de la part de SoftPorte qui demande l'établissement de la communication.}
                \item {ring() : void : \actT~sonne.}
                \item {updateCamState(bool) : void : Évènement en provenance de Cameraman indiquant à UISS l'état de la caméra en fonction du paramètre de type booléen passé.
                UISS met ensuite à jour l'écran avec un appel à la fonction display().}
                \item {updateDoorState(doorState : bool) : void : Évènement en provenance de DoorManager indiquant à UISS l'état de la Porte en fonction du paramètre de type booléen passé.
                UISS met ensuite à jour le label indiquant l'état de la Porte sur l'écran avec un appel à la fonction refreshDoorState().}
            \end{itemize}
\newpage
        \subparagraph{Description comportementale}%2.3.3.8.3
        La MAE suivante permet d'expliquer le fonctionnement de l'objet actif UISS.
        \begin{figure} [H]
            \centering
            \includegraphics[scale=.5,max width=\textwidth,max height=.9\textheight]{architecture_candidate_anime_uml-uiss-SM.eps}
            \caption{Machine à États d'UISS}
            \label{MAE-UISS}
        \end{figure}

\newpage

    \paragraph{[Object] DoorManager}%2.3.3.9
    \begin{figure} [H]
        \centering
        \includegraphics[scale=.5,max width=\textwidth,max height=.9\textheight]{architecture_candidate_anime_uml-class-DoorManager.eps}
        \caption{Diagramme de classe représentant DoorManager}
        \label{DoorManager}
    \end{figure}
        \subparagraph{Philosophie de conception}%2.3.3.9.1
        La classe DoorManager est en charge du contrôle de la Porte simulée.
        Elle permet de commander son déverrouillage, et de fournir l'état de cette dernière.
    
        \subparagraph{Description structurelle}%2.3.3.9.2
            \subsubparagraph{Attributs}%2.3.3.9.2.1
            \begin{itemize}
                \item {doorState : bool : Indique l'état de la Porte (ouvert/fermé).}
            \end{itemize}
            \subsubparagraph{Services offerts}%2.3.3.9.2.2
            \begin{itemize}
                \item {unlockDoor() : void : Déverrouille la Porte pendant \hyperlink{top}{\textit{TOP}}.}
                \item {askOpenDoor() : void : Envoie la demande d'ouverture de la Porte vers UISP.}
                \item {askDoorState() : void : Demande l'état de la porte à UISP (retour asynchrone).}
            \end{itemize}
        \subparagraph{Description comportementale}%2.3.3.9.3
        \begin{figure} [H]
            \centering
            \includegraphics[scale=.5,max width=\textwidth,max height=.9\textheight]{architecture_candidate_anime_uml-doorManager-SM.eps}
            \caption{Machine à état représentant DoorManager}
            \label{MaE_DoorManager}
        \end{figure}

\newpage

    \paragraph{[IHM] UISP}%2.3.3.10
    \begin{figure} [H]
        \centering
        \includegraphics[scale=.5,max width=\textwidth,max height=.9\textheight]{architecture_candidate_anime_uml-class-UISP.eps}
        \caption{Diagramme de classe d'UISP}
        \label{UISP}
    \end{figure}
        \subparagraph{Philosophie de conception}%2.3.3.10.1
        La classe UISP permet de mettre à jour les différents éléments physiques de la Board.
        Elle met en place la communication entre SoftPorte et SoftSonnette, signale son bon démarrage en agissant sur la LED LD5 verte et signale lorsqu'un Testeur n'est pas autorisé à entrer avec la LED LD6 rouge.
        \subparagraph{Description structurelle}%2.3.3.10.2
            \subsubparagraph{Attributs}%2.3.3.10.2.1
            N.A.
            \subsubparagraph{Services offerts}%2.3.3.10.2.2
            \begin{itemize}
                \item {signalAppState() : void : Signale le lancement de l'application en allumant la LED verte.}
                \item {launchSP() : void : Lance SoftPorte.}
                \item {quitSP() : void : Quitte SoftPorte.}
                \item {signalNotAllowed(): void : Signale l'interdiction d'entrer en allumant la LED rouge.}
                \item {ackStartCom() : void : Confirme l'initialisation de la communication entre SoftPorte et SoftSonnette. 
                Cette fonction appel ensuite signalAppState() si la communication est établie avec succès.}
            \end{itemize}

\newpage

    \paragraph{[Object] RecognitionAI}%2.3.3.11
        \begin{figure} [H]
            \centering
            \includegraphics[scale=.5,max width=\textwidth,max height=.9\textheight]{architecture_candidate_anime_uml-class-RecognitionAI.eps}
            \caption{Diagramme de classe de RecognitionAI}
            \label{Classe-RecognitionAI}
        \end{figure}
            \subparagraph{Philosophie de conception}%2.3.3.11.1
            La classe RecognitionAI interface la librairie client de reconnaissance faciale.
            En lui fournissant la photo d’un Testeur, celle-ci détermine si le Testeur en question est autorisé à entrer.
            Cet objet est appelé par Bouncer lorsqu’une demande d’ouverture est faite.
            Une fois la reconnaissance faciale terminée, RecognitionAI envoie à Bouncer le résultat.
            \subparagraph{Description structurelle}%2.3.3.11.2
                \subsubparagraph{Attributs}%2.3.3.11.2.1
                N.A.
                \subsubparagraph{Services offerts}%2.3.3.11.2.2
                \begin{itemize}
                    \item{launch(picture : Picture, employeeList : Employee[]) : void : Lance le processus de reconnaissance faciale.}
                \end{itemize}
