\newpage
\subsection{Gestion du multitâche}
\subsubsection{Identification des accès concurrents dans \appliLin}%3.4.1

Dans un soucis de lisibilité, le nom des classes est abrégé dans le tableau \ref{tableau-acces-concurrents-1} répertoriant les accès concourants.
Les acronymes utilisés sont répertoriés dans le tableau ci-dessous:

\begin{table}[H]
    \centering
    \begin{tabularx}{\textwidth}{|X|p{2cm}|X|p{2cm}|}
      \hline 
         \textbf{Tâche}  &  \textbf{Abrégé} & \textbf{Autre} & \textbf{Abrégé} \\ 
        \hline
        RecognitionAI  & RAI & Clock &  Cl \\ 
        \hline
        Bouncer  & Bo & EmployeeManager &  EM \\ 
        \hline
        Cameraman  & Ca & ProxyGUI &  PG \\ 
        \hline
        UISS  &  UISS & ProxyConnectionManager &  PCM \\ 
        \hline
        Starter  & Sta & ProtocolSP &  ProSP \\ 
        \hline
        PostmanAOP  & PAOP & Guard &  G \\ 
        \hline
        PostmanSP  & PSP & ProxyDoorManager &  PDM \\ 
        \hline
        DispatcherAOP  & DAOP & DataEmployee &  DE \\ 
        \hline
        DispatcherSP  & DSP & ProtocolSS &  PSS \\ 
        \hline
        Streamer & Str & ProxyUISP &  PUISP \\ 
        \hline

    \end{tabularx}
    \caption{Tableau des classes abrégées  dans \appliLin}
    \label{tableau-abreger-classe-1}
  \end{table}

\begin{table}[H]
    \centering
    \begin{tabularx}{\textwidth}{|X|X|X|X|X|X|X|X|X|X|X|}
      \hline 
            & Cl & EM & PG & PCM & ProSP & G & PDM & DE & PSS & PUISP \\ 
        \hline
            RAI  &   &   &   &   &   &   &   &   &   &      \\ 
        \hline
            Bo   & X & X & X  &   &   &   & X  & X &   & X     \\ 
        \hline
            Ca   &   &   &   &   &  &   &   &   &   &    \\ 
        \hline
            UISS &   &   &   &   &   &   &   &   &   &  X  \\ 
        \hline
            Sta  &   &   &   &   &   &   &   &   &   &    \\ 
        \hline                                  
            PAOP &   &   &   &   &   &   &   &   &   &      \\ 
        \hline
            PSP  &   &   &   &   &   &   &   &   &   &    \\ 
        \hline
            DAOP & X & X &  X & X &   & X & X & X  & X &    \\ 
        \hline
            DSP  &   &   & X &   & X  &   &   &   &   &    \\ 
        \hline
            Str  &   &   &  &   &   &   &   &   &   &    \\ 
        \hline
    \end{tabularx}
    \caption{Tableau des accès concurrents dans \appliLin}
    \label{tableau-acces-concurrents-1}
  \end{table}

  Les classes DispatcherAOP et Bouncer appellent concurremment Clock et EmployeeManager.
  Pour éviter tout soucis ces deux classes possèdent l'attribut "protected" indiquant aux développeur de mettre en place des stratégies de verrouillage des ressources critiques permettant ainsi d'éviter les accès concourants.
  

\subsubsection{Identification des accès concurrents dans \appliPo}%3.4.2

\begin{table}[H]
  \centering
  \begin{tabularx}{\textwidth}{|X|X|X|X|X|}
    \hline 
                        & ProxyGUI  & ProxyUISS & UISP  & DispatcherSS  \\ 
      \hline
          DoorManager   &     X     &     X     &       &               \\ 
      \hline
          PostmanSS     &           &           &   X   &       X       \\ 
      \hline
  \end{tabularx}
  \caption{Tableau des accès concurrents dans \appliPo}
  \label{tableau-acces-concurrents-2}
\end{table}

Il n'y a pas de problème de concurrence.

\newpage
\subsubsection{Identification des accès concurrents dans \appliA}%3.4.3

Dans un soucis de lisibilité, le nom des classes est abrégé dans le tableau \ref{tableau-acces-concurrents-3} répertoriant les accès concourants.
Les acronymes utilisés sont répertoriés dans le tableau ci-dessous:

\begin{table}[H]
    \centering
    \begin{tabularx}{\textwidth}{|X|p{2cm}|X|p{2cm}|}
      \hline 
         \textbf{Tâche}  &  \textbf{abréger} & \textbf{Autre} & \textbf{abréger} \\ 
        \hline
        Communication  & Com & Protocol &  Pr \\ 
        \hline
        PostmanVideo  & PV & ConnectionManager &  CM \\ 
        \hline
        PostmanSoftSonnette  & PS & Dispatcher &  Di \\ 
        \hline
        GUI  & GUI & CacheEmployeeManager  &  CEM \\ 
        \hline
        CacheCameraman  &  CCa & WeeklyCalendar  &  WC \\ 
        \hline
          &  & ProxyDoorManager   &  PDM \\ 
        \hline
          &  & ProxyClock   &  PC \\ 
        \hline
          &  & ProxyGuard   &  PG \\ 
        \hline 


    \end{tabularx}
    \caption{Tableau des classe abrégées dans \appliA}
    \label{tableau-abreger-classe-2}
  \end{table}

\begin{table}[H]
    \centering
    \begin{tabularx}{\textwidth}{|X|X|X|X|X|X|X|X|X|X|}
      \hline 
                 & Pr & CM & Di  & CEM  & WC & PDM & PC & PG \\ 
        \hline
            GUI  &    &  X &  X  &  X   &  X &  X  & X  & X  \\ 
        \hline
            Com  &  X &  X &  X  &  X   &  X &  X  & X  & X  \\ 
        \hline
            PV   &    &    &     &      &    &     &    &    \\ 
        \hline
            PS   &    &    &     &      &    &     &    &    \\ 
        \hline
            CCa  &    &    &     &      &    &     &    &    \\      
        \hline
    \end{tabularx}
    \caption{Tableau des accès concurrents dans \appliA}
    \label{tableau-acces-concurrents-3}
  \end{table}


  La tâche MainActivity, appartenant à GUI, est le cœur d'AOP. 
  Or, MainActivity n’est pas susceptible de manipuler des ressources critiques puisqu'elle délègue ses tâches aux objets concernés, il n’y a donc pas de problèmes de concurrence ici.
  
  La tâche Communication se trouve en haut de la hiérarchie, régissant les demandes, les appels à ses méthodes de lecture et d’écriture, et gère le socket de communication, comme son nom l’indique elle communique avec tous les objets.
    
  Les tâches PostmanVideo, PostmanSoftSonnette et CacheCameraman n’accèdent à aucun objet “non actif”, aucun problème de concurrence n’est à détecter ici.
  
