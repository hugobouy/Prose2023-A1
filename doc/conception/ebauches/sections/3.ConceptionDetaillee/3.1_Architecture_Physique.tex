\subsection{Architecture détaillée}

Les diagrammes ci-dessous représentent le SàE dans son intégralité. 
En plus des classes vues précédemment, ce diagramme présente la gestion de la communication ainsi que les objets frontières nécessaires au bon fonctionnement du logiciel.

Ces diagrammes, dans un souci de lisibilité, ne détaillent pas les méthodes ainsi que les attributs des classes.
Ces derniers sont détaillés dans les diagrammes de classe qui suivent.

\begin{figure} [H]
    \centering
    \includegraphics[scale=.5,max width=\textwidth,max height=.9\textheight]{architecture_candidate_AOP_plantuml.eps}
    \caption{Architecture candidate détaillée de AOP représentée par un diagramme de communication}
    \label{archiDetailleAOP}
\end{figure}

Il est a noté que le classe GUI de la conception générale a été divisée en plusieurs Fragment représentant les différents écrans.
Ces fragments sont regroupés dans le package gui.

AOP adopte une stratégie de communication en utilisant une classe Communication, Protocol, Postman et Dispatcher.
A l'inverse, SoftSonnette et SoftPorte adoptent une stratégie avec des Proxys, Postamn, Protcol et Dispatcher.
Des détails sur ces deux stratégies sont proposés dans la partie 3.3. "Protocole de communication".

\begin{figure} [H]
    \centering
    \includegraphics[scale=.5,max width=\textwidth,max height=.9\textheight]{architecture_candidate_softSonnette.eps}
    \caption{Architecture candidate détaillée de SoftSonnette représentée par un diagramme de communication}
    \label{archiDetailleSS}
\end{figure}

% Important : Par soucis de lisibilité, ne sont pas représentés sur ce diagramme les classes protocoles qui interfacent les objets Proxy et Dispatcher comme suit :

% \begin{figure} [H]
%     \centering
%     \includegraphics[scale=.5,max width=\textwidth,max height=.9\textheight]{architecture_candidate_softSonnette_Protocol}
%     \caption{Diagramme de classes présentant les interactions avec les objets ProtocolSP et ProtocolSS}
%     \label{archiSimplifieeProtocol}
% \end{figure}

\begin{figure} [H]
    \centering
    \includegraphics[scale=.5,max width=\textwidth,max height=.9\textheight]{architecture_candidate_SP}
    \caption{Architecture candidate détaillée de SoftPorte représentée par un diagramme de communication}
    \label{archiDetailleSoftPorte}
\end{figure}

Les objets Starter ne sont pas représentés dans les diagrammes de communication et sont détaillés dans la suite du document.
