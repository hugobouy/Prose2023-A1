\newpage
\subsection{Environnement}

\subsubsection{Architecture matérielle et logicielle}

Le diagramme de déploiement UML de la Figure \ref{ArchitectureMaterielle} représente l’architecture logicielle et matérielle du SàE.
Les conventions graphiques utilisées sont explicitées en Figure \ref{LegendeDiagrammeDeploiement}.
Ce diagramme de déploiement identifie les entités matérielles et/ou logicielles avec lesquelles le SàE doit interagir et permet ainsi de déterminer les principaux échanges qu’il entretient avec son environnement.\\

\begin{figure} [H]
    \centering
    \includegraphics[width=\textwidth]{architecture_materielle}
    \caption{Architecture matérielle et logicielle représentée par un diagramme de déploiement UML}
    \label{ArchitectureMaterielle}
\end{figure}

Le SàE se compose de trois briques logicielles :
\begin{itemize}
    \item AOP : Application Ouverture Porte (Android)
    \item SoftSonnette : Application C déployée sur le Microprocesseur sur Linux
    \item SoftPorte : Application C déployée sur le Microcontrôleur\\
\end{itemize}

En plus des briques logicielles, le SàE stocke les \hyperlink{caracEmploye}{\textit{informations testeurs}} dans l'entité \hyperlink{donneesTesteurs}{\textit{Données\_Testeurs}} .
%Inc
%Le choix de la méthode de stockage des données sera définie lors de la conception et sera implémenté lors de l'incrément 2.
%Les données textuelles sont stockées dans un fichier csv et les images dans un dossier picture.

\begin{figure} [H]
    \centering
    \includegraphics[scale=0.7]{legende_diagramme_deploiement}
    \caption{Légende d'un diagramme UML}
    \label{LegendeDiagrammeDeploiement}
\end{figure}

\newpage
Comme indiqué sur la figure \ref{ArchitectureMaterielle}, les logiciels du SàE sont déployés sur deux supports Hardware : \hyperlink{bd}{\textit{La Board}} et le Téléphone.
La Board est en interaction avec deux dispositifs matériels supplémentaires contrôlés par le SàE : E\_Ecran et E\_Caméra.  
Ces dispositifs sont fournis et imposés par le client, ce sont les entités externes au SàE.
Par convention, le nom de ces entités est préfixé par les caractères "E\_" (E pour Externe).

La Board est une STM32MP15 composée d'un Microcontrôleur ARM Cortex-M4 et d'un Microprocesseur ARM-based dual Cortex-A7 communiquant en VUART (Virtual UART).
La Board est fournie et imposée par le client.

Les composants externes au SàE ont les caractéristiques suivantes :
\begin{itemize}
    \item E\_Ecran : 4" TFT 480x800 pixels avec LED backlight et interface MIPI DSISM.
    \item E\_Camera : LifeCam HD-3000, USB 2.0. \\
\end{itemize}

Le client impose également certains composants logiciel pour la production des applications :
\begin{itemize}
    \item OpenSTLinux (OSTL) référé par \hyperlink{linux}{\textit{Linux}} dans la suite du document et dont la version est précisée dans le dictionnaire du domaine. Il s'agit de l'OS sur lequel est déployé l'application SoftSonnette.
    \item Cube FW Package version 1.6, ensemble de librairies à destination du Microcontrôleur utilisé par SoftPorte.
    \item CubeIDE version 1.12.0, IDE pour le développement de l'application SoftPorte. \\
\end{itemize}

Il est convenu avec le client que l'application AOP est développée sur Android Studio version 2022.1.1.
La Téléphone est un Samsung Galaxy A20e fonctionnant sous Android 9.0. 
L'application est prévue pour fonctionner en mode portrait.
L'écran du téléphone mesure 5.8" et a une résolution de 720 x 1560 pixels.

La communication entre le Microprocesseur et le Téléphone se fait par réseau Wifi.
La \hyperlink{bd}{\textit{Board}} est configurée pour créer son propre HotSpot Wifi. 
Le Téléphone vient ensuite s'y connecter. 
Les deux entités utilisent ensuite le protocole TCP/IP pour communiquer. 
L'envoi et la réception de la vidéo se fait en UDP.

\subsubsection{Les interfaces du système}

Ce chapitre décrit les entrées et sorties dites «logiques» et «physiques» du SàE. 
En effet, nous différencions dans cette étude deux grands types d’entrées/sorties :
\begin{itemize}
    \item Celles dites de haut niveau (dites aussi logiques) qui décrivent les évènements et données échangés entre l’utilisateur et les périphériques PSC. 
    Ces entrées et sorties portent sur les intentions des acteurs interagissant avec le SàE.
    \item Celles dites de bas-niveau (dites aussi physiques) qui sont les entrées/sorties réellement échangées entre le SàE et les périphériques PSC. 
    Les entrées/sorties physiques (ou bas niveau) sont décrites au chapitre 2.2.2.3 (page \pageref{2.2.2.3}).
\end{itemize}

\newpage

\paragraph{Les interfaces logiques}

La figure \ref{ContexteLogique} (page \pageref{ContexteLogique}) présente le contexte de PSC en faisant figurer les entrées/sorties dites de haut niveau (ou logiques). 
Elles sont regroupées en grandes familles. 
Pour représenter ce contexte logique, un diagramme de communication UML est utilisé. 
Nous retrouvons les périphériques PSC déjà présentés en Figure \ref{ArchitectureMaterielle} (page \pageref{ArchitectureMaterielle}).

\begin{figure} [H]
    \centering
    \includegraphics[width=\textwidth]{contexte_logique}
    \caption{Contexte logique du SàE, représenté par un diagramme de communication UML}
    \label{ContexteLogique}
\end{figure}

Dans ce diagramme de communication, seules les entrées/sorties logiques entre les acteurs et les périphériques de PSC sont présentées.

\paragraph{Les interfaces avec les acteurs}

Nous allons maintenant détailler ces entrées et sorties logiques.

\subparagraph{\textit{En provenance du Testeur vers E\_Ecran}} 
\noindent Voici les évènements logiques du Testeur vers E\_Ecran:

    \begin{itemize}
        \item sonner() : Le Testeur demande à SoftSonnette (dans le SàE) la permission de rentrer.
    \end{itemize}

\subparagraph{\textit{A destination du Testeur depuis E\_Ecran}} 
\noindent Voici l'évènement logique du E\_Ecran vers le Testeur:

    \begin{itemize}
        \item afficheEcranSoftSonnette() : affiche sur E\_Ecran les éléments de « \hyperlink{EcranWebcamConnec}{Écran\_SoftSonnette} ».
    \end{itemize}

\subparagraph{\textit{A destination du Démonstrateur depuis E\_Ecran}} 
\noindent Voici l'évènement logique du E\_Ecran vers le Démonstrateur:

    \begin{itemize}
        \item afficheEcranSoftSonnette() : affiche sur E\_Ecran les éléments de « \hyperlink{EcranWebcamConnec}{Écran\_SoftSonnette} ».
    \end{itemize}

\subparagraph{\textit{En provenance du Démonstrateur vers E\_Ecran}} 
\noindent Voici les évènements logique du Démonstrateur vers E\_Ecran:
    
        \begin{itemize}
            \item démarrerSoftSonnette() : Le Démonstrateur lance l'application SoftSonnette en exécutant le fichier binaire SoftSonnette.out sur Linux.
            \item quitterSoftSonnette() :  Le Démonstrateur quitte SoftSonnette.
        \end{itemize}

\subparagraph{\textit{En provenance du Testeur vers E\_Camera}} 
\noindent Voici l'évènement logique du Testeur vers E\_Camera:

    \begin{itemize}
        \item présenterVisage(photo) : Le Testeur se positionne devant E\_Camera qui capture une \hyperlink{photo}{\textit{photo}}.
    \end{itemize}

\subparagraph{\textit{A destination de la Porte depuis le SàE}} 
\noindent Voici l'évènement logique du SàE vers le Porte:

    \begin{itemize}
        \item déverrouiller() : Le SàE \hyperlink{deverrouiller}{\textit{déverrouille}} la Porte. 
    \end{itemize}

\newpage

\subparagraph{\textit{A destination du Testeur depuis le SàE}} 
\noindent Voici l'évènement logique du SàE vers le Testeur:
    
        \begin{itemize}
            \item signalerVisageNonReconnu() : SoftPorte signale au Testeur que son visage est non reconnu via l'IHM \hyperlink{EcranSoftPorte}{\textit{vue board}}. 
        \end{itemize}

\subparagraph{\textit{En provenance du Démonstrateur vers le SàE}} 
\noindent Voici les évènements logiques du Démonstrateur vers le SàE:

    \begin{itemize}
        \item démarrerAOP() : Le Démonstrateur lance AOP depuis le Téléphone.
        \item entrerInfoConnexion(\hyperlink{IP}{\textit{IP}}, \hyperlink{mdp}{\textit{mdp}}) : Le Démonstrateur entre les informations de connexion.
        Cette fonction prend en paramètre l'adresse IP et le mot de passe.
        \item seConnecter() : Le Démonstrateur valide les informations de connexion.
        \item demanderAjoutEmployé() : Le Démonstrateur demande l'ouverture du menu pour ajouter un \hyperlink{emp}{\textit{employé}}.
        \item importerImage(\hyperlink{photo}{\textit{photo}}) : Le Démonstrateur importe l'image du nouvel employé à ajouter.
        Cette fonction prend en paramètre photo.        
        \item entrerInfosEmployé(caractéristiques\_employé) : Le Démonstrateur entre les informations relatives au nouvel employé.
        Cette fonction prend en paramètre les caractéristiques d'un employé (son \hyperlink{nom}{\textit{nom}}, son \hyperlink{prenom}{\textit{prénom}} et son \hyperlink{rol}{\textit{rôle}}).
        Dans le cas d'un employé spécial les horaires autorisés sont manuellement entrés par le Démonstrateur.
        \item demanderSuppressionEmployé(employéID) : Le Démonstrateur demande à supprimer un employé. 
        Cette fonction prend en paramètre l'identifiant d'un employé.
        \item quitterAOP() : Le Démonstrateur quitte AOP.
        \item return() : Le Démonstrateur demande à retourner à l'écran précédent.
        \item confirmer() : Le Démonstrateur confirme les informations entrées.
        \item annuler() : Le Démonstrateur demande à annuler ses modifications.
        \item demanderOuverturePorte() : Le Démonstrateur demande l'ouverture à distance de la Porte via AOP.
        
        \item L'évènement demanderEcran() est le regroupement des évènements suivants :
        \begin{itemize}
            \item[-] demanderCalendrier(employéID) : Le Démonstrateur demande à voir le calendrier d'un employé. 
            Cette fonction prend en paramètre l'identifiant d'un employé. 
            \item[-] demanderListeEmployés() : Le Démonstrateur demande à voir la liste des employés.
            \item[-] demanderVideo() : Le Démonstrateur demande à voir le \hyperlink{video}{\textit{flux vidéo}} de E\_Camera.
            \item[-] demanderContrôlePorte() : Le Démonstrateur demande à voir l'état de la Porte et le menu permettant d'en prendre le contrôle.
        \end{itemize}
    \end{itemize}

\newpage

\subparagraph{\textit{A destination du Démonstrateur depuis le SàE}}
Voici les évènements logiques du SàE vers le Démonstrateur:\newline
    \begin{itemize}
    \item L'évènement afficherEcran() est le regroupement des évènements suivants :
        \begin{itemize}
            \item[-] afficherCalendrier(employéID) : affiche l'écran « \hyperlink{EcranCalendrier}{Écran\_Calendrier} » avec la variante lié à employéID. 
            Cette fonction prend en paramètre l'identifiant d'un employé. 
            \item[-] afficheListeEmployés() : affiche l'écran « \hyperlink{EcranListe}{Écran\_Liste} ».
            \item[-] afficherVideo() : affiche l'écran « \hyperlink{EcranVideo}{Écran\_Vidéo} ».
            \item[-] afficherContrôlePorte() : affiche l'écran « \hyperlink{EcranPorte}{Écran\_Ouverture\_Porte} ».
            \item[-] afficherDémarrageAOP() : affiche l'écran « \hyperlink{EcranDemarrage}{Écran\_Démarrage} ».
            \item[-] afficherAccueil() : affiche l'écran « \hyperlink{EcranAccueil}{Écran\_Accueil} ».
            \item[-] afficherAjoutEmployé() : affiche l'écran « \hyperlink{EcranAjoutStandard}{Écran\_Ajout} ».
        \end{itemize}
    \item L'évènement afficherPopUp() est le regroupement des évènements suivants :
        \begin{itemize}
            \item[-] afficherPopUpAttenteConnexion() : affiche la Pop-Up « \hyperlink{popUpAttenteConnexion}{PopUP\_Attente\_Connexion}~».
            \item[-] afficherPopUpErreurConnexion() : affiche la Pop-Up « \hyperlink{popUpErreurConnexion}{PopUp\_Erreur\_Connexion}~».
            \item[-] afficherPopUpErreurMDP() : affiche la Pop-Up « \hyperlink{popUpErreurMDPAdmin}{PopUp\_Erreur\_MDP\_Admin}~».
            \item[-] afficherPopUpSuppression() : affiche la Pop-Up « \hyperlink{popUpSuppression}{PopUp\_Suppression}~».
            \item[-] afficherPopUpVideoIndisponible() : affiche la Pop-Up « \hyperlink{popUpVideoIndisponible}{PopUp\_VideoIndisponible}~».
        \end{itemize}
    \item signalerLancementSoftPorte() : SoftPorte signale au Démonstrateur son bon lancement via l'IHM \hyperlink{EcranSoftPorte}{\textit{vue board}}.  
    \end{itemize}

\newpage

\paragraph{Les interfaces physiques}
\label{2.2.2.3}

Ce paragraphe précise les caractéristiques de chaque interface entre le logiciel et les composants matériels du système. 
Il s’agit des entrées/sorties bas-niveaux (dites aussi physiques).
Ce sont celles que doit réellement traiter le SàE en les interprétant ou les générant en évènement logique. 
Cela comprend aussi les caractéristiques de configuration (nombre de ports, jeux d’instruction, etc.), les contraintes électriques... 

La suite de ce chapitre décrit chacune de ses interfaces. 
La figure \ref{ContextePhysique} représente ce contexte physique avec un diagramme de communication UML.

\begin{figure} [H]
    \centering
    \includegraphics[scale=0.75]{contexte_physique}
    \caption{Contexte physique du SàE, représenté par un diagramme de communication UML}
    \label{ContextePhysique}
\end{figure}

Le SàE interface les périphériques E\_Camera et E\_Ecran en utilisant des librairies standards fournis par \hyperlink{linux}{\textit{Linux}}.

Le client n'impose pas de dispositions pour interfacer les périphériques autres que les contraintes matérielles inhérentes.

Le détail des interactions entre le SàE et les périphériques E\_Camera et E\_Ecran est laissé à la discrétion de l'équipe projet et n'est donc pas documenté dans le présent dossier de spécification. Ces dernières sont définies en conception.

%Exploration 

\newpage

\paragraph{Les interfaces de communication}
N.A.

\subsubsection{Les contraintes de mémoire}
%Inc
%Lors de l'incrément 1, une base de données temporaire est créée sur le Téléphone.
%Lors de l'incrément 2 la base de données (BdD\_Testeurs) sera implémenté sur la Board.
Le stockage des \hyperlink{donneesPersistantes}{\textit{données persistantes}} est limité par les contraintes matérielles de stockage sur la \hyperlink{bd}{\textit{Board}}.
La principale contrainte réside dans le stockage des photos.
Afin d'assurer un fonctionnement normal du PSC, il est nécessaire de disposer de \hyperlink{maxEmployee}{\textit{MAX\_EMPLOYE}} * \hyperlink{poidsPhoto}{\textit{POIDS\_MAX\_PHOTO}} méga-octets de stockage disponible sur la \hyperlink{bd}{\textit{Board}}.

\subsubsection{Les activités}

\paragraph{Fréquences d'utilisation}
N.A.

\paragraph{Activité de maintenance}
N.A.

\subsubsection{Les exigences d'adaptation}
N.A.