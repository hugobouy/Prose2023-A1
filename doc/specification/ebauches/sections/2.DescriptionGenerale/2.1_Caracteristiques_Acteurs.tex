\newpage
\subsection{Caractéristiques des acteurs}

Par le terme d’acteur, nous désignons tout rôle joué par une entité (morale ou physique) qui interagit directement ou non avec le SàE. 
Cette entité peut être une personne (généralement un utilisateur du système) ou un autre système.

Nous distinguons les acteurs, dits directs (qui interagissent directement avec le SàE) et les acteurs dits indirects (qui n’ont pas d’interaction directe avec le SàE) mais qui sont à l’origine d’exigences à respecter par le SàE.

\subsubsection{Acteurs directs} 

Les acteurs directs sont : 
\begin{itemize}
    \item \textbf{Démonstrateur} : Le Démonstrateur est l'utilisateur qui interagit avec l'application "AOP". 
    Cet acteur peut visualiser l'état de la Porte, le flux vidéo de la caméra embarquée ainsi que la liste des Testeurs et leur calendrier associé.
    Il peut contrôler manuellement l'ouverture de la porte et ajouter ou supprimer des testeurs.
    \item \textbf{Testeur} : Le Testeur est un utilisateur qui interagit directement avec l'application "SoftSonnette".
    Le Testeur appuie sur le bouton de la sonnette, présente son visage à la caméra intégrée et déclenche ou non l'ouverture de la Porte si son visage est reconnu.
    Dans la suite du document, l'acteur Testeur pourra être présenté comme employé, en particulier dans les IHM pour les besoins de la démonstration.
    \item \textbf{Porte} : Entité qui désigne la porte contrôlée par le SoftPorte.
    La Porte n'a pas d'existence physique, elle est simulé pour les besoins de la démonstration.
\end{itemize}

\subsubsection{Acteurs indirects}

N.A.