\newacronym[first=PSC]{psc}{PSC}{Prototype Sonnette Connectée}

\newacronym[first=Client]{client}{Client}{Société STMicroelectronics}

\newacronym[first=Cahier des charges (CdC)]{cdc}{CdC}{Cahier des charges fourni par le client}

\newacronym[first=Cas d'utilisation]{cu}{CU}{Cas d'utilisation}

\newacronym[first=Interface Homme Machine (IHM)]{ihm}{IHM}{Interface Homme Machine. Moyens permettant aux utilisateurs d'AOP d'interagir avec AOP}

\newacronym[first=Disponibilité]{disponibilité}{Disponibilité}{La disponibilité est l'aptitude d'un composant ou d'un système à être en état de marche à un instant donné}

\newacronym[first=Fiabilité]{fiabilité}{Fiabilité}{La fiabilité est l'aptitude d'un composant ou d'un système à fonctionner pendant un intervalle de temps}

\newacronym[first=Maintenabilité]{maintenabilité}{Maintenabilité}{La maintenabilité est l'aptitude d'un composant ou d'un système à être maintenu ou remis en état de fonctionnement}

\newacronym[first=Non Applicable (N.A)]{na}{N.A}{Non Applicable}

\newacronym[first=Object Management Group (OMG)]{omg}{OMG}{(Object Management Group) Association professionelle internationale définissant entre autres des normes dans le domaine informatique}

\newacronym[first=Système à l'Étude (SàE)]{sae}{SàE}{Système à l'Étude. Il s'agit de l'ensemble des logiciels AOP, SoftPorte et SoftSonnette}

\newacronym[first=Unified Modeling Language (UML)]{uml}{UML}{Notation graphique normalisée, définie par l'OMG et utilisée en génie logiciel}

\newacronym[first=Real Time Clock (RTC)]{rtc}{RTC}{Une RTC est une horloge temps-réel, c'est un module présent sur la Board}
