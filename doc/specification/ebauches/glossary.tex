\newglossaryentry{bd}
{
    name={Board},
    description={\hypertarget{bd}{Est} désigné par le terme Board la STM32MP15 formant l'ensemble Microcontrôleur + Microprocesseur.}
}

\newglossaryentry{deverrouiller}
{
    name={Déverrouiller la Porte},
    description={\hypertarget{deverrouiller}{} L'action de déverrouillage de la Porte est réalisée à titre indicatif. 
    L'ouverture de la Porte est simulée par la mise à jour du voyant indicant l'état de cette dernière sur les écrans d'AOP et de SoftSonnette.}
}

\newglossaryentry{emp}
{
    name={Employé},
    description={\hypertarget{emp}{Un} employé est équivalent à un Testeur dans le cadre du prototype PSC.
    Il s'agit d'une personne susceptible de se présenter devant le PSC pour entrer, et dont les informations peuvent être entrées dans l'application AOP.}
}

\newglossaryentry{idEmp}
{
    name={employeID},
    description={\hypertarget{idEmp}{} Correspond à l'identifiant de l'employé dans \hyperlink{donneesTesteurs}{\textit{Données\_Testeurs}}. 
    %Conc
    Il s'agit d'un entier compris entre 1 et MAX\_EMPLOYE}
}

\newglossaryentry{maxEmployee}
{
    name={MAX\_EMPLOYE},
    description={\hypertarget{maxEmployee}{} Régit le nombre maximal d'employés enregistrés dans \hyperlink{donneesTesteurs}{\textit{Données\_Testeurs}} supporté par le PSC. 
    MAX\_EMPLOYE vaut 10.}
}

\newglossaryentry{idc}
{
    name={Infomations de connexion},
    description={\hypertarget{idc}{Les} informations de connexion correspondent à l'adresse \hyperlink{IP}{\textit{IP}} de la Board et au \hyperlink{mdp}{\textit{mot de passe}} requis pour entrer dans AOP.}
}

\newglossaryentry{linux}
{
    name={Linux},
    description={\hypertarget{linux}{}Fait référence à l'OS OpenSTLinux (OSTL) version 4.1 (Kernel Linux : 5.15) déployé sur le Microprocesseur.}
}

\newglossaryentry{tac}
{
    name={TAC},
    description={\hypertarget{tac}{}Le Temps d'Attente de Connexion correspond à la durée maximale après laquelle une connexion est considérée comme non établie, entre une demande de connexion et la réponse associée. 
    %Conc
    Il est défini que TAC vaut 5 secondes.}
}

\newglossaryentry{fr}
{
    name={FR},
    description={\hypertarget{fr}{}La Fréquence de Rafraîchissement correspond à la fréquence d'affichage de chaque image lors de flux vidéo. 
    Il est défini que FR vaut 15 Hz.}
}

\newglossaryentry{tal}
{
    name={TAL},
    description={\hypertarget{tal}{}Le Temps d'Allumage LED rouge correspond à la durée pendant laquelle la LED LD6 s'allume pour informer que le Testeur est refusé. 
    %Conc
    Il est défini que TAL vaut 5 secondes.}
}

\newglossaryentry{top}
{
    name={TOP},
    description={\hypertarget{top}{}Le Temps d'Ouverture Porte correspond à la durée pendant laquelle la Porte s'ouvre. 
    %Conc
    Il est défini que TOP vaut 10 secondes.}
}

\newglossaryentry{tass}
{
    name={TASS},
    description={\hypertarget{tass}{}Le Temps d'Attente SoftSonnette correspond à la durée pendant laquelle SoftSonnette attend une réponse de SoftPorte lors de l'initialisation de ce dernier. 
    %Conc
    Il est défini que TASS vaut 1 seconde.}
}

\newglossaryentry{initComBoard}
{
    name={Initialiser la communication avec SoftSonnette},
    description={\hypertarget{initComBoard}{} Correspond au processus durant lequel SoftPorte configure la Board et met en place la communication entre les deux applications.}
}

\newglossaryentry{IP}
{
    name={IP},
    description={\hypertarget{IP}{L'adresse} IP permettant la connexion entre AOP et SoftSonnette. 
    Cette adresse est représentée sous la forme IPv4 "x.x.x.x" où x représente un entier allant de 0 à 255}
}

\newglossaryentry{mdp}
{
    name={Mot de passe},
    description={\hypertarget{mdp}{Le} mot de passe permettant au Démonstrateur de se connecter à AOP. 
    Celui-ci est composé de 4 chiffres.}
}

\newglossaryentry{hor}
{
    name = {Horaires},
    description={\hypertarget{hor}{}Les horaires d'accès pour déclencher l'ouverture de la Porte sont sous la forme d'un bloc horaire unique pour chaque journée. 
    Les horaires sont regroupés en créneaux de 30 min et le format d'affichage est le format 24h.
    Ces horaires se basent sur l'heure de la Board, qui est équipée d'une RTC.
    %Conc
    Il est convenu que sa synchronisation est effectuée lors de la connexion entre l'AOP et SoftSonnette.}
}

\newglossaryentry{rol}
{
    name = {Rôle},
    description={\hypertarget{rol}{}Le rôle d'un employé définit sa plage horaire d'accès.
    Il existe cinq rôles différents : 
    \begin{itemize}
        \item Employé matin : 3h-13h, du lundi au vendredi.
        \item Employé journée : 8h-20h, du lundi au vendredi.
        \item Employé soir : 13h-23h, du lundi au vendredi.
        \item Employé sécurité : Accès illimité (24/7).
        \item Employé spécial : Les horaires sont définis manuellement pour chaque jour lors de l'ajout de l'employé dans l'application. 
    \end{itemize}
    }
}

\newglossaryentry{install}
{
    name = {SàE correctement installé},
    description={\hypertarget{install}{}Pour une installation correcte il faut les éléments suivants :
    \begin{itemize}
        \item AOP installé sur le téléphone.
        \item Linux installé sur le Microprocesseur. 
        \item SoftSonnette installé sur Linux.
        \item SoftPorte déployé sur le Microcontrôleur.
    \end{itemize}
    }
}

\newglossaryentry{video}
{
    name = {Flux vidéo},
    description={\hypertarget{video}{}Le flux vidéo est généré par la webcam (E\_Caméra), qui peut filmer avec une résolution de 1280x720p.
    %Conc
    Une résolution plus faible est préférée pour l'envoi du flux, cette résolution est définie dans le dossier de conception.  
    Stopper l'affichage du flux vidéo correspond à arrêter son affichage sur l'Écran\_SoftSonnette et son envoie à AOP (si ce dernier le requiert pour l'Écran\_Video).
    Reprendre l'affichage du flux vidéo permet d'afficher de nouveau cette dernière sur Écran\_SoftSonnette mais ne redémarre pas son envoie à AOP, ce dernier doit être redemandé par le Démonstrateur.
    }
}

\newglossaryentry{photo}
{
    name = {Photo},
    description={\hypertarget{photo}{}%Conc
    Les photos sont des fichiers au format jpeg ou jpg d'un poids maximal de POIDS\_MAX\_PHOTO.
    }
}

\newglossaryentry{poidsPhoto}
{
    name = {POIDS\_MAX\_PHOTO},
    description={\hypertarget{poidsPhoto}{} Défini le poids maximal autorisé d'une photo en méga-octets.
    POIDS\_MAX\_PHOTO vaut 3.5 Mo.
    }
}

\newglossaryentry{Aop}
{
    name = {AOP (Application Ouverture Porte)},
    description={\hypertarget{Aop}{}AOP est l'application Android utilisée par le Démonstrateur comme interface utilisateur pour gérer les différentes fonctionnalités du PSC.
    }
}

\newglossaryentry{SoftSonnette}
{
    name = {SoftSonnette},
    description={\hypertarget{SoftSonnette}{}SoftSonnette est l'application C déployée sur l'OS de la Board tournant sur le Microprocesseur en charge des interactions avec le Testeur.
    }
}

\newglossaryentry{SoftPorte}
{
    name = {SoftPorte},
    description={\hypertarget{SoftPorte}{}SoftPorte est l'application C déployée sur le Microcontrôleur de la Board en charge du contrôle de la Porte.
    }
}

\newglossaryentry{nom}
{
    name = {Nom},
    description={\hypertarget{nom}{}Le nom d'un employé est une chaine de caractère comprenant entre 1 et 12 alphanumériques encodés en UTF-8.
    }
}

\newglossaryentry{prenom}
{
    name = {Prénom},
    description={\hypertarget{prenom}{}Le prénom d'un employé est une chaine de caractère comprenant entre 2 et 12 alphanumériques encodés en UTF-8.
    }
}

\newglossaryentry{caracEmploye}
{
    name={Informations testeurs},
    description={\hypertarget{caracEmploye}{}Chaque testeur est identifé par : un \hyperlink{nom}{\textit{nom}}, un \hyperlink{prenom}{\textit{prénom}}, un \hyperlink{rol}{\textit{rôle}}, une \hyperlink{photo}{\textit{photo}} et pour les employés spéciaux des \hyperlink{hor}{\textit{horaires}} d'accès. 
    La convention d'encodage des données textuelles est l'UTF-8.
    }
}

\newglossaryentry{donneesTesteurs}
{
    name = {Données\_Testeurs},
    description={\hypertarget{donneesTesteurs}{}Correspond à l'entité qui stock de manière persistantes les \hyperlink{caracEmploye}{\textit{informations Testeurs}} sur la Board.
    En fonction du type de donnée, le mode de stockage varie : 
    \begin{itemize}
        \item Les données textuelles sont stockées dans un fichier csv, cela comprend le \hyperlink{nom}{\textit{nom}}, le \hyperlink{prenom}{\textit{prénom}}, le \hyperlink{rol}{\textit{rôle}} et les \hyperlink{hor}{\textit{horaires}}.
        \item Les \hyperlink{photo}{\textit{photos}} sont stockées dans un dossier "picture".
    \end{itemize}
    }
}

\newglossaryentry{donneesPersistantes}
{
    name = {Données persistantes},
    description={\hypertarget{donneesPersistantes}{}Les données persistantes sont les données sauvegardées par le PSC d'une session à une autre. 
    Ces données sont sauvegardées dans l'entité \hyperlink{donneesTesteurs}{\textit{Données\_Testeurs}}.
    L'opération de chargement des données persistantes diffère en fonction du type de donnée :
    \begin{itemize}
        \item Les données textuelles sont chargées en mémoire vive lors du lancement de SoftSonnette.
        \item Les \hyperlink{photo}{\textit{photos}} demeurent stockées en mémoire non volatile.
    \end{itemize}
    Le processus de sauvegarde des données persistantes diffère en fonction du type de donnée :
    \begin{itemize}
        \item La copie en mémoire vive des données textuelles en mémoire non volatile est assurée en fin d'exécution de SoftSonnette.
        \item L'ajout ou la suppression de \hyperlink{photo}{\textit{photos}} se fait au fur et à mesure de l'exécution du PSC.
    \end{itemize}
    }
}

