%-----------------------------------
% Plan de test simplifié
%
% ProSE
%
% Auteur : Camille Constant
%
%-----------------------------------

\documentclass[a4paper,11pt,titlepage,french]{article}

\usepackage{lipsum} % Sert à avoir l'exemple de texte avec \lipsum

%----------------------------------------
%       MACROS (à changer)
%----------------------------------------
\author{ProSE~2024~A1}        % Auteur
\title{Plan de Test Système}     % Titre
\date{\normalsize\today}    % Date actuelle, automatique
\newcommand{\client}{STMicroelectronics}
\newcommand{\groupe}{Prose-2024-A1}

\newcommand{\amodifier}{{\it À compléter et/ou modifier\\}}

\newcommand{\equipe}{Projet Prose A1}
\newcommand{\projet}{PSC (Prototype Sonnette Connectée)}
\newcommand{\produit}{PSC (Prototype Sonnette Connectée)}

\newcommand{\rqt}{TROVALLET Romain}
\newcommand{\cdp}{BOUY Hugo}

\newcommand{\versionPdT}{2.1}
\newcommand{\versionProjet}{2.0}

\newcommand{\refSpec}{[V2\_SPEC\_A1]}
\newcommand{\verSpec}{2.0}
\newcommand{\refPAQL}{[PAQL\_A1]}
\newcommand{\verPAQL}{4.0}

\newcommand{\appliC}{SoftSonnette et SoftPorte}
\newcommand{\appliLin}{SoftSonnette}
\newcommand{\appliPo}{SoftPorte}
\newcommand{\appliA}{AOP}

\usepackage[
            imageFooterLeft      =  {latex.png},                        % Optionnel
            imageHeaderLeft      =  {eseo.png},                         % Optionnel
            subject              =  {{Subject of the document}},        % Optionnel
            keywords             =  {{Example, Template, LaTeX}},       % Optionnel
            language             =  {french},                      
            % encoding             =  {},                               % Defaut utf8
            ]{Parameter}

\usepackage[
            logoRight       =   {ST-logo.png},                          % Optionnel
            logoLeft        =   {eseo.png},                             % Optionnel
            %logoCenter      =   {latex.png},                          % Optionnel
            subject         =   {{Incrément 2}},                        % Optionnel
            projectName     =   {\projet},                          % Optionnel
            enterpriseName  =   {\client},                          % Optionnel
            responsableName =   {\rqt},                            % Optionnel
            etatDocument    =   {Version~finale},                       % Optionnel
            % revision        =   {},                                   % Optionnel
            version         =   {\versionPdT},                                  % Optionnel
            ]{FrontPage}


%----------------------------------------
%       MISE EN PAGE
%----------------------------------------



\graphicspath{{../schemas/}{../figures/}}

% Exemple d'arrière plan
\backgroundsetup{
    scale=1,
    opacity=0.1,
    angle=0,
    contents={%
            \includegraphics[width=\textwidth]{eseo.png}
        }%
}

\setlength{\parskip}{0.5em} % Espacement entre les paragraphes
\setlength{\belowcaptionskip}{-0.5em} % Espacement entre les images

%----------------------------------------
%       DOCUMENT
%----------------------------------------

\begin{document}

\maketitle

\BgThispage % Utiliser l'arrière plan
\vspace*{\fill}
\noindent
AVERTISSEMENT : \\
Le présent document est un document à but pédagogique.
Il a été réalisé sous la direction de Camille CONSTANT, en collaboration avec des enseignants et les étudiants de l'option SE, groupe A1 (Hugo BOUY, Bastien CASSAR, Paul CHIRON, Paul JURET, Laurent LETENNEUR, Mathis MOULIN, Romain TROVALLET) du groupe ESEO.
Ce document est la propriété de Camille CONSTANT du groupe ESEO. En dehors des activités pédagogiques de l'ESEO, ce document ne peut être diffusé ou recopié sans l'autorisation écrite de ses propriétaires.
\vspace*{\fill}
\clearpage

\newpage

% Auteur : Camille Constant

%\section{Table des versions}

\noindent
% Modifie l’espacement horizontal entre les colonnes
%\setlength{\tabcolsep}{5pt}
\begin{tabularx}{\linewidth}{|p{1.8cm}|X|p{2.2cm}|p{1.8cm}|p{1.5cm}|}
\hline
\textbf{Date} & \textbf{Actions} & \textbf{Auteur} & \textbf{Version} & \textbf{Révision}  \\
\hline
26/05/2023 & Corrections suite à l'AN & \rqt & 2.1 & 0\\
\hline
25/05/2023 & Mise à jour des fonctionnalités pour l'incrément 2 & \rqt & 2.0 & 0\\
\hline
22/05/2023 & Mise à jour pour l'incrément 2 & \rqt & 1.2 & 0\\
\hline
17/03/2023 & Correction mineur & \rqt & 1.1 & 0\\
\hline
17/03/2023 & Correction mineur et approbation & \cdp & 1.0 & 0\\
\hline
16/03/2023 & Relecture finale & \rqt & 0.7 & 0\\
\hline
16/03/2023 & Rédaction de différentes parties du document& \rqt & 0.6 & 0\\
\hline
14/03/2023 & Rédaction de l'environnement de test& \cdp & 0.5 & 0\\
\hline
14/03/2023 & Mise à jour des caractéristiques du projet& \cdp & 0.4 & 1\\
\hline
13/03/2023 & Rédaction du périmètre de test& \rqt & 0.4 & 0\\
\hline
06/03/2023 & Modifications pour mieux coller à la norme ISO 829-2008 & C. Constant & 0.3 & 1\\
\hline
29/03/2022 & Modification des critères d’acceptation et mention de la couverture fonctionnelle & C. Constant & 0.2 & 3\\
\hline
03/03/2022 & Mise à jour des logiciels utilisés & C. Constant & 0.2 & 2\\
\hline
22/02/2021 & Modification des critères d’acceptation, ajout des exemples de diagramme d’activité, ajout de la section Validation du document & C. Constant & 0.2 & 1\\
\hline
05/03/2019 & Création de la trame du Plan de Test. & C. Constant & 0.1 & 1\\
\hline
\end{tabularx}

\newpage

%------------------------------- 

\addtocontents{toc}{\protect\setlength{\parskip}{0pt}}
\tableofcontents
% alternative pour réduire l'espacement entre les entrées de la table des matières
% (la valeur numérique peut être adaptée au besoin) : 
%{\setlength{\baselineskip}{0.96\baselineskip}\tableofcontents\par}
\newpage

%-------------------------------

% Auteur : Camille Constant

\section{Introduction}
\label{sec:intro}

\subsection{Contexte}
\label{sec:intro:contexte}


\begin{table}[H]
    \centering
    \begin{tabularx}{\textwidth}{|>{\hsize=.5\hsize}X|>{\hsize=1.5\hsize}X|}
      \hline
    {\bf Produit à tester~:} & \projet~- version \versionProjet\\
    \hline
    {\bf Type de produit~:} & {\it Système embarqué commandé et supervisé par une application Android}\\
    \hline
    {\bf Commanditaire~:} & \client \\
    \hline
    {\bf Développeur~:} & \equipe \\
    \hline
    {\bf Testeur~:} & \equipe\\
    \hline
\end{tabularx}
\end{table}

\subsection{Objet}
\label{sec:intro:objet}

Ce document décrit l'activité de test système qui sera menée par \equipe~durant le projet \projet~dans le but de valider le produit \produit. 
Il est rédigé sous la responsabilité du Responsable Qualité-Test (RQT), sous la direction du Chef de Projet (CdP), conformément au Plan d’Assurance Qualité Logicielle (PAQL) élaboré sous la responsabilité du RQT (cf. section~\ref{sec:eqTest}, Équipe de test).\\

\subsection{Portée} 
\label{sec:intro:portee}

Sont concernés par ce document :
\begin{itemize}
  \item Les testeurs : afin que ceux-ci connaissent le périmètre des tests (ce qu'ils vont tester), l’environnement de test (comment les tests seront mis en {\oe}uvre) et le processus de test (comment s’y prendre et rendre compte des résultats lors de l’exécution des tests) ;
  \item Les développeurs : à titre informatif, afin que ceux-ci sachent comment va être validée leur production ; à titre indicatif afin qu'ils sachent, par la description de la gestion des anomalies, comment ils s'interfaceront avec l'équipe de test ;
  \item Le client : ce plan de test fait l'objet d'une contractualisation avec le client pour déterminer le périmètre des tests menés pour valider le produit livré et les niveaux d'acceptation de cette validation ;
  \item Les auditeurs : ce plan de test, ainsi que son implication, feront l'objet d'audits par la société Formato.
\end{itemize}

%
%\subsection{Copyright}
%\label{sec:intro:copyright}
%
%Cf. PAQL.

\subsection{Présentation du système} 
\label{sec:intro:scope}

%{\it Résumer ici quel est le système avec ses principales fonctionnalités et citer les spécifications.}\linebreak
Le projet dénommé \projet , consiste à mettre en place les logiciels AOP (application Android), SoftSonnette et SoftPorte (applications C).
L’objectif du projet est de concevoir un prototype pour une démonstration des capacités de la STM32MP15. 
Le prototype est capable d'autoriser ou non l'accès aux Testeurs, de simuler l'ouverture d'une porte en fonction de leurs horaires et de permettre au Démonstrateur de lire le flux vidéo sur l’application Android.\\
%Le prototype est capable de reconnaître le visage de testeurs à l’aide de reconnaissance faciale, de simuler l’ouverture de la porte en fonction de leurs horaires et de permettre au Démonstrateur de lire le flux vidéo sur l’application Android.\\


\subsection{Références}
\label{sec:intro:ref}

\subsubsection{Documents de référence}

\noindent\begin{tabularx}{\linewidth}{|p{3cm}|X|p{2cm}|p{2cm}|}
\hline
\textbf{Ref.} & \textbf{Nom et auteur} & \textbf{Version} & \textbf{Source} \\
\hline
\refSpec & Dossier de Spécifications - \equipe & \centering \verSpec~- 25/05/2023 & pdf sur le dépôt\\
\hline
% si client impliqué dans le PAQL
\refPAQL & Plan d'Assurance Qualité Logiciel - \rqt & \centering \verPAQL~- 15/05/2023 & pdf sur le dépôt\\
\hline
\end{tabularx}

\subsubsection{Documents de référence}
%doc applicatifs
\noindent\begin{tabularx}{\linewidth}{|p{3cm}|X|}
\hline
\textbf{Ref.} & \textbf{Nom} \\
\hline
[ISO-829-2008] & Documentation de test logiciel\\
\hline
\end{tabularx}


\subsection{Glossaire et abréviations}
\label{sec:intro:termes}

Ce sont les termes et abréviations nécessaires à la compréhension de l'activité de test (les termes techniques propres au projet seront indiqués dans le dossier de spécification).\\ 
%Prenez par défaut les définitions du CFTL (\href{https://www.cftl.fr/wp-content/uploads/2018/10/Glossaire-des-tests-logiciels-v3_2F-ISTQB-CFTL-1.pdf}{ici}).

\subsubsection{Abréviations}


\begin{table}[H]
    \centering
    \begin{tabularx}{\textwidth}{|>{\hsize=.5\hsize}X|>{\hsize=1.5\hsize}X|}
        \hline
        IHM & Interface Homme-Machine\\
        \hline
        \appliLin~& voir \refSpec\\
        \hline
        \appliPo~& voir \refSpec\\
        \hline
        \appliA~& voir \refSpec\\
        \hline
        \projet~& voir \refSpec\\
        \hline
        TI & Tests d'intégration\\
        \hline
        TV ~& Tests de validation\\
        \hline
        TU & Tests unitaires\\
        \hline
        PdT &   Plan de Test : Document qui a pour but de piloter l'activité de test  \\
        \hline
        RDP & Référentiel Document Projet : Dépôt de tous les artefacts numériques du projet. 
        Ce dépôt est mis à la disposition de l'équipe \equipe, ainsi qu'à l'équipe de consultant Formato.\\
        \hline
    \end{tabularx}
\end{table}

\newpage
\subsubsection{Glossaire}

\begin{table}[H]
    \centering
    \begin{tabularx}{\textwidth}{|>{\hsize=.5\hsize}X|>{\hsize=1.5\hsize}X|}
      \hline
    {\bf Campagne de test} & Activité qui consiste à dérouler un ensemble de jeux de test. 
    Un dossier de test est produit à l’issue d’une campagne.\\
    \hline
    {\bf Cas de test} & Déclinaison d’un test précisant les valeurs utilisées pour les variables du test ainsi que les résultats attendus.\\
    \hline
    {\bf Dossier de test} & Ensemble documentaire qui contient la description des scénarios et des cas de tests, ainsi que l’exécution des jeux de test. 
    Le dossier de test est le reflet d’une campagne de test. \\
    \hline
    {\bf Jeux de test} & Ensemble des scénarios et cas de tests permettant de tester un produit logiciel. 
    L’enchaînement des cas et scénarios de tests est relatif à une stratégie de test précisée dans le plan de test.\\
    \hline
    {\bf Plan de test} & Document décrivant le déroulement d’un jeu de test : stratégie de test, critères d’arrêt, planification.\\
    \hline
    {\bf Scénarios de test} & Ensemble de cas de tests cohérents permettant de traiter un objectif fonctionnel.\\
    \hline
    {\bf Test fonctionnel} & Test (vu de l’utilisateur) du bon fonctionnement d’un produit logiciel, d’une fonctionnalité ou d’une fonction de base. 
    Vérification par rapport aux spécifications.\\
    \hline
    {\bf Test de non régression} & Vérification qu’une nouvelle version du produit fonctionne sans dégradation (technique, fonctionnelle, performance) par rapport à la version précédente.\\
    \hline
    {\bf Test de validation} & Vérification que le produit est cohérent et complet par rapport aux spécifications fonctionnelles.\\
    \hline
    {\bf Test système} & Vérification que le système dans son ensemble est cohérent et complet par rapport aux spécifications fonctionnelles et techniques.\\
    \hline
    {\bf Test d'intégration} & Vérification des interfaces et des interactions entre les composants intégrés.\\
    \hline
    {\bf Test unitaire} & Vérification de composants logiciels individuels.\\
    \hline
  \end{tabularx}
\end{table}

\subsection{Conformité}
\label{sec:intro:conf}

Ce plan de test est conforme aux normes~:
\begin{itemize}
\item IEEE Std. 1012-1986
\item IEEE Std. 829-1983
\item IEEE Std. 1008-1987
\end{itemize}
\newpage

% Auteur : Camille Constant

\section{Périmètre de test}
\label{sec:perimetre}

Cette section a pour objet l’élaboration d’un tableau des fonctionnalités et/ou composants / traitements / données du système mentionnant pour chacun l’effort de test à mener.
Cet effort est fonction de la pondération des exigences, risques et criticité retenus.

\subsection{Caractéristiques du projet}
\label{sec:peri:caract}

Le produit \produit~permet au client \client~d'étudier la faisabilité d'un système utilisant la reconnaissance faciale afin de l'utiliser lors de démonstrations à destination de ses différents clients.
L’objectif est de montrer les capacités de la STM32MP15 et de prouver qu'une telle application est possible.\\
Il se décompose en 2 lots/incréments~:
\begin{itemize}
    \item [\textbf{-}] lot 1~: 
        \begin {itemize}
            \item [\textbf{§}] L'application \appliA~doit établir une connexion avec \appliLin, afficher le flux vidéo provenant de la webcam et pouvoir consulter le calendrier d'un employé.
            \item [\textbf{§}] L'application \appliLin~doit pouvoir autoriser ou refuser un Testeur, envoyer le flux vidéo à \appliA~et l'afficher sur son écran.
            \item [\textbf{§}] L'application \appliPo~doit allumer un voyant signalant son bon démarrage.
        \end{itemize}
    \item [\textbf{-}] En plus des fonctionnalités du lot 1, dans le lot 2~: 
        \begin {itemize}
            \item [\textbf{§}] L'application \appliA~doit gérer la liste des employés qu'elle synchronise avec l'application \appliLin, afficher l'état de la Porte et pouvoir se connecter à l'application \appliLin~de manière sécurisée avec un mot de passe.
            \item [\textbf{§}] L'application \appliLin~doit pouvoir gérer l'autorisation d'un Testeur en fonction de ses horaires, stocker les données persistantes des employés et communiquer avec \appliPo.
            \item [\textbf{§}] L'application \appliPo~doit communiquer avec \appliLin~et simuler l'ouverture de la Porte.\\
        \end{itemize}
\end{itemize}

\client~souhaite connaître la qualité globale de \produit~après chaque lot afin d'éventuellement redéfinir chacun des lots. 

%Ce plan de test concerne le niveau de test système pour l'incrément 1. 
Pour information, des tests d’intégration (lots 1 et 2) et des tests unitaires (lot 2) auront été réalisés par \equipe.



\subsection{Éléments à tester}
\label{sec:peri:elements} 


Cette partie s'appuie sur la section 2.2.1 Architecture matérielle et logicielle du document de spécifications \refSpec. 

\medskip


\pagebreak

{\bf Logiciels développés :}\\
\begin{itemize}
    \item [\textbf{-}] Applications C : \appliC
    \item [\textbf{-}] Application Java : \appliA\\
\end{itemize}


{\bf Support d'exécution :}\\
\begin{itemize}
    \item [\textbf{-}] \appliLin~s'exécutera sous OpenSTLinux (OSTL) version 4.1 sur la Board STM32MP15
    \item [\textbf{-}] \appliA~s'exécutera sous Android version 9.0 sur le smartphone Samsung Galaxy A20e\\
\end{itemize}


{\bf Support de communication :}\\
\begin{itemize}
    \item [\textbf{-}] Communication WiFi (Pile TCP/IP)
\end{itemize}


\subsubsection{Éléments concernés par les tests}
\label{sec:peri:comp:test}
          
Cette section désigne ce qui est testé (composant, logiciel, sous-système).\\
          
Seront concernés par l'activité de test les composants logiciels développés durant le projet \projet~: 
\begin{itemize}
    \item \appliLin
    \item \appliPo
    \item \appliA
    \item Communication entre \appliA~et \appliLin
\end{itemize}

\subsubsection{Éléments non concernés par les tests}
\label{sec:peri:comp:nontest}

Cette section désigne ce qui ne va pas être testé (composant, logiciel, sous-système).\\

Ne seront pas concernés par les tests :
\begin{itemize}
    \item Le support logiciel Cube FW Package étant une bibliothèque
    \item Les supports d'exécution logiciels (Linux, Android)
    \item Les supports d'exécution matériels (carte STM32MP15, Samsung Galaxy A20e, Webcam)
    \item Les supports de communication matériels (Borne WiFi, etc.)
    \item Les supports de communication logiciels (Pile TCP/IP)
\end{itemize}

\subsection{Spécifications fonctionnelles ou techniques à tester}
\label{sec:peri:spec}


Cette section s'appuie sur la section 2.3 Fonctions principales développées du document de spécifications \refSpec. 
%Toutes les fonctionnalités ne seront pas testées car ce document se limite à l'incrément 1.\\

\subsubsection{Fonctionnalités à tester}
\label{sec:peri:fct:test}

%{\it Tenir compte des incréments $\Rightarrow$ lors de l'incrément 1, valider en priorité les fonctionnalités basiques permettant de valider la communication dans les 2 sens (C vers A et A vers C).}\\

\medskip

Les fonctionnalités suivantes sont à tester~:
\noindent\begin{longtable}[c]{|p{.38\textwidth}|p{.1\textwidth}|p{.38\textwidth}|}
    \hline
        \bf Fonctionnalité & \bf \centering Priorité & \bf Commentaire\\[-1ex]
                            & \centering (P0 : priorité max) & \\
    \hline
    \endhead
    Initialisation de la Board & \centering P0 & RaS \\
    \hline
    Initialisation de la Board - Webcam non connectée & \centering P1 & Initialisation réussie sans Webcam \\
    \hline
    Établissement de la connexion entre \appliA~et \appliLin & \centering P0 & \appliA~vers \appliLin~et inversement \\
    \hline
    Échec de l'authentification lors de l'établissement de la connexion & \centering P0 & Mot de passe incorrect lors de l'entrée des informations de connexion \\
    \hline
    Échec de l'établissement la connexion entre \appliA~et \appliLin & \centering P1 & \appliA~et \appliLin~ne parvienne pas à communiquer \\
    \hline
    Autorisation du Testeur qui souhaite entrer pendant ses horaires & \centering P0 & Le Testeur est reconnu et y accède lors de ses horaires \\
    \hline
    Refus du Testeur inconnu qui souhaite entrer & \centering P0 & Le Testeur n'est pas reconnu \\
    \hline
    Refus du Testeur reconnu qui souhaite entrer & \centering P0 & Le Testeur est reconnu mais essaie d'entrer hors de ses horaires \\
    \hline
    Ouverture de la Porte et mise à jour de son état & \centering P0 & Demande d'ouverture en local \\
    \hline
    Transmission et affichage du flux vidéo de \appliLin~à \appliA & \centering P0 & RaS \\
    \hline
    Échec de la transmission du flux vidéo de \appliLin~à \appliA & \centering P0 & RaS \\
    \hline
    Consultation du calendrier employé & \centering P0 & L'affichage doit se faire suivant l'employé sélectionné \\
    \hline 
    Ouverture de la Porte à distance et mise à jour de son état & \centering P0 & Demande d'ouverture depuis l'application \\
    \hline
    Consultation de la liste des employés &\centering P0 & Seulement l'affichage des données persistantes, permet la vérification des modifications effectuées \\
    \hline
    Ajout d'un employé standard dans la liste des employés & \centering P0 & RaS \\
    \hline
    Ajout d'un employé spécial dans la liste des employés & \centering P0 & Sélection du rôle "spécial" et de ses horaires personnalisables \\
    \hline
    Suppression d'un employé dans la liste des employés & \centering P0 & RaS \\
    \hline
    Annulation d'une action dans la liste des employés & \centering P2 & RaS \\
    \hline           
    Fermeture \appliA~et \appliLin &\centering P0 & RaS \\
    \hline
\end{longtable}

% Pour plus de détails, voici la matrice de conformité de l'incrément 1 : 
% \href{
% https://docs.google.com/spreadsheets/d/1aaes5xD0STobJdoepx0t9ATB4pgT6jGYcr2hy5ju8Fo/edit#gid=0
% }{ici}

\noindent Pour plus de détails, voici la matrice de conformité du \projet: 
\href{https://docs.google.com/spreadsheets/d/1xq-QIusS_guV91z53ERgtteP4NOb_qmYkGOM_DKjgew/edit?usp=sharing
}{ici}


\subsubsection{Caractéristiques techniques à tester}
\label{sec:peri:tech:test}
%Peut être tout ce qui est lié à des demandes de responsiveness/robustesse
Le cahier des charges de \client~ne contient aucune caractéristique technique à tester.\\

\subsection{Spécifications fonctionnelles ou techniques non testées}
\label{sec:peri:nontest}

%{\it Fonctionnalités non critiques, non prioritaires ou cas d’exception dans CU. \\}

Les fonctionnalités suivantes sont considérées comme non testées car elles représentent une variante d'une fonctionnalité déjà testée au-dessus :

\noindent\begin{longtable}[c]{|p{.35\textwidth}|p{.15\textwidth}|p{.35\textwidth}|}
    \hline
        \bf Fonctionnalité & \bf \centering Priorité & \bf Commentaire\\[-1ex]
                            & (P0 : priorité max) & \\
    \hline
    \endhead
    \hline
    Échec de l'initialisation de la Board & \centering P2 & Cas où \appliPo~ne communique pas avec \appliLin \\
    \hline 
    Abandon de l'établissement de la connexion &\centering P3 & Démonstrateur quitte \appliA \\
    \hline
    Enchaînement de plusieurs Testeurs & \centering P2 & Cas où plusieurs Testeurs veulent rentrer \\
    \hline    
    \appliA~non utilisé &\centering P3 & - Ne pas envoyer de notification vers \appliA~lors de l'ouverture de la Porte. \\
    & & - Pas de fermeture d'\appliA~lors de la fonctionnalité "Fermeture d'\appliA~et \appliLin". \\
    \hline
\end{longtable}

L'ergonomie ainsi que la conformité de l'emplacement des éléments de l'IHM aux spécifications ne sera pas testée. 
L'IHM ne sera validée qu'au travers des tests fonctionnels. 

\subsection{Criticité}
\label{sec:peri:criticite}

Seul l'envoi de données entre \appliLin~et \appliA~est considéré comme critique.

\newpage

\subsection{Risques}
\label{sec:peri:risques}

\noindent Id~: identifiant du risque\\
Description~: description du risque\\
Effet~: effet du risque\\
P~: probabilité (3 – très probable, 2 – probable, 1 – peu probable)\\
I~: impact (3 – impact fort, 2 – impact moyen, 1 – impact faible)\\
EI (élément impacté)~: coût / qualité / délai\\
Action~: description de l’action pour maîtriser le risque

\subsubsection{Risques projet}
\label{sec:peri:risques:projet}

\noindent\begin{longtable}[c]{|p{1.4cm}|p{3cm}|p{3cm}|p{0.4cm}|p{0.4cm}|p{1.2cm}|p{3.5cm}|}
\hline
\bf Id & \bf Intitulé & \bf Effet & \bf \centering P & \bf \centering I & \bf \centering EI & \bf Action\\
\hline
\endhead
RPRJ1 & Pas de test unitaire & Instabilité de l’application lors des tests système & \centering 1 & \centering 2 & \centering C/Q/D & Faire une phase de smoke tests sur l’application avant de réaliser les tests système.\\
\hline
RPRJ2 & Pas de test d’intégration & Instabilité de l’application lors des tests système & \centering 1 & \centering 2 & \centering C/Q/D & Faire une phase de smoke tests sur l’application avant de réaliser les tests système.\\
\hline
RPRJ3 & Pas de test unitaire ou de test d’intégration & Instabilité de l’application lors de tests d’acceptation & \centering 1 & \centering 2 & \centering C/Q/D & Réaliser des tests système sur toutes les fonctionnalités système\\
\hline
RPRJ4 & Problème de disponibilité des intervenants & Dérive dans le planning des tests & \centering 1 & \centering 1 & \centering D & Planifier au plus tôt les actions des différents intervenants\\
\hline
RPRJ5 & Spécifications du produit \produit~non à jour & Déviations entre les spécifications et le système d’où une difficulté pour concevoir des tests pertinents & \centering 2 & \centering 2 & \centering Q & Analyse des spécifications pour identifier des écarts. Poser toutes les questions nécessaires à une bonne compréhension des spécifications.\\
\hline
\end{longtable}

\pagebreak
\subsubsection{Risques produit}
\label{sec:peri:risques:produit}

\noindent\begin{longtable}[c]{|p{1.4cm}|p{3cm}|p{3cm}|p{0.4cm}|p{0.4cm}|p{1.2cm}|p{3.5cm}|}
\hline
\bf Id & \bf Intitulé & \bf Effet & \bf \centering P & \bf \centering I & \bf \centering EI & \bf Action\\
\hline
\endhead
RPRD1 & Mauvaise implémentation de la communication & Système non fonctionnel & \centering 2 & \centering 3 & \centering C/Q/D & Tester la communication en priorité par des tests d'intégration (voir unitaires) avant de faire les tests système.\\
\hline
RPRD2 & Reconnaissance faciale échoue & Système non fonctionnel & \centering 3 & \centering 3 & \centering C/Q/D & Si le script IA ne fonctionne pas, l'équipe \equipe~peut demander l'API de reconnaissance faciale dont dispose \client .\\
\hline
\end{longtable}

\subsection{Effort de test}
\label{sec:peri:effort}

L’effort de test sera priorisé de la façon suivante~:
\begin{itemize}
    \item Phase de \og Smoke test \fg pour vérifier la stabilité de l’application avant de réaliser la campagne de tests fonctionnels système ;
    \item Campagne de tests fonctionnels système selon les priorités ;
    \item Campagne de tests non fonctionnels système par priorité.
\end{itemize}

\newpage

% Auteur : Camille Constant

\section{Processus et Stratégie de test}
\label{sec:process}

\subsection{Objectifs (actions de test)}
\label{sec:process:objectifs}


\noindent Exigence : Exigence concernée\\
Risque : Risque concerné (cf. section~\ref{sec:peri:risques})\\
Niveau : Niveau de test (S : Système, I : Intégration, U : Unitaire)\\
Technique : Technique de test (AP : Analyse Partitionnelle ou Classes d'équivalence, AL : Analyse aux limites, CU : Cas d'Utilisation, PC : Protocole de Communication)

\noindent\begin{longtable}[c]{|p{0.3cm}|p{2.8cm}|p{1.7cm}|p{1.2cm}|p{1.2cm}|p{1.8cm}|p{3.8cm}|}
\hline
\bf N° & \bf Énoncé de l'objectif & \bf Exigence & \bf Risque & \bf Niveau & \bf Technique & \bf Conditions de mesure / niveau d'atteinte prévu\\
\hline
\endhead
1 & Tester la communication & Fonct. P0 & RPRD1 & I, S, U & AP, PC, CU & Fonctionnalités P0 / 100\% des fonctionnalités testées en utilisant les classes d’équivalence, le protocole de communication et les cas d’utilisation\\
\hline
2 & Smoke test & Toutes & RPRJ1, RPRJ2, RPRJ4 & S, U & Test par expérience & Nombre d’anomalies bloquantes / pas d’anomalie bloquante\\
\hline
3 & Tester 100\% des fonctionnalités P0 & Fonct. P0 & RPRJ3, RPRD1 & S, U & AP, CU & Fonctionnalités P0 / 100\% des fonctionnalités testées en utilisant les classes d’équivalence et les cas d’utilisation\\
\hline
4 & Tester 100\% des fonctionnalités non P0 & Fonct. non P0 & RPRJ3, RPRD1 & S, U & AP, CU & Nombre de cas d’utilisation / tous les cas d’utilisation testés\\
\hline
\end{longtable}

\subsection{Organisation}
\label{sec:process:orga}

\subsubsection{Découpage en phase de tests / campagnes}
\label{sec:process:orga:decoupage}
Deux campagnes de tests système sont prévues dans le projet pour chaque lot/incrément~:
\begin{itemize}
    \item Campagne de tests système comprenant les tests fonctionnels pour atteindre les différents objectifs de tests ci-dessus ;
    \item Campagne de retest (vérification de la correction des anomalies détectées et de non-régression).
\end{itemize}

\subsubsection{Gestion des rapports d’anomalie}
\label{sec:process:orga:anomalies}

Les anomalies sont gérées dans Redmine sous forme de tâche.
Dès l’observation d’une défaillance dans le produit, un rapport d’anomalie est rédigé dans Redmine.

\subsection{Critères d’acceptation des tests}
\label{sec:process:accept}

Pour le passage en test de validation système, la phase de smoke test ne doit pas détecter d’anomalie bloquante.\\

Pour la mise en production, aucune anomalie bloquante ni majeure n’est acceptée.

\begin{table}[H]
    \centering
    \begin{tabularx}{\textwidth}{|>{\hsize=.5\hsize}X|>{\hsize=1.5\hsize}X|}
        \hline
        {\bf Anomalie bloquante} & La fonctionnalité n’est pas utilisable. \\
        \hline
        {\bf Anomalie majeure} & La fonctionnalité ne répond pas à ses exigences mais une solution de contournement existe pour utiliser la fonctionnalité, ou la fonctionnalité est utilisable en l’état (par exemple, anomalie dans une règle de calcul). \\
        \hline
        {\bf Anomalie mineure} & La fonctionnalité est utilisable mais pas de façon optimale (par exemple, problème d’ergonomie ou de charte graphique). \\
        \hline
    \end{tabularx}
\end{table}
  

\subsection{Critères d’arrêt}
\label{sec:process:arret}

Les tests d’une fonctionnalité s’arrêteront si une anomalie bloquante est découverte ne permettant pas de poursuivre les tests de cette fonctionnalité.

\subsection{Activités de test}
\label{sec:process:activites}  
  
L’activité de test sera faite par \equipe~tout au long du cycle de développement, via notamment :
\begin{itemize}
    \item des tests de validation sur le comportement nominal du système ;
    \item des tests d'intégration sur le comportement nominal de \appliC.
    \item des tests unitaires nominaux sur certaines classes de \appliA~ et sur certains modules de \appliC. 
    %Des tests unitaires aux limites seront menés sur les méthodes ou fonctions dont les données d'entrée proviennent de l'environnement. 
\end{itemize}

\subsubsection{Planification}
La planification des tests système est réalisée par \equipe.

\subsubsection{Conception}
La conception des tests système est réalisée par \equipe.
La conception des jeux de données de test est réalisée par \equipe.

\subsubsection{Exécution}
L’exécution des tests système est réalisée par \equipe.
L’exécution des tests d’acceptation est réalisée par \client.

\subsubsection{Bilan}
\equipe~rédige un bilan de test en fin de campagne de test système.

\subsection{Documents de test et livrables}
\label{sec:process:livrables}

\noindent\begin{longtable}[c]{|l|c|}
\hline
 & {\bf Livrable à transmettre} \\
\hline
{\bf Documentation} & \\
\hline
Plan de test & X \\
\hline
Dossier de test & X \\
\hline
Rapport de test & X \\
\hline
Matrice de conformité exigences et tests & X \\
\hline
%demander client
Rapport d’anomalies & \\
\hline
{\bf Données} & \\
\hline
Documents d'analyses partitionnelles et aux limites & \\
\hline
Jeux de données de tests & X\\
\hline
{\bf Automatisation des tests} & \\
\hline
%demander client
Code de test & \\
\hline
\end{longtable}


\newpage

% Auteur : Camille Constant

\section{Environnement de test}
\label{sec:env}

\subsection{Environnement matériel et logiciel de test}
\label{sec:env:env}


%voir avec Hugo mais pas de problèmes normalement

%{\it Matériel et logiciels nécessaires à l'exécution des tests. 

%Exemples : PC, OS, simulateur, réseau (quel wifi ?), BDD, robot composé d'une carte Armadeus sous Linux xx avec capteurs Lego x, y et actionneurs t, z, etc.}

Afin de réaliser les différents tests de validation système, d'intégration et unitaires, l'environnement matériel suivant est mis en place :

\begin{itemize}
    \item La STM32MP15 (Board) est alimentée et connectée en USB à un PC disposant d'un OS Linux permettant d'accéder à distance à cette dernière.
    \item La STM32MP15 déploie sur son Microprocesseur l'OS OpenSTLinux (OSTL, appelé Linux dans la suite du document) (version 4.1)
    \item Est installé sur le Linux de la Board le logiciel SoftSonnette.
    \item Est déployé sur le Microcontrôleur de la Board le logiciel SoftPorte et les libraries provenant de Cube FW Package (version 1.6).
    \item Est mis à disposition du Démonstrateur un smartphone Samsung Galaxy A20e avec Android (version 9.0).
    \item Est installé sur le smartphone l'application AOP.
\end{itemize}

\subsection{Outils de test}
\label{sec:env:outils}

Les tests seront au maximum automatisés grâce aux outils suivants~:
\begin{itemize}
    \item Tests unitaires : Framework de test Android (basé sur JUnit), bouchonnage Mockito, CMocka ;
    \item Tests d'intégration : JMeter ;
    \item Tests de validation : automatisation avec Robot Framework, sinon tests manuels ;
    \item Dossier de test : Squash TM avec intégration de la gestion d'anomalies via Redmine.
\end{itemize}




\newpage

% Auteur : Camille Constant

\section{Rôles et responsabilités}
\label{sec:resp}

\equipe~:
\begin{itemize}
    \item Gestionnaire de tests
    \begin{itemize}
        \item Rédaction du plan de tests
        \item Rédaction du bilan de tests
        \item Suivi de la réalisation des tests système
        \item Reporting auprès de \client
    \end{itemize}
    \item Analyste de tests
    \begin{itemize}
        \item Conception des tests
        \item Conception des jeux de données
        \item Exécution des tests
        \item Gestion des rapports d’anomalies (création et clôture)\\
    \end{itemize}
\end{itemize}

\client~:
\begin{itemize}
\item Validation des documents produits\\
\end{itemize}

\equipe~ne peut être tenu responsable des répercussions d'une défaillance d'une fonctionnalité non validée. 

\newpage

% Auteur : Camille Constant

\section{Équipe de test}
\label{sec:eqTest}



%Le CdP est responsable des moyens mis à disposition pour mener à bien l'activité de test. 
%\medskip

%Le RQT est responsable de l'organisation et du déroulement de l'activité de test. 
%\medskip

%Les testeurs sont responsables des résultats de test reportés dans le cahier de test. 


\noindent\begin{longtable}[c]{|p{.30\textwidth}|p{.20\textwidth}|p{.40\textwidth}|}
    \hline
        \bf NOM Prénom & \bf \centering Rôle & \bf Implication\\
        \hline
    \endhead
    BOUY Hugo &\centering Chef de Projet & Se porte garant des moyens mis à disposition pour mener à bien l'activité de test en plus de rédiger, implémenter et exécuter les tests de \appliC\\
    \hline
    TROVALLET Romain &\centering Responsable Qualité Test & Organise le déroulement de l'activité de test en plus de rédiger, implémenter et exécuter les tests de \appliC\\
    \hline
    JURET Paul &\centering Responsable C & Se porte garant de la bonne conduite des tests en plus de rédiger, implémenter et exécuter les tests de \appliC\\
    \hline
    CHIRON Paul &\centering Développeur C & Rédige, implémente et exécute les tests de \appliC\\
    \hline
    LETENNEUR Laurent &\centering Développeur C & Rédige, implémente et exécute les tests de \appliC\\
    \hline
    CASSAR Bastien &\centering Responsable Android & Se porte garant de la bonne conduite des tests en plus de rédiger, implémenter et exécuter les tests de \appliA\\
    \hline
    MOULIN Mathis &\centering Développeur Android & Rédige, implémente et exécute les tests de \appliA\\
    \hline
\end{longtable}
\newpage

% Auteur : Camille Constant

\section{Planning prévisionnel}
\label{sec:planning}


Ci-dessous le planning prévisionnel indiquant les phases de réalisations et de test.\\

\begin{figure} [H]
    \centering
    \includegraphics[width=0.9\textwidth]{PLANNING_TEST_A1}
    \label{Planning prévisionnel}
\end{figure}




\newpage

% Auteur : Camille Constant

\section{Validation du document}
\label{sec:validation}

\bigskip


\noindent\begin{tabularx}{\linewidth}{XX}
Document fait à & , le\\[2ex]
Pour la société \client & Pour la société \equipe \\
Mention "Lu et approuvé" : & Mention "Lu et approuvé" : \\[5ex]
Signature(s) : & Signature(s) : \\
\end{tabularx}
\newpage

\end{document}



%--- END 

