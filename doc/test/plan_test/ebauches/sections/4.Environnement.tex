% Auteur : Camille Constant

\section{Environnement de test}
\label{sec:env}

\subsection{Environnement matériel et logiciel de test}
\label{sec:env:env}


%voir avec Hugo mais pas de problèmes normalement

%{\it Matériel et logiciels nécessaires à l'exécution des tests. 

%Exemples : PC, OS, simulateur, réseau (quel wifi ?), BDD, robot composé d'une carte Armadeus sous Linux xx avec capteurs Lego x, y et actionneurs t, z, etc.}

Afin de réaliser les différents tests de validation système, d'intégration et unitaires, l'environnement matériel suivant est mis en place :

\begin{itemize}
    \item La STM32MP15 (Board) est alimentée et connectée en USB à un PC disposant d'un OS Linux permettant d'accéder à distance à cette dernière.
    \item La STM32MP15 déploie sur son Microprocesseur l'OS OpenSTLinux (OSTL, appelé Linux dans la suite du document) (version 4.1)
    \item Est installé sur le Linux de la Board le logiciel SoftSonnette.
    \item Est déployé sur le Microcontrôleur de la Board le logiciel SoftPorte et les libraries provenant de Cube FW Package (version 1.6).
    \item Est mis à disposition du Démonstrateur un smartphone Samsung Galaxy A20e avec Android (version 9.0).
    \item Est installé sur le smartphone l'application AOP.
\end{itemize}

\subsection{Outils de test}
\label{sec:env:outils}

Les tests seront au maximum automatisés grâce aux outils suivants~:
\begin{itemize}
    \item Tests unitaires : Framework de test Android (basé sur JUnit), bouchonnage Mockito, CMocka ;
    \item Tests d'intégration : JMeter ;
    \item Tests de validation : automatisation avec Robot Framework, sinon tests manuels ;
    \item Dossier de test : Squash TM avec intégration de la gestion d'anomalies via Redmine.
\end{itemize}



