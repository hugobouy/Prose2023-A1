% Auteur : Camille Constant

\section{Périmètre de test}
\label{sec:perimetre}

Cette section a pour objet l’élaboration d’un tableau des fonctionnalités et/ou composants / traitements / données du système mentionnant pour chacun l’effort de test à mener.
Cet effort est fonction de la pondération des exigences, risques et criticité retenus.

\subsection{Caractéristiques du projet}
\label{sec:peri:caract}

Le produit \produit~permet au client \client~d'étudier la faisabilité d'un système utilisant la reconnaissance faciale afin de l'utiliser lors de démonstrations à destination de ses différents clients.
L’objectif est de montrer les capacités de la STM32MP15 et de prouver qu'une telle application est possible.\\
Il se décompose en 2 lots/incréments~:
\begin{itemize}
    \item [\textbf{-}] lot 1~: 
        \begin {itemize}
            \item [\textbf{§}] L'application \appliA~doit établir une connexion avec \appliLin, afficher le flux vidéo provenant de la webcam et pouvoir consulter le calendrier d'un employé.
            \item [\textbf{§}] L'application \appliLin~doit pouvoir autoriser ou refuser un Testeur, envoyer le flux vidéo à \appliA~et l'afficher sur son écran.
            \item [\textbf{§}] L'application \appliPo~doit allumer un voyant signalant son bon démarrage.
        \end{itemize}
    \item [\textbf{-}] En plus des fonctionnalités du lot 1, dans le lot 2~: 
        \begin {itemize}
            \item [\textbf{§}] L'application \appliA~doit gérer la liste des employés qu'elle synchronise avec l'application \appliLin, afficher l'état de la Porte et pouvoir se connecter à l'application \appliLin~de manière sécurisée avec un mot de passe.
            \item [\textbf{§}] L'application \appliLin~doit pouvoir gérer l'autorisation d'un Testeur en fonction de ses horaires, stocker les données persistantes des employés et communiquer avec \appliPo.
            \item [\textbf{§}] L'application \appliPo~doit communiquer avec \appliLin~et simuler l'ouverture de la Porte.\\
        \end{itemize}
\end{itemize}

\client~souhaite connaître la qualité globale de \produit~après chaque lot afin d'éventuellement redéfinir chacun des lots. 

%Ce plan de test concerne le niveau de test système pour l'incrément 1. 
Pour information, des tests d’intégration (lots 1 et 2) et des tests unitaires (lot 2) auront été réalisés par \equipe.



\subsection{Éléments à tester}
\label{sec:peri:elements} 


Cette partie s'appuie sur la section 2.2.1 Architecture matérielle et logicielle du document de spécifications \refSpec. 

\medskip


\pagebreak

{\bf Logiciels développés :}\\
\begin{itemize}
    \item [\textbf{-}] Applications C : \appliC
    \item [\textbf{-}] Application Java : \appliA\\
\end{itemize}


{\bf Support d'exécution :}\\
\begin{itemize}
    \item [\textbf{-}] \appliLin~s'exécutera sous OpenSTLinux (OSTL) version 4.1 sur la Board STM32MP15
    \item [\textbf{-}] \appliA~s'exécutera sous Android version 9.0 sur le smartphone Samsung Galaxy A20e\\
\end{itemize}


{\bf Support de communication :}\\
\begin{itemize}
    \item [\textbf{-}] Communication WiFi (Pile TCP/IP)
\end{itemize}


\subsubsection{Éléments concernés par les tests}
\label{sec:peri:comp:test}
          
Cette section désigne ce qui est testé (composant, logiciel, sous-système).\\
          
Seront concernés par l'activité de test les composants logiciels développés durant le projet \projet~: 
\begin{itemize}
    \item \appliLin
    \item \appliPo
    \item \appliA
    \item Communication entre \appliA~et \appliLin
\end{itemize}

\subsubsection{Éléments non concernés par les tests}
\label{sec:peri:comp:nontest}

Cette section désigne ce qui ne va pas être testé (composant, logiciel, sous-système).\\

Ne seront pas concernés par les tests :
\begin{itemize}
    \item Le support logiciel Cube FW Package étant une bibliothèque
    \item Les supports d'exécution logiciels (Linux, Android)
    \item Les supports d'exécution matériels (carte STM32MP15, Samsung Galaxy A20e, Webcam)
    \item Les supports de communication matériels (Borne WiFi, etc.)
    \item Les supports de communication logiciels (Pile TCP/IP)
\end{itemize}

\subsection{Spécifications fonctionnelles ou techniques à tester}
\label{sec:peri:spec}


Cette section s'appuie sur la section 2.3 Fonctions principales développées du document de spécifications \refSpec. 
%Toutes les fonctionnalités ne seront pas testées car ce document se limite à l'incrément 1.\\

\subsubsection{Fonctionnalités à tester}
\label{sec:peri:fct:test}

%{\it Tenir compte des incréments $\Rightarrow$ lors de l'incrément 1, valider en priorité les fonctionnalités basiques permettant de valider la communication dans les 2 sens (C vers A et A vers C).}\\

\medskip

Les fonctionnalités suivantes sont à tester~:
\noindent\begin{longtable}[c]{|p{.38\textwidth}|p{.1\textwidth}|p{.38\textwidth}|}
    \hline
        \bf Fonctionnalité & \bf \centering Priorité & \bf Commentaire\\[-1ex]
                            & \centering (P0 : priorité max) & \\
    \hline
    \endhead
    Initialisation de la Board & \centering P0 & RaS \\
    \hline
    Initialisation de la Board - Webcam non connectée & \centering P1 & Initialisation réussie sans Webcam \\
    \hline
    Établissement de la connexion entre \appliA~et \appliLin & \centering P0 & \appliA~vers \appliLin~et inversement \\
    \hline
    Échec de l'authentification lors de l'établissement de la connexion & \centering P0 & Mot de passe incorrect lors de l'entrée des informations de connexion \\
    \hline
    Échec de l'établissement la connexion entre \appliA~et \appliLin & \centering P1 & \appliA~et \appliLin~ne parvienne pas à communiquer \\
    \hline
    Autorisation du Testeur qui souhaite entrer pendant ses horaires & \centering P0 & Le Testeur est reconnu et y accède lors de ses horaires \\
    \hline
    Refus du Testeur inconnu qui souhaite entrer & \centering P0 & Le Testeur n'est pas reconnu \\
    \hline
    Refus du Testeur reconnu qui souhaite entrer & \centering P0 & Le Testeur est reconnu mais essaie d'entrer hors de ses horaires \\
    \hline
    Ouverture de la Porte et mise à jour de son état & \centering P0 & Demande d'ouverture en local \\
    \hline
    Transmission et affichage du flux vidéo de \appliLin~à \appliA & \centering P0 & RaS \\
    \hline
    Échec de la transmission du flux vidéo de \appliLin~à \appliA & \centering P0 & RaS \\
    \hline
    Consultation du calendrier employé & \centering P0 & L'affichage doit se faire suivant l'employé sélectionné \\
    \hline 
    Ouverture de la Porte à distance et mise à jour de son état & \centering P0 & Demande d'ouverture depuis l'application \\
    \hline
    Consultation de la liste des employés &\centering P0 & Seulement l'affichage des données persistantes, permet la vérification des modifications effectuées \\
    \hline
    Ajout d'un employé standard dans la liste des employés & \centering P0 & RaS \\
    \hline
    Ajout d'un employé spécial dans la liste des employés & \centering P0 & Sélection du rôle "spécial" et de ses horaires personnalisables \\
    \hline
    Suppression d'un employé dans la liste des employés & \centering P0 & RaS \\
    \hline
    Annulation d'une action dans la liste des employés & \centering P2 & RaS \\
    \hline           
    Fermeture \appliA~et \appliLin &\centering P0 & RaS \\
    \hline
\end{longtable}

% Pour plus de détails, voici la matrice de conformité de l'incrément 1 : 
% \href{
% https://docs.google.com/spreadsheets/d/1aaes5xD0STobJdoepx0t9ATB4pgT6jGYcr2hy5ju8Fo/edit#gid=0
% }{ici}

\noindent Pour plus de détails, voici la matrice de conformité du \projet: 
\href{https://docs.google.com/spreadsheets/d/1xq-QIusS_guV91z53ERgtteP4NOb_qmYkGOM_DKjgew/edit?usp=sharing
}{ici}


\subsubsection{Caractéristiques techniques à tester}
\label{sec:peri:tech:test}
%Peut être tout ce qui est lié à des demandes de responsiveness/robustesse
Le cahier des charges de \client~ne contient aucune caractéristique technique à tester.\\

\subsection{Spécifications fonctionnelles ou techniques non testées}
\label{sec:peri:nontest}

%{\it Fonctionnalités non critiques, non prioritaires ou cas d’exception dans CU. \\}

Les fonctionnalités suivantes sont considérées comme non testées car elles représentent une variante d'une fonctionnalité déjà testée au-dessus :

\noindent\begin{longtable}[c]{|p{.35\textwidth}|p{.15\textwidth}|p{.35\textwidth}|}
    \hline
        \bf Fonctionnalité & \bf \centering Priorité & \bf Commentaire\\[-1ex]
                            & (P0 : priorité max) & \\
    \hline
    \endhead
    \hline
    Échec de l'initialisation de la Board & \centering P2 & Cas où \appliPo~ne communique pas avec \appliLin \\
    \hline 
    Abandon de l'établissement de la connexion &\centering P3 & Démonstrateur quitte \appliA \\
    \hline
    Enchaînement de plusieurs Testeurs & \centering P2 & Cas où plusieurs Testeurs veulent rentrer \\
    \hline    
    \appliA~non utilisé &\centering P3 & - Ne pas envoyer de notification vers \appliA~lors de l'ouverture de la Porte. \\
    & & - Pas de fermeture d'\appliA~lors de la fonctionnalité "Fermeture d'\appliA~et \appliLin". \\
    \hline
\end{longtable}

L'ergonomie ainsi que la conformité de l'emplacement des éléments de l'IHM aux spécifications ne sera pas testée. 
L'IHM ne sera validée qu'au travers des tests fonctionnels. 

\subsection{Criticité}
\label{sec:peri:criticite}

Seul l'envoi de données entre \appliLin~et \appliA~est considéré comme critique.

\newpage

\subsection{Risques}
\label{sec:peri:risques}

\noindent Id~: identifiant du risque\\
Description~: description du risque\\
Effet~: effet du risque\\
P~: probabilité (3 – très probable, 2 – probable, 1 – peu probable)\\
I~: impact (3 – impact fort, 2 – impact moyen, 1 – impact faible)\\
EI (élément impacté)~: coût / qualité / délai\\
Action~: description de l’action pour maîtriser le risque

\subsubsection{Risques projet}
\label{sec:peri:risques:projet}

\noindent\begin{longtable}[c]{|p{1.4cm}|p{3cm}|p{3cm}|p{0.4cm}|p{0.4cm}|p{1.2cm}|p{3.5cm}|}
\hline
\bf Id & \bf Intitulé & \bf Effet & \bf \centering P & \bf \centering I & \bf \centering EI & \bf Action\\
\hline
\endhead
RPRJ1 & Pas de test unitaire & Instabilité de l’application lors des tests système & \centering 1 & \centering 2 & \centering C/Q/D & Faire une phase de smoke tests sur l’application avant de réaliser les tests système.\\
\hline
RPRJ2 & Pas de test d’intégration & Instabilité de l’application lors des tests système & \centering 1 & \centering 2 & \centering C/Q/D & Faire une phase de smoke tests sur l’application avant de réaliser les tests système.\\
\hline
RPRJ3 & Pas de test unitaire ou de test d’intégration & Instabilité de l’application lors de tests d’acceptation & \centering 1 & \centering 2 & \centering C/Q/D & Réaliser des tests système sur toutes les fonctionnalités système\\
\hline
RPRJ4 & Problème de disponibilité des intervenants & Dérive dans le planning des tests & \centering 1 & \centering 1 & \centering D & Planifier au plus tôt les actions des différents intervenants\\
\hline
RPRJ5 & Spécifications du produit \produit~non à jour & Déviations entre les spécifications et le système d’où une difficulté pour concevoir des tests pertinents & \centering 2 & \centering 2 & \centering Q & Analyse des spécifications pour identifier des écarts. Poser toutes les questions nécessaires à une bonne compréhension des spécifications.\\
\hline
\end{longtable}

\pagebreak
\subsubsection{Risques produit}
\label{sec:peri:risques:produit}

\noindent\begin{longtable}[c]{|p{1.4cm}|p{3cm}|p{3cm}|p{0.4cm}|p{0.4cm}|p{1.2cm}|p{3.5cm}|}
\hline
\bf Id & \bf Intitulé & \bf Effet & \bf \centering P & \bf \centering I & \bf \centering EI & \bf Action\\
\hline
\endhead
RPRD1 & Mauvaise implémentation de la communication & Système non fonctionnel & \centering 2 & \centering 3 & \centering C/Q/D & Tester la communication en priorité par des tests d'intégration (voir unitaires) avant de faire les tests système.\\
\hline
RPRD2 & Reconnaissance faciale échoue & Système non fonctionnel & \centering 3 & \centering 3 & \centering C/Q/D & Si le script IA ne fonctionne pas, l'équipe \equipe~peut demander l'API de reconnaissance faciale dont dispose \client .\\
\hline
\end{longtable}

\subsection{Effort de test}
\label{sec:peri:effort}

L’effort de test sera priorisé de la façon suivante~:
\begin{itemize}
    \item Phase de \og Smoke test \fg pour vérifier la stabilité de l’application avant de réaliser la campagne de tests fonctionnels système ;
    \item Campagne de tests fonctionnels système selon les priorités ;
    \item Campagne de tests non fonctionnels système par priorité.
\end{itemize}
